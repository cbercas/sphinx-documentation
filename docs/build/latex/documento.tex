%% Generated by Sphinx.
\def\sphinxdocclass{report}
\documentclass[a4paper,10pt,oneside,spanish,openany]{sphinxmanual}
\ifdefined\pdfpxdimen
   \let\sphinxpxdimen\pdfpxdimen\else\newdimen\sphinxpxdimen
\fi \sphinxpxdimen=.75bp\relax
\ifdefined\pdfimageresolution
    \pdfimageresolution= \numexpr \dimexpr1in\relax/\sphinxpxdimen\relax
\fi
%% let collapsible pdf bookmarks panel have high depth per default
\PassOptionsToPackage{bookmarksdepth=5}{hyperref}

\PassOptionsToPackage{booktabs}{sphinx}
\PassOptionsToPackage{colorrows}{sphinx}

\PassOptionsToPackage{warn}{textcomp}
\usepackage[utf8]{inputenc}
\ifdefined\DeclareUnicodeCharacter
% support both utf8 and utf8x syntaxes
  \ifdefined\DeclareUnicodeCharacterAsOptional
    \def\sphinxDUC#1{\DeclareUnicodeCharacter{"#1}}
  \else
    \let\sphinxDUC\DeclareUnicodeCharacter
  \fi
  \sphinxDUC{00A0}{\nobreakspace}
  \sphinxDUC{2500}{\sphinxunichar{2500}}
  \sphinxDUC{2502}{\sphinxunichar{2502}}
  \sphinxDUC{2514}{\sphinxunichar{2514}}
  \sphinxDUC{251C}{\sphinxunichar{251C}}
  \sphinxDUC{2572}{\textbackslash}
\fi
\usepackage{cmap}
\usepackage[T1]{fontenc}
\usepackage{amsmath,amssymb,amstext}
\usepackage{babel}



\usepackage{tgtermes}
\usepackage{tgheros}
\renewcommand{\ttdefault}{txtt}



\usepackage[Sonny]{fncychap}
\ChNameVar{\Large\normalfont\sffamily}
\ChTitleVar{\Large\normalfont\sffamily}
\usepackage{sphinx}

\fvset{fontsize=auto}
\usepackage{geometry}


% Include hyperref last.
\usepackage{hyperref}
% Fix anchor placement for figures with captions.
\usepackage{hypcap}% it must be loaded after hyperref.
% Set up styles of URL: it should be placed after hyperref.
\urlstyle{same}


\usepackage{sphinxmessages}
\setcounter{tocdepth}{1}


        \usepackage{makeidx}
        \makeindex
    

\title{Documento Equipo 5 }
\date{09 de enero de 2025}
\release{1}
\author{Cristina, Pedro, Noah}
\newcommand{\sphinxlogo}{\vbox{}}
\renewcommand{\releasename}{Versión}
\makeindex
\begin{document}

\ifdefined\shorthandoff
  \ifnum\catcode`\=\string=\active\shorthandoff{=}\fi
  \ifnum\catcode`\"=\active\shorthandoff{"}\fi
\fi

\pagestyle{empty}
\sphinxmaketitle
\pagestyle{plain}

\pagestyle{normal}
\phantomsection\label{\detokenize{index::doc}}



\chapter{Indice:}
\label{\detokenize{index:indice}}
\sphinxstepscope


\section{\sphinxstylestrong{Entorno Virtual}}
\label{\detokenize{configuracion_inicial/001.env:entorno-virtual}}\label{\detokenize{configuracion_inicial/001.env::doc}}
\sphinxAtStartPar
El entorno virtual permite aislar las dependencias de un proyecto en Python, evitando conflictos entre diferentes proyectos que puedan requerir distintas versiones de bibliotecas.


\subsection{1. Cómo crear el entorno virtual en el proyecto}
\label{\detokenize{configuracion_inicial/001.env:como-crear-el-entorno-virtual-en-el-proyecto}}
\sphinxAtStartPar
Para crear un entorno virtual, primero debemos abrir una terminal y navegar al directorio del proyecto. Luego, ejecutamos el siguiente comando:

\begin{sphinxVerbatim}[commandchars=\\\{\}]
python\PYG{+w}{ }\PYGZhy{}m\PYG{+w}{ }venv\PYG{+w}{ }nombre\PYGZus{}entorno
\end{sphinxVerbatim}

\sphinxAtStartPar
Reemplazamos \sphinxcode{\sphinxupquote{nombre\_entorno}} por el nombre que deseamos dar al entorno virtual. Esto creará una carpeta en el directorio del proyecto que contendrá el entorno virtual.


\subsection{2. Cómo activar el entorno virtual}
\label{\detokenize{configuracion_inicial/001.env:como-activar-el-entorno-virtual}}
\sphinxAtStartPar
Para activar el entorno virtual, utilizamos los siguientes comandos dependiendo del sistema operativo y ubicándonos en el directorio raíz del proyecto:
\begin{itemize}
\item {} 
\sphinxAtStartPar
\sphinxstylestrong{Windows:}

\begin{sphinxVerbatim}[commandchars=\\\{\}]
.\PYG{l+s+se}{\PYGZbs{}n}ombre\PYGZus{}entorno\PYG{l+s+se}{\PYGZbs{}S}cripts\PYG{l+s+se}{\PYGZbs{}a}ctivate
\end{sphinxVerbatim}

\item {} 
\sphinxAtStartPar
\sphinxstylestrong{Linux/Mac:}

\begin{sphinxVerbatim}[commandchars=\\\{\}]
\PYG{n+nb}{source}\PYG{+w}{ }nombre\PYGZus{}entorno/bin/activate
\end{sphinxVerbatim}

\end{itemize}

\sphinxAtStartPar
Una vez activado, deberías ver el nombre del entorno en el prompt de la terminal, indicando que el entorno virtual está activo.

\sphinxstepscope


\section{\sphinxstylestrong{Instalación de dependencias del proyecto}}
\label{\detokenize{configuracion_inicial/002.instalacion_librerias:instalacion-de-dependencias-del-proyecto}}\label{\detokenize{configuracion_inicial/002.instalacion_librerias::doc}}
\sphinxAtStartPar
Para ejecutar correctamente este proyecto de Python, es necesario instalar las dependencias especificadas en el archivo \sphinxcode{\sphinxupquote{requirements.txt}}. A continuación, se detallan los pasos necesarios para realizar esta instalación de manera efectiva.


\subsection{1. Instalar las dependencias con el archivo requirements.txt}
\label{\detokenize{configuracion_inicial/002.instalacion_librerias:instalar-las-dependencias-con-el-archivo-requirements-txt}}
\sphinxAtStartPar
Con el entorno virtual activado, instala las dependencias ejecutando:

\begin{sphinxVerbatim}[commandchars=\\\{\}]
pip\PYG{+w}{  }install\PYG{+w}{  }\PYGZhy{}r\PYG{+w}{  }requirements.txt\PYG{l+s+sb}{`}
\end{sphinxVerbatim}

\sphinxAtStartPar
Este comando utiliza \sphinxcode{\sphinxupquote{pip}} para leer el archivo \sphinxcode{\sphinxupquote{requirements.txt}} e instala todas las librerías listadas en él.


\subsection{2. Verificar la instalación}
\label{\detokenize{configuracion_inicial/002.instalacion_librerias:verificar-la-instalacion}}
\sphinxAtStartPar
Para asegurarte de que las dependencias se instalaron correctamente, puedes listar los paquetes instalados con:

\begin{sphinxVerbatim}[commandchars=\\\{\}]
pip\PYG{+w}{  }list
\end{sphinxVerbatim}

\sphinxAtStartPar
Esto mostrará una lista de todas las librerías instaladas en tu entorno virtual.


\subsection{3. Desactivar el entorno virtual (opcional)}
\label{\detokenize{configuracion_inicial/002.instalacion_librerias:desactivar-el-entorno-virtual-opcional}}
\sphinxAtStartPar
Cuando hayas terminado de trabajar en el proyecto, puedes desactivar si quieres el entorno virtual con:

\begin{sphinxVerbatim}[commandchars=\\\{\}]
deactivate
\end{sphinxVerbatim}


\subsubsection{Archivo \sphinxstyleliteralintitle{\sphinxupquote{requirements.txt}}}
\label{\detokenize{configuracion_inicial/002.instalacion_librerias:archivo-requirements-txt}}
\sphinxAtStartPar
Nuestro archivo \sphinxcode{\sphinxupquote{requirements.txt}} se vería tal que así:

\begin{sphinxVerbatim}[commandchars=\\\{\}]
\PYG{n+nv}{alabaster}\PYG{o}{=}\PYG{o}{=}\PYG{l+m}{1}.0.0
\PYG{n+nv}{babel}\PYG{o}{=}\PYG{o}{=}\PYG{l+m}{2}.16.0
\PYG{n+nv}{certifi}\PYG{o}{=}\PYG{o}{=}\PYG{l+m}{2024}.8.30
charset\PYGZhy{}normalizer\PYG{o}{=}\PYG{o}{=}\PYG{l+m}{3}.4.0
\PYG{n+nv}{colorama}\PYG{o}{=}\PYG{o}{=}\PYG{l+m}{0}.4.6
\PYG{n+nv}{docutils}\PYG{o}{=}\PYG{o}{=}\PYG{l+m}{0}.21.2
\PYG{n+nv}{idna}\PYG{o}{=}\PYG{o}{=}\PYG{l+m}{3}.10
\PYG{n+nv}{imagesize}\PYG{o}{=}\PYG{o}{=}\PYG{l+m}{1}.4.1
\PYG{n+nv}{Jinja2}\PYG{o}{=}\PYG{o}{=}\PYG{l+m}{3}.1.4
markdown\PYGZhy{}it\PYGZhy{}py\PYG{o}{=}\PYG{o}{=}\PYG{l+m}{3}.0.0
\PYG{n+nv}{MarkupSafe}\PYG{o}{=}\PYG{o}{=}\PYG{l+m}{3}.0.2
mdit\PYGZhy{}py\PYGZhy{}plugins\PYG{o}{=}\PYG{o}{=}\PYG{l+m}{0}.4.2
\PYG{n+nv}{mdurl}\PYG{o}{=}\PYG{o}{=}\PYG{l+m}{0}.1.2
myst\PYGZhy{}parser\PYG{o}{=}\PYG{o}{=}\PYG{l+m}{4}.0.0
\PYG{n+nv}{packaging}\PYG{o}{=}\PYG{o}{=}\PYG{l+m}{24}.2
\PYG{n+nv}{pandoc}\PYG{o}{=}\PYG{o}{=}\PYG{l+m}{2}.4
\PYG{n+nv}{plumbum}\PYG{o}{=}\PYG{o}{=}\PYG{l+m}{1}.9.0
\PYG{n+nv}{ply}\PYG{o}{=}\PYG{o}{=}\PYG{l+m}{3}.11
\PYG{n+nv}{psycopg}\PYG{o}{=}\PYG{o}{=}\PYG{l+m}{3}.2.1
psycopg\PYGZhy{}binary\PYG{o}{=}\PYG{o}{=}\PYG{l+m}{3}.2.1
psycopg\PYGZhy{}pool\PYG{o}{=}\PYG{o}{=}\PYG{l+m}{3}.2.2
\PYG{n+nv}{Pygments}\PYG{o}{=}\PYG{o}{=}\PYG{l+m}{2}.18.0
\PYG{n+nv}{PySide6}\PYG{o}{=}\PYG{o}{=}\PYG{l+m}{6}.7.2
\PYG{n+nv}{PySide6\PYGZus{}Addons}\PYG{o}{=}\PYG{o}{=}\PYG{l+m}{6}.7.2
\PYG{n+nv}{PySide6\PYGZus{}Essentials}\PYG{o}{=}\PYG{o}{=}\PYG{l+m}{6}.7.2
\PYG{n+nv}{pywin32}\PYG{o}{=}\PYG{o}{=}\PYG{l+m}{308}
\PYG{n+nv}{PyYAML}\PYG{o}{=}\PYG{o}{=}\PYG{l+m}{6}.0.2
\PYG{n+nv}{requests}\PYG{o}{=}\PYG{o}{=}\PYG{l+m}{2}.32.3
\PYG{n+nv}{setuptools}\PYG{o}{=}\PYG{o}{=}\PYG{l+m}{72}.1.0
\PYG{n+nv}{shiboken6}\PYG{o}{=}\PYG{o}{=}\PYG{l+m}{6}.7.2
\PYG{n+nv}{snowballstemmer}\PYG{o}{=}\PYG{o}{=}\PYG{l+m}{2}.2.0
\PYG{n+nv}{Sphinx}\PYG{o}{=}\PYG{o}{=}\PYG{l+m}{8}.1.3
sphinx\PYGZhy{}rtd\PYGZhy{}theme\PYG{o}{=}\PYG{o}{=}\PYG{l+m}{3}.0.2
sphinxcontrib\PYGZhy{}applehelp\PYG{o}{=}\PYG{o}{=}\PYG{l+m}{2}.0.0
sphinxcontrib\PYGZhy{}devhelp\PYG{o}{=}\PYG{o}{=}\PYG{l+m}{2}.0.0
sphinxcontrib\PYGZhy{}htmlhelp\PYG{o}{=}\PYG{o}{=}\PYG{l+m}{2}.1.0
sphinxcontrib\PYGZhy{}jquery\PYG{o}{=}\PYG{o}{=}\PYG{l+m}{4}.1
sphinxcontrib\PYGZhy{}jsmath\PYG{o}{=}\PYG{o}{=}\PYG{l+m}{1}.0.1
sphinxcontrib\PYGZhy{}qthelp\PYG{o}{=}\PYG{o}{=}\PYG{l+m}{2}.0.0
sphinxcontrib\PYGZhy{}serializinghtml\PYG{o}{=}\PYG{o}{=}\PYG{l+m}{2}.0.0
\PYG{n+nv}{typing\PYGZus{}extensions}\PYG{o}{=}\PYG{o}{=}\PYG{l+m}{4}.12.2
\PYG{n+nv}{tzdata}\PYG{o}{=}\PYG{o}{=}\PYG{l+m}{2024}.1
\PYG{n+nv}{urllib3}\PYG{o}{=}\PYG{o}{=}\PYG{l+m}{2}.2.3
\PYG{n+nv}{wheel}\PYG{o}{=}\PYG{o}{=}\PYG{l+m}{0}.44.0
\end{sphinxVerbatim}


\subsubsection{Detalles sobre cada una de las dependencias}
\label{\detokenize{configuracion_inicial/002.instalacion_librerias:detalles-sobre-cada-una-de-las-dependencias}}\begin{enumerate}
\sphinxsetlistlabels{\arabic}{enumi}{enumii}{}{.}%
\item {} 
\sphinxAtStartPar
\sphinxstylestrong{alabaster==1.0.0}

\end{enumerate}
\begin{itemize}
\item {} 
\sphinxAtStartPar
\sphinxstylestrong{Descripción:}
\begin{itemize}
\item {} 
\sphinxAtStartPar
Alabaster es un tema de apariencia para \sphinxstylestrong{Sphinx}.

\end{itemize}

\item {} 
\sphinxAtStartPar
\sphinxstylestrong{Propósito:}
\begin{itemize}
\item {} 
\sphinxAtStartPar
Proporciona un diseño limpio y sencillo para la documentación generada con Sphinx. Es el tema por defecto de Sphinx.

\end{itemize}

\end{itemize}


\bigskip\hrule\bigskip

\begin{enumerate}
\sphinxsetlistlabels{\arabic}{enumi}{enumii}{}{.}%
\setcounter{enumi}{1}
\item {} 
\sphinxAtStartPar
\sphinxstylestrong{babel==2.16.0}

\end{enumerate}
\begin{itemize}
\item {} 
\sphinxAtStartPar
\sphinxstylestrong{Descripción:}
\begin{itemize}
\item {} 
\sphinxAtStartPar
Babel es una librería para internacionalización y localización (\sphinxstylestrong{i18n}) de aplicaciones.

\end{itemize}

\item {} 
\sphinxAtStartPar
\sphinxstylestrong{Propósito:}
\begin{itemize}
\item {} 
\sphinxAtStartPar
Facilita la traducción de textos, así como el formateo de fechas, números y otros datos específicos de cada región.

\end{itemize}

\end{itemize}


\bigskip\hrule\bigskip

\begin{enumerate}
\sphinxsetlistlabels{\arabic}{enumi}{enumii}{}{.}%
\setcounter{enumi}{2}
\item {} 
\sphinxAtStartPar
\sphinxstylestrong{certifi==2024.8.30}

\end{enumerate}
\begin{itemize}
\item {} 
\sphinxAtStartPar
\sphinxstylestrong{Descripción:}
\begin{itemize}
\item {} 
\sphinxAtStartPar
Certifi proporciona una lista de certificados de autoridades de certificación (\sphinxstylestrong{CA}) actualizados.

\end{itemize}

\item {} 
\sphinxAtStartPar
\sphinxstylestrong{Propósito:}
\begin{itemize}
\item {} 
\sphinxAtStartPar
Garantiza solicitudes HTTPS seguras verificando la validez de los certificados SSL/TLS.

\end{itemize}

\end{itemize}


\bigskip\hrule\bigskip

\begin{enumerate}
\sphinxsetlistlabels{\arabic}{enumi}{enumii}{}{.}%
\setcounter{enumi}{3}
\item {} 
\sphinxAtStartPar
\sphinxstylestrong{charset\sphinxhyphen{}normalizer==3.4.0}

\end{enumerate}
\begin{itemize}
\item {} 
\sphinxAtStartPar
\sphinxstylestrong{Descripción:}
\begin{itemize}
\item {} 
\sphinxAtStartPar
Librería para detectar y manejar codificaciones de texto.

\end{itemize}

\item {} 
\sphinxAtStartPar
\sphinxstylestrong{Propósito:}
\begin{itemize}
\item {} 
\sphinxAtStartPar
Facilita decodificar correctamente textos con diversas codificaciones, especialmente en respuestas HTTP.

\end{itemize}

\end{itemize}


\bigskip\hrule\bigskip

\begin{enumerate}
\sphinxsetlistlabels{\arabic}{enumi}{enumii}{}{.}%
\setcounter{enumi}{4}
\item {} 
\sphinxAtStartPar
\sphinxstylestrong{colorama==0.4.6}

\end{enumerate}
\begin{itemize}
\item {} 
\sphinxAtStartPar
\sphinxstylestrong{Descripción:}
\begin{itemize}
\item {} 
\sphinxAtStartPar
Proporciona soporte para imprimir texto con color en la consola de \sphinxstylestrong{Windows}.

\end{itemize}

\item {} 
\sphinxAtStartPar
\sphinxstylestrong{Propósito:}
\begin{itemize}
\item {} 
\sphinxAtStartPar
Permite aplicaciones de consola con salida en color en sistemas Windows.

\end{itemize}

\end{itemize}


\bigskip\hrule\bigskip

\begin{enumerate}
\sphinxsetlistlabels{\arabic}{enumi}{enumii}{}{.}%
\setcounter{enumi}{5}
\item {} 
\sphinxAtStartPar
\sphinxstylestrong{docutils==0.21.2}

\end{enumerate}
\begin{itemize}
\item {} 
\sphinxAtStartPar
\sphinxstylestrong{Descripción:}
\begin{itemize}
\item {} 
\sphinxAtStartPar
Librería para procesar documentos en \sphinxstylestrong{reStructuredText}.

\end{itemize}

\item {} 
\sphinxAtStartPar
\sphinxstylestrong{Propósito:}
\begin{itemize}
\item {} 
\sphinxAtStartPar
Fundamental para Sphinx en la generación de documentación en diferentes formatos.

\end{itemize}

\end{itemize}


\bigskip\hrule\bigskip

\begin{enumerate}
\sphinxsetlistlabels{\arabic}{enumi}{enumii}{}{.}%
\setcounter{enumi}{6}
\item {} 
\sphinxAtStartPar
\sphinxstylestrong{idna==3.10}

\end{enumerate}
\begin{itemize}
\item {} 
\sphinxAtStartPar
\sphinxstylestrong{Descripción:}
\begin{itemize}
\item {} 
\sphinxAtStartPar
Implementa el estándar \sphinxstylestrong{IDNA} para nombres de dominio internacionalizados.

\end{itemize}

\item {} 
\sphinxAtStartPar
\sphinxstylestrong{Propósito:}
\begin{itemize}
\item {} 
\sphinxAtStartPar
Permite trabajar con nombres de dominio que contienen caracteres no ASCII.

\end{itemize}

\end{itemize}


\bigskip\hrule\bigskip

\begin{enumerate}
\sphinxsetlistlabels{\arabic}{enumi}{enumii}{}{.}%
\setcounter{enumi}{7}
\item {} 
\sphinxAtStartPar
\sphinxstylestrong{imagesize==1.4.1}

\end{enumerate}
\begin{itemize}
\item {} 
\sphinxAtStartPar
\sphinxstylestrong{Descripción:}
\begin{itemize}
\item {} 
\sphinxAtStartPar
Librería para obtener dimensiones de imágenes sin cargarlas completamente.

\end{itemize}

\item {} 
\sphinxAtStartPar
\sphinxstylestrong{Propósito:}
\begin{itemize}
\item {} 
\sphinxAtStartPar
Utilizada por Sphinx para determinar tamaños de imágenes en la documentación.

\end{itemize}

\end{itemize}


\bigskip\hrule\bigskip

\begin{enumerate}
\sphinxsetlistlabels{\arabic}{enumi}{enumii}{}{.}%
\setcounter{enumi}{8}
\item {} 
\sphinxAtStartPar
\sphinxstylestrong{Jinja2==3.1.4}

\end{enumerate}
\begin{itemize}
\item {} 
\sphinxAtStartPar
\sphinxstylestrong{Descripción:}
\begin{itemize}
\item {} 
\sphinxAtStartPar
Motor de plantillas para Python inspirado en \sphinxstylestrong{Django}.

\end{itemize}

\item {} 
\sphinxAtStartPar
\sphinxstylestrong{Propósito:}
\begin{itemize}
\item {} 
\sphinxAtStartPar
Utilizado por Sphinx para generar HTML dinámico a partir de plantillas.

\end{itemize}

\end{itemize}


\bigskip\hrule\bigskip

\begin{enumerate}
\sphinxsetlistlabels{\arabic}{enumi}{enumii}{}{.}%
\setcounter{enumi}{9}
\item {} 
\sphinxAtStartPar
\sphinxstylestrong{markdown\sphinxhyphen{}it\sphinxhyphen{}py==3.0.0}

\end{enumerate}
\begin{itemize}
\item {} 
\sphinxAtStartPar
\sphinxstylestrong{Descripción:}
\begin{itemize}
\item {} 
\sphinxAtStartPar
Analizador de \sphinxstylestrong{Markdown} escrito en Python.

\end{itemize}

\item {} 
\sphinxAtStartPar
\sphinxstylestrong{Propósito:}
\begin{itemize}
\item {} 
\sphinxAtStartPar
Facilita convertir texto Markdown a HTML en aplicaciones de documentación.

\end{itemize}

\end{itemize}


\bigskip\hrule\bigskip

\begin{enumerate}
\sphinxsetlistlabels{\arabic}{enumi}{enumii}{}{.}%
\setcounter{enumi}{10}
\item {} 
\sphinxAtStartPar
\sphinxstylestrong{MarkupSafe==3.0.2}

\end{enumerate}
\begin{itemize}
\item {} 
\sphinxAtStartPar
\sphinxstylestrong{Descripción:}
\begin{itemize}
\item {} 
\sphinxAtStartPar
Librería para escapar cadenas y evitar inyecciones de código.

\end{itemize}

\item {} 
\sphinxAtStartPar
\sphinxstylestrong{Propósito:}
\begin{itemize}
\item {} 
\sphinxAtStartPar
Utilizada por \sphinxstylestrong{Jinja2} para garantizar seguridad en plantillas.

\end{itemize}

\end{itemize}


\bigskip\hrule\bigskip

\begin{enumerate}
\sphinxsetlistlabels{\arabic}{enumi}{enumii}{}{.}%
\setcounter{enumi}{11}
\item {} 
\sphinxAtStartPar
\sphinxstylestrong{mdit\sphinxhyphen{}py\sphinxhyphen{}plugins==0.4.2}

\end{enumerate}
\begin{itemize}
\item {} 
\sphinxAtStartPar
\sphinxstylestrong{Descripción:}
\begin{itemize}
\item {} 
\sphinxAtStartPar
Plugins adicionales para \sphinxstylestrong{markdown\sphinxhyphen{}it\sphinxhyphen{}py}.

\end{itemize}

\item {} 
\sphinxAtStartPar
\sphinxstylestrong{Propósito:}
\begin{itemize}
\item {} 
\sphinxAtStartPar
Añade funcionalidades como tablas y matemáticas al procesador Markdown.

\end{itemize}

\end{itemize}


\bigskip\hrule\bigskip

\begin{enumerate}
\sphinxsetlistlabels{\arabic}{enumi}{enumii}{}{.}%
\setcounter{enumi}{12}
\item {} 
\sphinxAtStartPar
\sphinxstylestrong{mdurl==0.1.2}

\end{enumerate}
\begin{itemize}
\item {} 
\sphinxAtStartPar
\sphinxstylestrong{Descripción:}
\begin{itemize}
\item {} 
\sphinxAtStartPar
Librería para manejar URLs en documentos Markdown.

\end{itemize}

\item {} 
\sphinxAtStartPar
\sphinxstylestrong{Propósito:}
\begin{itemize}
\item {} 
\sphinxAtStartPar
Facilita el parseo y manipulación de enlaces en texto Markdown.

\end{itemize}

\end{itemize}


\bigskip\hrule\bigskip

\begin{enumerate}
\sphinxsetlistlabels{\arabic}{enumi}{enumii}{}{.}%
\setcounter{enumi}{13}
\item {} 
\sphinxAtStartPar
\sphinxstylestrong{myst\sphinxhyphen{}parser==4.0.0}

\end{enumerate}
\begin{itemize}
\item {} 
\sphinxAtStartPar
\sphinxstylestrong{Descripción:}
\begin{itemize}
\item {} 
\sphinxAtStartPar
Parser de \sphinxstylestrong{Markdown} para Sphinx.

\end{itemize}

\item {} 
\sphinxAtStartPar
\sphinxstylestrong{Propósito:}
\begin{itemize}
\item {} 
\sphinxAtStartPar
Permite escribir documentación en Markdown para proyectos Sphinx.

\end{itemize}

\end{itemize}


\bigskip\hrule\bigskip

\begin{enumerate}
\sphinxsetlistlabels{\arabic}{enumi}{enumii}{}{.}%
\setcounter{enumi}{14}
\item {} 
\sphinxAtStartPar
\sphinxstylestrong{packaging==24.2}

\end{enumerate}
\begin{itemize}
\item {} 
\sphinxAtStartPar
\sphinxstylestrong{Descripción:}
\begin{itemize}
\item {} 
\sphinxAtStartPar
Herramienta para manejar metadatos de paquetes.

\end{itemize}

\item {} 
\sphinxAtStartPar
\sphinxstylestrong{Propósito:}
\begin{itemize}
\item {} 
\sphinxAtStartPar
Facilita trabajar con versiones y dependencias de paquetes Python.

\end{itemize}

\end{itemize}


\bigskip\hrule\bigskip

\begin{enumerate}
\sphinxsetlistlabels{\arabic}{enumi}{enumii}{}{.}%
\setcounter{enumi}{15}
\item {} 
\sphinxAtStartPar
\sphinxstylestrong{pandoc==2.4}

\end{enumerate}
\begin{itemize}
\item {} 
\sphinxAtStartPar
\sphinxstylestrong{Descripción:}
\begin{itemize}
\item {} 
\sphinxAtStartPar
Herramienta universal de conversión de documentos.

\end{itemize}

\item {} 
\sphinxAtStartPar
\sphinxstylestrong{Propósito:}
\begin{itemize}
\item {} 
\sphinxAtStartPar
Convierte documentos entre formatos como Markdown, HTML y LaTeX.

\end{itemize}

\end{itemize}


\bigskip\hrule\bigskip

\begin{enumerate}
\sphinxsetlistlabels{\arabic}{enumi}{enumii}{}{.}%
\setcounter{enumi}{16}
\item {} 
\sphinxAtStartPar
\sphinxstylestrong{plumbum==1.9.0}

\end{enumerate}
\begin{itemize}
\item {} 
\sphinxAtStartPar
\sphinxstylestrong{Descripción:}
\begin{itemize}
\item {} 
\sphinxAtStartPar
Librería para ejecutar comandos del sistema desde Python.

\end{itemize}

\item {} 
\sphinxAtStartPar
\sphinxstylestrong{Propósito:}
\begin{itemize}
\item {} 
\sphinxAtStartPar
Facilita la automatización y ejecución de comandos en scripts Python.

\end{itemize}

\end{itemize}


\bigskip\hrule\bigskip

\begin{enumerate}
\sphinxsetlistlabels{\arabic}{enumi}{enumii}{}{.}%
\setcounter{enumi}{17}
\item {} 
\sphinxAtStartPar
\sphinxstylestrong{ply==3.11}

\end{enumerate}
\begin{itemize}
\item {} 
\sphinxAtStartPar
\sphinxstylestrong{Descripción:}
\begin{itemize}
\item {} 
\sphinxAtStartPar
Implementación de \sphinxstylestrong{Lex} y \sphinxstylestrong{Yacc} en Python.

\end{itemize}

\item {} 
\sphinxAtStartPar
\sphinxstylestrong{Propósito:}
\begin{itemize}
\item {} 
\sphinxAtStartPar
Se usa para crear analizadores léxicos y sintácticos.

\end{itemize}

\end{itemize}


\bigskip\hrule\bigskip

\begin{enumerate}
\sphinxsetlistlabels{\arabic}{enumi}{enumii}{}{.}%
\setcounter{enumi}{18}
\item {} 
\sphinxAtStartPar
\sphinxstylestrong{psycopg==3.2.1}

\end{enumerate}
\begin{itemize}
\item {} 
\sphinxAtStartPar
\sphinxstylestrong{Descripción:}
\begin{itemize}
\item {} 
\sphinxAtStartPar
Cliente para bases de datos \sphinxstylestrong{PostgreSQL}.

\end{itemize}

\item {} 
\sphinxAtStartPar
\sphinxstylestrong{Propósito:}
\begin{itemize}
\item {} 
\sphinxAtStartPar
Facilita ejecutar consultas SQL en bases de datos PostgreSQL.

\end{itemize}

\end{itemize}


\bigskip\hrule\bigskip

\begin{enumerate}
\sphinxsetlistlabels{\arabic}{enumi}{enumii}{}{.}%
\setcounter{enumi}{19}
\item {} 
\sphinxAtStartPar
\sphinxstylestrong{psycopg\sphinxhyphen{}binary==3.2.1}

\end{enumerate}
\begin{itemize}
\item {} 
\sphinxAtStartPar
\sphinxstylestrong{Descripción:}
\begin{itemize}
\item {} 
\sphinxAtStartPar
Versión binaria de \sphinxstylestrong{psycopg}.

\end{itemize}

\item {} 
\sphinxAtStartPar
\sphinxstylestrong{Propósito:}
\begin{itemize}
\item {} 
\sphinxAtStartPar
Permite instalar \sphinxcode{\sphinxupquote{psycopg}} sin necesidad de compilación.

\end{itemize}

\end{itemize}


\bigskip\hrule\bigskip

\begin{enumerate}
\sphinxsetlistlabels{\arabic}{enumi}{enumii}{}{.}%
\setcounter{enumi}{20}
\item {} 
\sphinxAtStartPar
\sphinxstylestrong{psycopg\sphinxhyphen{}pool==3.2.2}

\end{enumerate}
\begin{itemize}
\item {} 
\sphinxAtStartPar
\sphinxstylestrong{Descripción:}
\begin{itemize}
\item {} 
\sphinxAtStartPar
Librería para \sphinxstylestrong{pooling} de conexiones PostgreSQL.

\end{itemize}

\item {} 
\sphinxAtStartPar
\sphinxstylestrong{Propósito:}
\begin{itemize}
\item {} 
\sphinxAtStartPar
Mejora el rendimiento al reutilizar conexiones de base de datos.

\end{itemize}

\end{itemize}


\bigskip\hrule\bigskip

\begin{enumerate}
\sphinxsetlistlabels{\arabic}{enumi}{enumii}{}{.}%
\setcounter{enumi}{21}
\item {} 
\sphinxAtStartPar
\sphinxstylestrong{Pygments==2.18.0}

\end{enumerate}
\begin{itemize}
\item {} 
\sphinxAtStartPar
\sphinxstylestrong{Descripción:}
\begin{itemize}
\item {} 
\sphinxAtStartPar
Librería para resaltar sintaxis de código.

\end{itemize}

\item {} 
\sphinxAtStartPar
\sphinxstylestrong{Propósito:}
\begin{itemize}
\item {} 
\sphinxAtStartPar
Usada por Sphinx para colorear bloques de código.

\end{itemize}

\end{itemize}


\bigskip\hrule\bigskip

\begin{enumerate}
\sphinxsetlistlabels{\arabic}{enumi}{enumii}{}{.}%
\setcounter{enumi}{22}
\item {} 
\sphinxAtStartPar
\sphinxstylestrong{PySide6==6.7.2}

\end{enumerate}
\begin{itemize}
\item {} 
\sphinxAtStartPar
\sphinxstylestrong{Descripción:}
\begin{itemize}
\item {} 
\sphinxAtStartPar
Bindings de \sphinxstylestrong{Qt6} para Python.

\end{itemize}

\item {} 
\sphinxAtStartPar
\sphinxstylestrong{Propósito:}
\begin{itemize}
\item {} 
\sphinxAtStartPar
Permite desarrollar interfaces gráficas con Qt6 en Python.

\end{itemize}

\end{itemize}


\bigskip\hrule\bigskip

\begin{enumerate}
\sphinxsetlistlabels{\arabic}{enumi}{enumii}{}{.}%
\setcounter{enumi}{23}
\item {} 
\sphinxAtStartPar
\sphinxstylestrong{pywin32==308}

\end{enumerate}
\begin{itemize}
\item {} 
\sphinxAtStartPar
\sphinxstylestrong{Descripción:}
\begin{itemize}
\item {} 
\sphinxAtStartPar
Extensiones para trabajar con la API de Windows.

\end{itemize}

\item {} 
\sphinxAtStartPar
\sphinxstylestrong{Propósito:}
\begin{itemize}
\item {} 
\sphinxAtStartPar
Facilita automatización y manipulación de características del sistema Windows.

\end{itemize}

\end{itemize}


\bigskip\hrule\bigskip

\begin{enumerate}
\sphinxsetlistlabels{\arabic}{enumi}{enumii}{}{.}%
\setcounter{enumi}{24}
\item {} 
\sphinxAtStartPar
\sphinxstylestrong{PyYAML==6.0.2}

\end{enumerate}
\begin{itemize}
\item {} 
\sphinxAtStartPar
\sphinxstylestrong{Descripción:}
\begin{itemize}
\item {} 
\sphinxAtStartPar
Librería para leer y escribir archivos \sphinxstylestrong{YAML}.

\end{itemize}

\item {} 
\sphinxAtStartPar
\sphinxstylestrong{Propósito:}
\begin{itemize}
\item {} 
\sphinxAtStartPar
Facilita el manejo de configuraciones en YAML.

\end{itemize}

\end{itemize}


\bigskip\hrule\bigskip

\begin{enumerate}
\sphinxsetlistlabels{\arabic}{enumi}{enumii}{}{.}%
\setcounter{enumi}{25}
\item {} 
\sphinxAtStartPar
\sphinxstylestrong{requests==2.32.3}

\end{enumerate}
\begin{itemize}
\item {} 
\sphinxAtStartPar
\sphinxstylestrong{Descripción:}
\begin{itemize}
\item {} 
\sphinxAtStartPar
Librería para hacer solicitudes HTTP.

\end{itemize}

\item {} 
\sphinxAtStartPar
\sphinxstylestrong{Propósito:}
\begin{itemize}
\item {} 
\sphinxAtStartPar
Simplifica consumir APIs y descargar recursos web.

\end{itemize}

\end{itemize}


\bigskip\hrule\bigskip

\begin{enumerate}
\sphinxsetlistlabels{\arabic}{enumi}{enumii}{}{.}%
\setcounter{enumi}{26}
\item {} 
\sphinxAtStartPar
\sphinxstylestrong{setuptools==72.1.0}

\end{enumerate}
\begin{itemize}
\item {} 
\sphinxAtStartPar
\sphinxstylestrong{Descripción:}
\begin{itemize}
\item {} 
\sphinxAtStartPar
Herramienta para empaquetar proyectos Python.

\end{itemize}

\item {} 
\sphinxAtStartPar
\sphinxstylestrong{Propósito:}
\begin{itemize}
\item {} 
\sphinxAtStartPar
Facilita la distribución e instalación de paquetes.

\end{itemize}

\end{itemize}


\bigskip\hrule\bigskip

\begin{enumerate}
\sphinxsetlistlabels{\arabic}{enumi}{enumii}{}{.}%
\setcounter{enumi}{27}
\item {} 
\sphinxAtStartPar
\sphinxstylestrong{shiboken6==6.7.2}

\end{enumerate}
\begin{itemize}
\item {} 
\sphinxAtStartPar
\sphinxstylestrong{Descripción:}
\begin{itemize}
\item {} 
\sphinxAtStartPar
Generador de bindings para \sphinxstylestrong{PySide6}.

\end{itemize}

\item {} 
\sphinxAtStartPar
\sphinxstylestrong{Propósito:}
\begin{itemize}
\item {} 
\sphinxAtStartPar
Crea enlaces entre C++ y Python.

\end{itemize}

\end{itemize}


\bigskip\hrule\bigskip

\begin{enumerate}
\sphinxsetlistlabels{\arabic}{enumi}{enumii}{}{.}%
\setcounter{enumi}{28}
\item {} 
\sphinxAtStartPar
\sphinxstylestrong{snowballstemmer==2.2.0}

\end{enumerate}
\begin{itemize}
\item {} 
\sphinxAtStartPar
\sphinxstylestrong{Descripción:}
\begin{itemize}
\item {} 
\sphinxAtStartPar
Librería para \sphinxstylestrong{stemming} (reducción de palabras).

\end{itemize}

\item {} 
\sphinxAtStartPar
\sphinxstylestrong{Propósito:}
\begin{itemize}
\item {} 
\sphinxAtStartPar
Mejora la búsqueda en documentación.

\end{itemize}

\end{itemize}


\bigskip\hrule\bigskip

\begin{enumerate}
\sphinxsetlistlabels{\arabic}{enumi}{enumii}{}{.}%
\setcounter{enumi}{29}
\item {} 
\sphinxAtStartPar
\sphinxstylestrong{Sphinx==8.1.3}

\end{enumerate}
\begin{itemize}
\item {} 
\sphinxAtStartPar
\sphinxstylestrong{Descripción:}
\begin{itemize}
\item {} 
\sphinxAtStartPar
Generador de documentación automática.

\end{itemize}

\item {} 
\sphinxAtStartPar
\sphinxstylestrong{Propósito:}
\begin{itemize}
\item {} 
\sphinxAtStartPar
Crea documentación en HTML, PDF, etc.

\end{itemize}

\end{itemize}


\bigskip\hrule\bigskip

\begin{enumerate}
\sphinxsetlistlabels{\arabic}{enumi}{enumii}{}{.}%
\setcounter{enumi}{30}
\item {} 
\sphinxAtStartPar
\sphinxstylestrong{sphinx\sphinxhyphen{}rtd\sphinxhyphen{}theme==3.0.2}

\end{enumerate}
\begin{itemize}
\item {} 
\sphinxAtStartPar
\sphinxstylestrong{Descripción:}
\begin{itemize}
\item {} 
\sphinxAtStartPar
Tema de Sphinx diseñado para \sphinxstylestrong{Read the Docs}.

\end{itemize}

\item {} 
\sphinxAtStartPar
\sphinxstylestrong{Propósito:}
\begin{itemize}
\item {} 
\sphinxAtStartPar
Proporciona una apariencia moderna y fácil de navegar para documentación alojada en \sphinxstylestrong{Read the Docs}.

\end{itemize}

\end{itemize}


\bigskip\hrule\bigskip

\begin{enumerate}
\sphinxsetlistlabels{\arabic}{enumi}{enumii}{}{.}%
\setcounter{enumi}{31}
\item {} 
\sphinxAtStartPar
\sphinxstylestrong{sphinxcontrib\sphinxhyphen{}applehelp==2.0.0}

\end{enumerate}
\begin{itemize}
\item {} 
\sphinxAtStartPar
\sphinxstylestrong{Descripción:}
\begin{itemize}
\item {} 
\sphinxAtStartPar
Extensión de Sphinx para generar documentación en formato \sphinxstylestrong{Apple Help}.

\end{itemize}

\item {} 
\sphinxAtStartPar
\sphinxstylestrong{Propósito:}
\begin{itemize}
\item {} 
\sphinxAtStartPar
Permite crear archivos de ayuda compatibles con aplicaciones de \sphinxstylestrong{macOS}.

\end{itemize}

\end{itemize}


\bigskip\hrule\bigskip

\begin{enumerate}
\sphinxsetlistlabels{\arabic}{enumi}{enumii}{}{.}%
\setcounter{enumi}{32}
\item {} 
\sphinxAtStartPar
\sphinxstylestrong{sphinxcontrib\sphinxhyphen{}devhelp==2.0.0}

\end{enumerate}
\begin{itemize}
\item {} 
\sphinxAtStartPar
\sphinxstylestrong{Descripción:}
\begin{itemize}
\item {} 
\sphinxAtStartPar
Extensión de Sphinx para generar documentación en formato \sphinxstylestrong{Devhelp}.

\end{itemize}

\item {} 
\sphinxAtStartPar
\sphinxstylestrong{Propósito:}
\begin{itemize}
\item {} 
\sphinxAtStartPar
Facilita crear archivos de ayuda compatibles con el visor \sphinxstylestrong{Devhelp} de GNOME.

\end{itemize}

\end{itemize}


\bigskip\hrule\bigskip

\begin{enumerate}
\sphinxsetlistlabels{\arabic}{enumi}{enumii}{}{.}%
\setcounter{enumi}{33}
\item {} 
\sphinxAtStartPar
\sphinxstylestrong{sphinxcontrib\sphinxhyphen{}htmlhelp==2.1.0}

\end{enumerate}
\begin{itemize}
\item {} 
\sphinxAtStartPar
\sphinxstylestrong{Descripción:}
\begin{itemize}
\item {} 
\sphinxAtStartPar
Extensión de Sphinx para generar documentación en formato \sphinxstylestrong{HTML Help}.

\end{itemize}

\item {} 
\sphinxAtStartPar
\sphinxstylestrong{Propósito:}
\begin{itemize}
\item {} 
\sphinxAtStartPar
Permite crear archivos \sphinxcode{\sphinxupquote{.chm}} (HTML compilado) utilizados en sistemas \sphinxstylestrong{Windows}.

\end{itemize}

\end{itemize}


\bigskip\hrule\bigskip

\begin{enumerate}
\sphinxsetlistlabels{\arabic}{enumi}{enumii}{}{.}%
\setcounter{enumi}{34}
\item {} 
\sphinxAtStartPar
\sphinxstylestrong{sphinxcontrib\sphinxhyphen{}jquery==4.1}

\end{enumerate}
\begin{itemize}
\item {} 
\sphinxAtStartPar
\sphinxstylestrong{Descripción:}
\begin{itemize}
\item {} 
\sphinxAtStartPar
Extensión que incluye \sphinxstylestrong{jQuery} en la documentación generada por Sphinx.

\end{itemize}

\item {} 
\sphinxAtStartPar
\sphinxstylestrong{Propósito:}
\begin{itemize}
\item {} 
\sphinxAtStartPar
Asegura la compatibilidad con scripts basados en jQuery en proyectos de documentación.

\end{itemize}

\end{itemize}


\bigskip\hrule\bigskip

\begin{enumerate}
\sphinxsetlistlabels{\arabic}{enumi}{enumii}{}{.}%
\setcounter{enumi}{35}
\item {} 
\sphinxAtStartPar
\sphinxstylestrong{sphinxcontrib\sphinxhyphen{}jsmath==1.0.1}

\end{enumerate}
\begin{itemize}
\item {} 
\sphinxAtStartPar
\sphinxstylestrong{Descripción:}
\begin{itemize}
\item {} 
\sphinxAtStartPar
Extensión de Sphinx para renderizar fórmulas matemáticas usando \sphinxstylestrong{jsMath}.

\end{itemize}

\item {} 
\sphinxAtStartPar
\sphinxstylestrong{Propósito:}
\begin{itemize}
\item {} 
\sphinxAtStartPar
Permite mostrar ecuaciones matemáticas en la documentación HTML sin necesidad de otras herramientas.

\end{itemize}

\end{itemize}


\bigskip\hrule\bigskip

\begin{enumerate}
\sphinxsetlistlabels{\arabic}{enumi}{enumii}{}{.}%
\setcounter{enumi}{36}
\item {} 
\sphinxAtStartPar
\sphinxstylestrong{sphinxcontrib\sphinxhyphen{}qthelp==2.0.0}

\end{enumerate}
\begin{itemize}
\item {} 
\sphinxAtStartPar
\sphinxstylestrong{Descripción:}
\begin{itemize}
\item {} 
\sphinxAtStartPar
Extensión de Sphinx para generar documentación en formato \sphinxstylestrong{Qt Help}.

\end{itemize}

\item {} 
\sphinxAtStartPar
\sphinxstylestrong{Propósito:}
\begin{itemize}
\item {} 
\sphinxAtStartPar
Facilita crear archivos de ayuda compatibles con el visor \sphinxstylestrong{Qt Assistant}.

\end{itemize}

\end{itemize}


\bigskip\hrule\bigskip

\begin{enumerate}
\sphinxsetlistlabels{\arabic}{enumi}{enumii}{}{.}%
\setcounter{enumi}{37}
\item {} 
\sphinxAtStartPar
\sphinxstylestrong{sphinxcontrib\sphinxhyphen{}serializinghtml==2.0.0}

\end{enumerate}
\begin{itemize}
\item {} 
\sphinxAtStartPar
\sphinxstylestrong{Descripción:}
\begin{itemize}
\item {} 
\sphinxAtStartPar
Extensión de Sphinx para generar documentación en formato \sphinxstylestrong{HTML serializado}.

\end{itemize}

\item {} 
\sphinxAtStartPar
\sphinxstylestrong{Propósito:}
\begin{itemize}
\item {} 
\sphinxAtStartPar
Permite exportar la documentación a un formato adecuado para aplicaciones que requieren HTML serializado.

\end{itemize}

\end{itemize}


\bigskip\hrule\bigskip

\begin{enumerate}
\sphinxsetlistlabels{\arabic}{enumi}{enumii}{}{.}%
\setcounter{enumi}{38}
\item {} 
\sphinxAtStartPar
\sphinxstylestrong{typing\_extensions==4.12.2}

\end{enumerate}
\begin{itemize}
\item {} 
\sphinxAtStartPar
\sphinxstylestrong{Descripción:}
\begin{itemize}
\item {} 
\sphinxAtStartPar
Proporciona funcionalidades de anotaciones de tipo avanzadas para Python.

\end{itemize}

\item {} 
\sphinxAtStartPar
\sphinxstylestrong{Propósito:}
\begin{itemize}
\item {} 
\sphinxAtStartPar
Permite usar nuevas características de tipado en versiones antiguas de Python.

\end{itemize}

\end{itemize}


\bigskip\hrule\bigskip

\begin{enumerate}
\sphinxsetlistlabels{\arabic}{enumi}{enumii}{}{.}%
\setcounter{enumi}{39}
\item {} 
\sphinxAtStartPar
\sphinxstylestrong{tzdata==2024.1}

\end{enumerate}
\begin{itemize}
\item {} 
\sphinxAtStartPar
\sphinxstylestrong{Descripción:}
\begin{itemize}
\item {} 
\sphinxAtStartPar
Base de datos de zonas horarias globales.

\end{itemize}

\item {} 
\sphinxAtStartPar
\sphinxstylestrong{Propósito:}
\begin{itemize}
\item {} 
\sphinxAtStartPar
Permite manejar fechas y horas con información actualizada de zonas horarias.

\end{itemize}

\end{itemize}


\bigskip\hrule\bigskip

\begin{enumerate}
\sphinxsetlistlabels{\arabic}{enumi}{enumii}{}{.}%
\setcounter{enumi}{40}
\item {} 
\sphinxAtStartPar
\sphinxstylestrong{urllib3==2.2.3}

\end{enumerate}
\begin{itemize}
\item {} 
\sphinxAtStartPar
\sphinxstylestrong{Descripción:}
\begin{itemize}
\item {} 
\sphinxAtStartPar
Librería para manejar conexiones HTTP con funciones avanzadas.

\end{itemize}

\item {} 
\sphinxAtStartPar
\sphinxstylestrong{Propósito:}
\begin{itemize}
\item {} 
\sphinxAtStartPar
Facilita realizar solicitudes HTTP seguras y eficientes. Es una dependencia clave de \sphinxstylestrong{requests}.

\end{itemize}

\end{itemize}


\bigskip\hrule\bigskip

\begin{enumerate}
\sphinxsetlistlabels{\arabic}{enumi}{enumii}{}{.}%
\setcounter{enumi}{41}
\item {} 
\sphinxAtStartPar
\sphinxstylestrong{wheel==0.44.0}

\end{enumerate}
\begin{itemize}
\item {} 
\sphinxAtStartPar
\sphinxstylestrong{Descripción:}
\begin{itemize}
\item {} 
\sphinxAtStartPar
Herramienta para construir y empaquetar distribuciones en formato \sphinxstylestrong{Wheel} (\sphinxcode{\sphinxupquote{.whl}}).

\end{itemize}

\item {} 
\sphinxAtStartPar
\sphinxstylestrong{Propósito:}
\begin{itemize}
\item {} 
\sphinxAtStartPar
Mejora el proceso de instalación de paquetes Python al ofrecer un formato más rápido y eficiente.

\end{itemize}

\end{itemize}

\sphinxstepscope


\section{\sphinxstylestrong{Creación Inicial del Proyecto con Sphinx}}
\label{\detokenize{configuracion_inicial/003.Creacion_proyecto_Sphinx:creacion-inicial-del-proyecto-con-sphinx}}\label{\detokenize{configuracion_inicial/003.Creacion_proyecto_Sphinx::doc}}
\sphinxAtStartPar
A continuación, vamos a detallar todo el proceso para la creación inicial de un proyecto con Sphinx, explicando los comandos, las opciones necesarias y cómo configurarlo para usar \sphinxstylestrong{Markdown} con \sphinxstylestrong{MyST\sphinxhyphen{}Parser}.


\subsection{1. Instalación de Sphinx y MyST\sphinxhyphen{}Parser}
\label{\detokenize{configuracion_inicial/003.Creacion_proyecto_Sphinx:instalacion-de-sphinx-y-myst-parser}}
\sphinxAtStartPar
Con el entorno virtual activado, procedemos a instalar Sphinx, antes de empezar, hay que asegurarse de tener Sphinx y MyST\sphinxhyphen{}Parser instalados en el entorno. Se pueden instalar con \sphinxcode{\sphinxupquote{pip}}:

\begin{sphinxVerbatim}[commandchars=\\\{\}]
pip\PYG{+w}{ }install\PYG{+w}{ }sphinx\PYG{+w}{ }myst\PYGZhy{}parser
\end{sphinxVerbatim}
\begin{itemize}
\item {} 
\sphinxAtStartPar
\sphinxstylestrong{Sphinx}: Es la herramienta principal para generar documentación.

\item {} 
\sphinxAtStartPar
\sphinxstylestrong{MyST\sphinxhyphen{}Parser}: Permite usar archivos Markdown (\sphinxcode{\sphinxupquote{.md}}) en lugar de \sphinxcode{\sphinxupquote{reStructuredText}} (\sphinxcode{\sphinxupquote{.rst}}).

\end{itemize}


\bigskip\hrule\bigskip



\subsection{2. Crear un Proyecto con Sphinx}
\label{\detokenize{configuracion_inicial/003.Creacion_proyecto_Sphinx:crear-un-proyecto-con-sphinx}}
\sphinxAtStartPar
\sphinxcode{\sphinxupquote{sphinx\sphinxhyphen{}quickstart}} es la herramienta que utilizamos para inicializar un proyecto de documentación con Sphinx.
Esta aplicación interactiva genera la estructura base del proyecto, incluyendo los directorios necesarios, el archivo de configuración (\sphinxcode{\sphinxupquote{conf.py}}) y un índice inicial (\sphinxcode{\sphinxupquote{index.rst}} o \sphinxcode{\sphinxupquote{index.md}}).
Se ejecuta el comando para iniciar un nuevo proyecto de Sphinx:

\begin{sphinxVerbatim}[commandchars=\\\{\}]
sphinx\PYGZhy{}quickstart
\end{sphinxVerbatim}
\begin{itemize}
\item {} 
\sphinxAtStartPar
Este comando genera una estructura básica dentro de la carpeta \sphinxcode{\sphinxupquote{docs/}}.

\end{itemize}


\subsubsection{Opciones Importantes en el Proceso:}
\label{\detokenize{configuracion_inicial/003.Creacion_proyecto_Sphinx:opciones-importantes-en-el-proceso}}
\sphinxAtStartPar
Al ejecutar \sphinxcode{\sphinxupquote{sphinx\sphinxhyphen{}quickstart}}, se deben responder algunas preguntas:
\begin{enumerate}
\sphinxsetlistlabels{\arabic}{enumi}{enumii}{}{.}%
\item {} 
\sphinxAtStartPar
\sphinxstylestrong{Directorio donde se guardará la documentación}:
\begin{itemize}
\item {} 
\sphinxAtStartPar
Por defecto: \sphinxcode{\sphinxupquote{docs/}}.

\end{itemize}

\item {} 
\sphinxAtStartPar
\sphinxstylestrong{Separar archivos fuente y archivos compilados (build)}:
\begin{itemize}
\item {} 
\sphinxAtStartPar
Se responde \sphinxcode{\sphinxupquote{yes}}.

\item {} 
\sphinxAtStartPar
Esto crea una estructura con dos carpetas: \sphinxcode{\sphinxupquote{source/}} (archivos fuente) y \sphinxcode{\sphinxupquote{build/}} (archivos generados).

\end{itemize}

\item {} 
\sphinxAtStartPar
\sphinxstylestrong{Nombre del proyecto}: Introduce el nombre del proyecto.

\item {} 
\sphinxAtStartPar
\sphinxstylestrong{Nombre del autor}: Se escribe el autor del proyecto.

\item {} 
\sphinxAtStartPar
\sphinxstylestrong{Idioma}:  Para trabajar en español se escribe \sphinxcode{\sphinxupquote{es}}.

\end{enumerate}


\bigskip\hrule\bigskip



\subsection{3. Estructura del Proyecto}
\label{\detokenize{configuracion_inicial/003.Creacion_proyecto_Sphinx:estructura-del-proyecto}}
\sphinxAtStartPar
Tras ejecutar \sphinxcode{\sphinxupquote{sphinx\sphinxhyphen{}quickstart}}, se creara una estructura básica como esta:

\begin{sphinxVerbatim}[commandchars=\\\{\}]
docs/
├── build/                  \PYGZsh{} Archivos generados por Sphinx (HTML, PDF, etc.)
├── source/                 \PYGZsh{} Archivos fuente de la documentación
│   ├── \PYGZus{}static/            \PYGZsh{} Archivos estáticos (CSS, imágenes, etc.)
│   ├── \PYGZus{}templates/         \PYGZsh{} Plantillas HTML personalizadas
│   ├── conf.py             \PYGZsh{} Archivo de configuración principal
│   ├── index.md           \PYGZsh{} Archivo raíz del proyecto (puedes reemplazarlo por Markdown)
├── Makefile                \PYGZsh{} Comandos para compilar la documentación (Linux/macOS)
└── make.bat                \PYGZsh{} Comandos para compilar la documentación (Windows)
\end{sphinxVerbatim}


\bigskip\hrule\bigskip



\subsection{4. Configuración para Usar Markdown con MyST\sphinxhyphen{}Parser}
\label{\detokenize{configuracion_inicial/003.Creacion_proyecto_Sphinx:configuracion-para-usar-markdown-con-myst-parser}}
\sphinxAtStartPar
Para habilitar \sphinxstylestrong{Markdown} en lugar de \sphinxcode{\sphinxupquote{reStructuredText}}, se realizan los siguientes cambios:


\subsubsection{4.1. Instalación de MyST\sphinxhyphen{}Parser}
\label{\detokenize{configuracion_inicial/003.Creacion_proyecto_Sphinx:instalacion-de-myst-parser}}
\sphinxAtStartPar
Hay que asegurarse de que \sphinxcode{\sphinxupquote{myst\sphinxhyphen{}parser}} está instalado:

\begin{sphinxVerbatim}[commandchars=\\\{\}]
pip\PYG{+w}{ }install\PYG{+w}{ }myst\PYGZhy{}parser
\end{sphinxVerbatim}


\subsubsection{4.2. Modificación del Archivo \sphinxstyleliteralintitle{\sphinxupquote{conf.py}}}
\label{\detokenize{configuracion_inicial/003.Creacion_proyecto_Sphinx:modificacion-del-archivo-conf-py}}
\sphinxAtStartPar
Se edita el archivo \sphinxcode{\sphinxupquote{docs/source/conf.py}} y se añaden estas configuraciones:

\begin{sphinxVerbatim}[commandchars=\\\{\}]
\PYG{c+c1}{\PYGZsh{} Se habilitarán las extensiones necesarias}
\PYG{n}{extensions} \PYG{o}{=} \PYG{p}{[}
    \PYG{l+s+s1}{\PYGZsq{}}\PYG{l+s+s1}{myst\PYGZus{}parser}\PYG{l+s+s1}{\PYGZsq{}}\PYG{p}{,}  \PYG{c+c1}{\PYGZsh{} Habilita soporte para archivos Markdown}
\PYG{p}{]}
\end{sphinxVerbatim}


\subsubsection{4.2.1 Especificación de las extensiones de archivo fuente}
\label{\detokenize{configuracion_inicial/003.Creacion_proyecto_Sphinx:especificacion-de-las-extensiones-de-archivo-fuente}}
\begin{sphinxVerbatim}[commandchars=\\\{\}]
\PYG{n}{source\PYGZus{}suffix} \PYG{o}{=} \PYG{p}{\PYGZob{}}
    \PYG{l+s+s1}{\PYGZsq{}}\PYG{l+s+s1}{.rst}\PYG{l+s+s1}{\PYGZsq{}}\PYG{p}{:} \PYG{l+s+s1}{\PYGZsq{}}\PYG{l+s+s1}{restructuredtext}\PYG{l+s+s1}{\PYGZsq{}}\PYG{p}{,}  \PYG{c+c1}{\PYGZsh{} Opcional, si se usan archivos .rst}
    \PYG{l+s+s1}{\PYGZsq{}}\PYG{l+s+s1}{.md}\PYG{l+s+s1}{\PYGZsq{}}\PYG{p}{:} \PYG{l+s+s1}{\PYGZsq{}}\PYG{l+s+s1}{markdown}\PYG{l+s+s1}{\PYGZsq{}}\PYG{p}{,}  \PYG{c+c1}{\PYGZsh{} Soporte para archivos Markdown}
\PYG{p}{\PYGZcb{}}
\end{sphinxVerbatim}

\sphinxAtStartPar
Con esta configuración, Sphinx será capaz de procesar tanto archivos \sphinxcode{\sphinxupquote{.rst}} como \sphinxcode{\sphinxupquote{.md}}, facilitando la integración de diferentes formatos de contenido.


\subsubsection{4.2.2 Configuración del idioma}
\label{\detokenize{configuracion_inicial/003.Creacion_proyecto_Sphinx:configuracion-del-idioma}}
\begin{sphinxVerbatim}[commandchars=\\\{\}]
\PYG{n}{language} \PYG{o}{=} \PYG{l+s+s1}{\PYGZsq{}}\PYG{l+s+s1}{es}\PYG{l+s+s1}{\PYGZsq{}}  \PYG{c+c1}{\PYGZsh{} Idioma seleccionado}
\end{sphinxVerbatim}


\subsubsection{4.2.3 Configuración para archivos estáticos (CSS e imágenes)}
\label{\detokenize{configuracion_inicial/003.Creacion_proyecto_Sphinx:configuracion-para-archivos-estaticos-css-e-imagenes}}
\begin{sphinxVerbatim}[commandchars=\\\{\}]
html\PYGZus{}static\PYGZus{}path = [\PYGZsq{}\PYGZus{}static\PYGZsq{}]` 
\end{sphinxVerbatim}


\bigskip\hrule\bigskip



\subsection{5. Creación de la Documentación en Markdown}
\label{\detokenize{configuracion_inicial/003.Creacion_proyecto_Sphinx:creacion-de-la-documentacion-en-markdown}}
\sphinxAtStartPar
Se sustituye el archivo \sphinxcode{\sphinxupquote{index.rst}} por un archivo \sphinxcode{\sphinxupquote{index.md}}.


\bigskip\hrule\bigskip



\subsection{6. Compilación de la Documentación}
\label{\detokenize{configuracion_inicial/003.Creacion_proyecto_Sphinx:compilacion-de-la-documentacion}}
\sphinxAtStartPar
Para generar los archivos HTML, utiliza el comando:

\begin{sphinxVerbatim}[commandchars=\\\{\}]
\PYG{n}{make} \PYG{n}{html}
\end{sphinxVerbatim}
\begin{itemize}
\item {} 
\sphinxAtStartPar
Esto crea la salida HTML en \sphinxcode{\sphinxupquote{docs/build/html}}.

\item {} 
\sphinxAtStartPar
Se abre \sphinxcode{\sphinxupquote{index.html}} en el navegador para ver el resultado con el siguiente comando:

\end{itemize}

\begin{sphinxVerbatim}[commandchars=\\\{\}]
\PYG{n}{start} \PYG{o}{.}\PYGZbs{}\PYG{n}{build}\PYGZbs{}\PYG{n}{html}\PYGZbs{}\PYG{n}{index}\PYG{o}{.}\PYG{n}{html}
\end{sphinxVerbatim}


\bigskip\hrule\bigskip



\subsection{7. Personalización Adicional}
\label{\detokenize{configuracion_inicial/003.Creacion_proyecto_Sphinx:personalizacion-adicional}}

\subsubsection{7.1. Estilo del Código}
\label{\detokenize{configuracion_inicial/003.Creacion_proyecto_Sphinx:estilo-del-codigo}}
\sphinxAtStartPar
Se puede personalizar el estilo de resaltado de sintaxis con CSS en la carpeta \sphinxcode{\sphinxupquote{\_static/}}:
\begin{enumerate}
\sphinxsetlistlabels{\arabic}{enumi}{enumii}{}{.}%
\item {} 
\sphinxAtStartPar
Se crea un archivo CSS (\sphinxcode{\sphinxupquote{custom.css}}) en \sphinxcode{\sphinxupquote{docs/source/\_static/}}.

\item {} 
\sphinxAtStartPar
Se añade a \sphinxcode{\sphinxupquote{conf.py}}:

\end{enumerate}

\begin{sphinxVerbatim}[commandchars=\\\{\}]
\PYG{n}{html\PYGZus{}css\PYGZus{}files} \PYG{o}{=} \PYG{p}{[}\PYG{l+s+s1}{\PYGZsq{}}\PYG{l+s+s1}{custom.css}\PYG{l+s+s1}{\PYGZsq{}}\PYG{p}{]}
\end{sphinxVerbatim}

\sphinxAtStartPar
Ejemplo básico de CSS para el resaltado de código:

\begin{sphinxVerbatim}[commandchars=\\\{\}]
\PYG{n}{pre} \PYG{p}{\PYGZob{}}
    \PYG{n}{background}\PYG{o}{\PYGZhy{}}\PYG{n}{color}\PYG{p}{:} \PYG{c+c1}{\PYGZsh{}2d2d2d;}
    \PYG{n}{color}\PYG{p}{:} \PYG{c+c1}{\PYGZsh{}f8f8f2;}
    \PYG{n}{padding}\PYG{p}{:} \PYG{l+m+mi}{10}\PYG{n}{px}\PYG{p}{;}
    \PYG{n}{border}\PYG{o}{\PYGZhy{}}\PYG{n}{radius}\PYG{p}{:} \PYG{l+m+mi}{5}\PYG{n}{px}\PYG{p}{;}
    \PYG{n}{overflow}\PYG{o}{\PYGZhy{}}\PYG{n}{x}\PYG{p}{:} \PYG{n}{auto}\PYG{p}{;}
    \PYG{n}{font}\PYG{o}{\PYGZhy{}}\PYG{n}{family}\PYG{p}{:} \PYG{l+s+s2}{\PYGZdq{}}\PYG{l+s+s2}{Courier New}\PYG{l+s+s2}{\PYGZdq{}}\PYG{p}{,} \PYG{n}{Courier}\PYG{p}{,} \PYG{n}{monospace}\PYG{p}{;}
\PYG{p}{\PYGZcb{}}
\end{sphinxVerbatim}


\bigskip\hrule\bigskip


\sphinxstepscope


\section{\sphinxstylestrong{Estructura inicial del proyecto}}
\label{\detokenize{configuracion_inicial/004.estructura_inicial_proyecto:estructura-inicial-del-proyecto}}\label{\detokenize{configuracion_inicial/004.estructura_inicial_proyecto::doc}}
\sphinxAtStartPar
Cuando creas un proyecto Sphinx utilizando \sphinxcode{\sphinxupquote{sphinx\sphinxhyphen{}quickstart}}, se generan varias carpetas y archivos para estructurar tu documentación. Si se trabajan con \sphinxstylestrong{MyST\sphinxhyphen{}Parser} para usar Markdown (\sphinxcode{\sphinxupquote{.md}}) en lugar de ReStructuredText (\sphinxcode{\sphinxupquote{.rst}}), estas carpetas seguirán siendo útiles, pero la estructura se adaptara ligeramente. A continuación se muestra una descripción detallada del propósito de cada carpeta y archivo:


\bigskip\hrule\bigskip



\subsection{1. Estructura Generada}
\label{\detokenize{configuracion_inicial/004.estructura_inicial_proyecto:estructura-generada}}
\begin{sphinxVerbatim}[commandchars=\\\{\}]
docs/
├── build/
├── source/
│   ├── \PYGZus{}static/
│   ├── \PYGZus{}templates/
│   ├── conf.py
│   ├── index.rst
├── make.bat
└── Makefile
\end{sphinxVerbatim}


\bigskip\hrule\bigskip



\subsubsection{1.1. \sphinxstylestrong{Carpeta \sphinxstyleliteralintitle{\sphinxupquote{build/}}}}
\label{\detokenize{configuracion_inicial/004.estructura_inicial_proyecto:carpeta-build}}\begin{itemize}
\item {} 
\sphinxAtStartPar
Carpeta donde se almacenan los archivos generados por Sphinx en los distintos formatos de salida, como HTML, PDF o LaTeX, después de ejecutar el proceso de construcción (make html, make pdf, etc.).

\item {} 
\sphinxAtStartPar
\sphinxstylestrong{Subcarpetas}:
\begin{itemize}
\item {} 
\sphinxAtStartPar
\sphinxcode{\sphinxupquote{html/}}: Archivos HTML generados.

\item {} 
\sphinxAtStartPar
\sphinxcode{\sphinxupquote{doctrees/}}: Archivos intermedios que Sphinx usa para rastrear dependencias entre documentos.

\item {} 
\sphinxAtStartPar
\sphinxcode{\sphinxupquote{latex/}}: Archivos LaTeX generados (si se habilita la salida PDF).

\end{itemize}

\end{itemize}

\begin{sphinxadmonition}{note}{Nota}

\sphinxAtStartPar
Esta carpeta se regenera cada vez que se ejecuta el comando \sphinxcode{\sphinxupquote{make html}}.
\end{sphinxadmonition}


\bigskip\hrule\bigskip



\subsubsection{1.2. \sphinxstylestrong{Carpeta \sphinxstyleliteralintitle{\sphinxupquote{source/}}}}
\label{\detokenize{configuracion_inicial/004.estructura_inicial_proyecto:carpeta-source}}
\sphinxAtStartPar
Carpeta principal que contiene los archivos fuente de la documentación. Aquí se encuentran los archivos que definimos y editamos, como los documentos en formato reStructuredText (\sphinxcode{\sphinxupquote{.rst}}) y las configuraciones adicionales.


\paragraph{\sphinxstylestrong{1.2.1. Subcarpetas Importantes}}
\label{\detokenize{configuracion_inicial/004.estructura_inicial_proyecto:subcarpetas-importantes}}\begin{itemize}
\item {} 
\sphinxAtStartPar
\sphinxstylestrong{\sphinxcode{\sphinxupquote{\_static/}}}:
\begin{itemize}
\item {} 
\sphinxAtStartPar
Subcarpeta dentro de source reservada para incluir recursos estáticos, como imágenes, archivos CSS personalizados, scripts JavaScript, entre otros.

\item {} 
\sphinxAtStartPar
Aquí irán el archivo \sphinxcode{\sphinxupquote{custom.css}} y cualquier imagen que usemos en la documentación.

\item {} 
\sphinxAtStartPar
Ejemplo:

\begin{sphinxVerbatim}[commandchars=\\\{\}]
docs/source/\PYGZus{}static/
├── images/
│   ├── imagen1.png
└── custom.css 
\end{sphinxVerbatim}

\end{itemize}

\item {} 
\sphinxAtStartPar
\sphinxstylestrong{\sphinxcode{\sphinxupquote{\_templates/}}}:
\begin{itemize}
\item {} 
\sphinxAtStartPar
Carpeta utilizada para personalizar las plantillas HTML de la documentación. Es útil si queremos modificar el diseño o la estructura de las páginas generadas.

\item {} 
\sphinxAtStartPar
Se pueden modificar las plantillas HTML que Sphinx usa (como el diseño de encabezados y pies de página).

\end{itemize}

\end{itemize}


\bigskip\hrule\bigskip



\paragraph{\sphinxstylestrong{1.2.2. Archivos Importantes en \sphinxstyleliteralintitle{\sphinxupquote{source/}}}}
\label{\detokenize{configuracion_inicial/004.estructura_inicial_proyecto:archivos-importantes-en-source}}\begin{itemize}
\item {} 
\sphinxAtStartPar
\sphinxstylestrong{\sphinxcode{\sphinxupquote{conf.py}}}:
\begin{itemize}
\item {} 
\sphinxAtStartPar
El archivo de configuración principal de Sphinx.

\item {} 
\sphinxAtStartPar
Contiene ajustes para extensiones (como \sphinxcode{\sphinxupquote{myst\sphinxhyphen{}parser}}), rutas de búsqueda, tema visual (\sphinxcode{\sphinxupquote{html\_theme}}) etc.

\item {} 
\sphinxAtStartPar
Al usar MyST\sphinxhyphen{}Parser, hay que asegurarse de incluir:

\begin{sphinxVerbatim}[commandchars=\\\{\}]
\PYG{n}{extensions} \PYG{o}{=} \PYG{p}{[}\PYG{l+s+s1}{\PYGZsq{}}\PYG{l+s+s1}{myst\PYGZus{}parser}\PYG{l+s+s1}{\PYGZsq{}}\PYG{p}{]}
\PYG{n}{source\PYGZus{}suffix} \PYG{o}{=} \PYG{p}{\PYGZob{}}
    \PYG{l+s+s1}{\PYGZsq{}}\PYG{l+s+s1}{.md}\PYG{l+s+s1}{\PYGZsq{}}\PYG{p}{:} \PYG{l+s+s1}{\PYGZsq{}}\PYG{l+s+s1}{markdown}\PYG{l+s+s1}{\PYGZsq{}}\PYG{p}{,}
\PYG{p}{\PYGZcb{}}
\end{sphinxVerbatim}

\end{itemize}

\item {} 
\sphinxAtStartPar
\sphinxstylestrong{\sphinxcode{\sphinxupquote{index.md}}}:
\begin{itemize}
\item {} 
\sphinxAtStartPar
El archivo raíz de la documentación.

\item {} 
\sphinxAtStartPar
Define el índice inicial y vincula los archivos fuente usando la directiva \sphinxcode{\sphinxupquote{toctree}}.

\item {} 
\sphinxAtStartPar
Al usar Markdown con MyST, el archivo se ve como:

\begin{sphinxVerbatim}[commandchars=\\\{\}]
\PYG{c+c1}{\PYGZsh{} Título Principal de la Documentación}

\PYG{p}{\PYGZob{}}\PYG{n}{toctree}\PYG{p}{\PYGZcb{}}
\PYG{p}{:}\PYG{n}{maxdepth}\PYG{p}{:} \PYG{l+m+mi}{2}
\PYG{p}{:}\PYG{n}{caption}\PYG{p}{:} \PYG{n}{Índice}\PYG{p}{:}

\PYG{n}{archivo1}
\PYG{n}{carpeta}\PYG{o}{/}\PYG{n}{archivo2}
\end{sphinxVerbatim}

\item {} 
\sphinxAtStartPar
\sphinxcode{\sphinxupquote{\#}}: Indica un \sphinxstylestrong{título de nivel 1} en Markdown. En este caso, es el título principal de la documentación.

\item {} 
\sphinxAtStartPar
\sphinxcode{\sphinxupquote{\#\#}}: Indica un \sphinxstylestrong{título de nivel 2}. Aquí es un subtítulo para la sección del índice.

\item {} \subsubsection*{\sphinxstylestrong{Directiva \sphinxstyleliteralintitle{\sphinxupquote{\{toctree\}}}}}

\sphinxAtStartPar
La directiva \sphinxcode{\sphinxupquote{\{toctree\}}} es una característica especial de \sphinxstylestrong{MyST\sphinxhyphen{}Parser} y \sphinxstylestrong{Sphinx} que permite construir un índice o tabla de contenidos (ToC).

\item {} \subsubsection*{\sphinxstylestrong{Propósito de \sphinxstyleliteralintitle{\sphinxupquote{\{toctree\}}}}:}
\begin{itemize}
\item {} 
\sphinxAtStartPar
Crea un árbol de navegación que incluye enlaces a otros documentos o secciones.

\item {} 
\sphinxAtStartPar
Sphinx utiliza este árbol para generar automáticamente la estructura de navegación en los archivos HTML generados.

\item {} \subsubsection*{\sphinxstylestrong{Parámetros de \sphinxstyleliteralintitle{\sphinxupquote{\{toctree\}}}:}}
\begin{enumerate}
\sphinxsetlistlabels{\arabic}{enumi}{enumii}{}{.}%
\item {} 
\sphinxAtStartPar
\sphinxstylestrong{\sphinxcode{\sphinxupquote{:maxdepth: 6}}}
\begin{itemize}
\item {} 
\sphinxAtStartPar
Especifica el nivel de profundidad para los encabezados que se incluirán en la tabla de contenidos.

\item {} 
\sphinxAtStartPar
\sphinxcode{\sphinxupquote{6}} es el máximo, lo que significa que Sphinx incluirá encabezados desde \sphinxcode{\sphinxupquote{\#}} (nivel 1) hasta \sphinxcode{\sphinxupquote{\#\#\#\#\#\#}} (nivel 6) de los documentos referenciados.

\end{itemize}

\sphinxAtStartPar
Ejemplo:
\begin{itemize}
\item {} 
\sphinxAtStartPar
Si un archivo tiene:

\begin{sphinxVerbatim}[commandchars=\\\{\}]
\PYG{c+c1}{\PYGZsh{} Título de Nivel 1}
\PYG{c+c1}{\PYGZsh{}\PYGZsh{} Título de Nivel 2}
\PYG{c+c1}{\PYGZsh{}\PYGZsh{}\PYGZsh{} Título de Nivel 3` }
\end{sphinxVerbatim}

\sphinxAtStartPar
Con \sphinxcode{\sphinxupquote{:maxdepth: 2}}, solo se incluirían los niveles 1 y 2 en el índice.

\end{itemize}

\item {} 
\sphinxAtStartPar
\sphinxstylestrong{\sphinxcode{\sphinxupquote{:glob:}}}
\begin{itemize}
\item {} 
\sphinxAtStartPar
Habilita el uso de \sphinxstylestrong{comodines} para incluir varios archivos de forma automática.

\item {} 
\sphinxAtStartPar
Por ejemplo, \sphinxcode{\sphinxupquote{configuracion\_inicial/*}} seleccionará todos los archivos dentro de la carpeta \sphinxcode{\sphinxupquote{configuracion\_inicial/}}.

\end{itemize}

\end{enumerate}

\begin{sphinxVerbatim}[commandchars=\\\{\}]
\PYG{n}{configuracion\PYGZus{}inicial}\PYG{o}{/}\PYG{o}{*}
\end{sphinxVerbatim}

\sphinxAtStartPar
Este patrón selecciona \sphinxstylestrong{todos los archivos} en la carpeta \sphinxcode{\sphinxupquote{configuracion\_inicial/}}.
Sphinx incluirá cada archivo de forma automática en el índice generado.

\sphinxAtStartPar
Ejemplo:
\begin{itemize}
\item {} 
\sphinxAtStartPar
Si \sphinxcode{\sphinxupquote{configuracion\_inicial/}} contiene los siguientes archivos:

\end{itemize}

\begin{sphinxVerbatim}[commandchars=\\\{\}]
\PYG{n}{configuracion\PYGZus{}inicial}\PYG{o}{/}\PYG{l+m+mf}{001.}\PYG{n}{env}\PYG{o}{.}\PYG{n}{md}
\PYG{n}{configuracion\PYGZus{}inicial}\PYG{o}{/}\PYG{l+m+mf}{002.}\PYG{n}{instalacion\PYGZus{}librerias}\PYG{o}{.}\PYG{n}{md}
\PYG{n}{configuracion\PYGZus{}inicial}\PYG{o}{/}\PYG{l+m+mf}{003.}\PYG{n}{Creacion\PYGZus{}proyecto\PYGZus{}Sphinx}
\end{sphinxVerbatim}

\sphinxAtStartPar
Estos serán añadidos al índice y convertidos en enlaces en la navegación HTML generada.

\end{itemize}

\end{itemize}

\end{itemize}


\bigskip\hrule\bigskip



\subsubsection{1.3. \sphinxstylestrong{Archivo \sphinxstyleliteralintitle{\sphinxupquote{Makefile}}}}
\label{\detokenize{configuracion_inicial/004.estructura_inicial_proyecto:archivo-makefile}}\begin{itemize}
\item {} 
\sphinxAtStartPar
\sphinxstylestrong{Propósito}: Contiene comandos para construir la documentación en diferentes formatos.

\item {} 
\sphinxAtStartPar
\sphinxstylestrong{Uso}:
\begin{itemize}
\item {} 
\sphinxAtStartPar
Se ejecuta \sphinxcode{\sphinxupquote{make html}} para construir documentación HTML.

\item {} 
\sphinxAtStartPar
Se utiliza con el comando make desde una terminal.

\item {} 
\sphinxAtStartPar
Sistema Operativo: Es un archivo diseñado común en sistemas \sphinxstylestrong{Unix, Linux y macOS}.

\item {} 
\sphinxAtStartPar
Otros comandos incluyen:
\begin{itemize}
\item {} 
\sphinxAtStartPar
\sphinxcode{\sphinxupquote{make latexpdf}}: Genera un PDF si se tiene LaTeX configurado.

\item {} 
\sphinxAtStartPar
\sphinxcode{\sphinxupquote{make epub}}: Genera un eBook.

\end{itemize}

\end{itemize}

\end{itemize}


\bigskip\hrule\bigskip



\subsubsection{1.4. \sphinxstylestrong{Archivo \sphinxstyleliteralintitle{\sphinxupquote{make.bat}}}}
\label{\detokenize{configuracion_inicial/004.estructura_inicial_proyecto:archivo-make-bat}}\begin{itemize}
\item {} 
\sphinxAtStartPar
\sphinxstylestrong{Propósito}: Archivo equivalente al \sphinxcode{\sphinxupquote{Makefile}} para sistemas Windows.

\item {} 
\sphinxAtStartPar
\sphinxstylestrong{Uso}:
\begin{itemize}
\item {} 
\sphinxAtStartPar
Ejecuta comandos como \sphinxcode{\sphinxupquote{make.bat html}} en el terminal de Windows, para construir documentación HTML.

\item {} 
\sphinxAtStartPar
Sistema operativo: Es un archivo diseñado para el sistema \sphinxstylestrong{Windows}.

\end{itemize}

\end{itemize}


\bigskip\hrule\bigskip



\subsection{2. Flujo de Trabajo con MyST\sphinxhyphen{}Parser}
\label{\detokenize{configuracion_inicial/004.estructura_inicial_proyecto:flujo-de-trabajo-con-myst-parser}}\begin{enumerate}
\sphinxsetlistlabels{\arabic}{enumi}{enumii}{}{.}%
\item {} 
\sphinxAtStartPar
\sphinxstylestrong{Configuración de MyST\sphinxhyphen{}Parser en \sphinxcode{\sphinxupquote{conf.py}}}: Asegúrate de que \sphinxcode{\sphinxupquote{myst\sphinxhyphen{}parser}} esté habilitado:

\begin{sphinxVerbatim}[commandchars=\\\{\}]
\PYG{n}{extensions} \PYG{o}{=} \PYG{p}{[}\PYG{l+s+s1}{\PYGZsq{}}\PYG{l+s+s1}{myst\PYGZus{}parser}\PYG{l+s+s1}{\PYGZsq{}}\PYG{p}{]}
\PYG{n}{source\PYGZus{}suffix} \PYG{o}{=} \PYG{p}{\PYGZob{}}
    \PYG{l+s+s1}{\PYGZsq{}}\PYG{l+s+s1}{.md}\PYG{l+s+s1}{\PYGZsq{}}\PYG{p}{:} \PYG{l+s+s1}{\PYGZsq{}}\PYG{l+s+s1}{markdown}\PYG{l+s+s1}{\PYGZsq{}}\PYG{p}{,}
\PYG{p}{\PYGZcb{}}
\end{sphinxVerbatim}

\item {} 
\sphinxAtStartPar
\sphinxstylestrong{Estructura Markdown}:
\begin{itemize}
\item {} 
\sphinxAtStartPar
Los archivos Markdown (\sphinxcode{\sphinxupquote{.md}}) reemplazan los \sphinxcode{\sphinxupquote{.rst}}. Escribe tus documentos con encabezados \sphinxcode{\sphinxupquote{\#}}, listas, bloques de código, etc.

\item {} 
\sphinxAtStartPar
Usa MyST para integrar características de Sphinx en Markdown. Por ejemplo:

\begin{sphinxVerbatim}[commandchars=\\\{\}]
\PYG{c+c1}{\PYGZsh{} Título del Documento}

\PYG{p}{\PYGZob{}}\PYG{n}{toctree}\PYG{p}{\PYGZcb{}}
\PYG{p}{:}\PYG{n}{maxdepth}\PYG{p}{:} \PYG{l+m+mi}{2}

\PYG{n}{archivo1}\PYG{o}{.}\PYG{n}{md}
\PYG{n}{carpeta}\PYG{o}{/}\PYG{n}{archivo2}\PYG{o}{.}\PYG{n}{md}
\end{sphinxVerbatim}

\end{itemize}

\item {} 
\sphinxAtStartPar
\sphinxstylestrong{Compilación de la documentación}:
\begin{itemize}
\item {} 
\sphinxAtStartPar
Ejecuta:

\sphinxAtStartPar
\sphinxcode{\sphinxupquote{make html}}

\item {} 
\sphinxAtStartPar
Los archivos HTML generados estarán en \sphinxcode{\sphinxupquote{docs/build/html}}.

\end{itemize}

\item {} 
\sphinxAtStartPar
\sphinxstylestrong{Personalización}:
\begin{itemize}
\item {} 
\sphinxAtStartPar
Añade CSS en \sphinxcode{\sphinxupquote{\_static/}} para personalizar el diseño.

\item {} 
\sphinxAtStartPar
Usa \sphinxcode{\sphinxupquote{\_templates/}} para modificar las plantillas.

\end{itemize}

\end{enumerate}


\bigskip\hrule\bigskip


\sphinxAtStartPar
Esta estructura te permite trabajar con Markdown fácilmente, aprovechando la potencia de Sphinx y MyST\sphinxhyphen{}Parser para generar documentación de alta calidad.

\sphinxstepscope

\sphinxstepscope

\sphinxstepscope


\section{\sphinxstylestrong{Creación de ficheros de código y generación automática de documentación}}
\label{\detokenize{configuracion_inicial/007.Creacion_de_ficheros_de_codigo_y_generacion_automatica_de_documentacion:creacion-de-ficheros-de-codigo-y-generacion-automatica-de-documentacion}}\label{\detokenize{configuracion_inicial/007.Creacion_de_ficheros_de_codigo_y_generacion_automatica_de_documentacion::doc}}

\subsection{1. Creación del fichero de código Python}
\label{\detokenize{configuracion_inicial/007.Creacion_de_ficheros_de_codigo_y_generacion_automatica_de_documentacion:creacion-del-fichero-de-codigo-python}}
\sphinxAtStartPar
Se crea un fichero de código Python con \sphinxstylestrong{docstring en formato Google Style}. Este formato facilita la generación automática de documentación, ya que utiliza una estructura clara y legible tanto para humanos como para herramientas de generación de documentación.

\begin{sphinxVerbatim}[commandchars=\\\{\},numbers=left,firstnumber=1,stepnumber=1]
\PYG{k}{def} \PYG{n+nf}{add\PYGZus{}numbers}\PYG{p}{(}\PYG{n}{a}\PYG{p}{,} \PYG{n}{b}\PYG{p}{)}\PYG{p}{:}
\PYG{+w}{    }\PYG{l+s+sd}{\PYGZdq{}\PYGZdq{}\PYGZdq{}Add two numbers.}

\PYG{l+s+sd}{    Args:}
\PYG{l+s+sd}{        a (float): The first number.}
\PYG{l+s+sd}{        b (float): The second number.}

\PYG{l+s+sd}{    Returns:}
\PYG{l+s+sd}{        float: The sum of the two numbers.}
\PYG{l+s+sd}{    \PYGZdq{}\PYGZdq{}\PYGZdq{}}
    \PYG{k}{return} \PYG{n}{a} \PYG{o}{+} \PYG{n}{b}


\PYG{k}{def} \PYG{n+nf}{subtract\PYGZus{}numbers}\PYG{p}{(}\PYG{n}{a}\PYG{p}{,} \PYG{n}{b}\PYG{p}{)}\PYG{p}{:}
\PYG{+w}{    }\PYG{l+s+sd}{\PYGZdq{}\PYGZdq{}\PYGZdq{}}
\PYG{l+s+sd}{    Subtract one number from another.}

\PYG{l+s+sd}{    Args:}
\PYG{l+s+sd}{        a (float): The number to subtract from.}
\PYG{l+s+sd}{        b (float): The number to subtract.}

\PYG{l+s+sd}{    Returns:}
\PYG{l+s+sd}{        float: The result of the subtraction.}
\PYG{l+s+sd}{    \PYGZdq{}\PYGZdq{}\PYGZdq{}}
    \PYG{k}{return} \PYG{n}{a} \PYG{o}{\PYGZhy{}} \PYG{n}{b}


\PYG{k}{def} \PYG{n+nf}{multiply\PYGZus{}numbers}\PYG{p}{(}\PYG{n}{a}\PYG{p}{,} \PYG{n}{b}\PYG{p}{)}\PYG{p}{:}
\PYG{+w}{    }\PYG{l+s+sd}{\PYGZdq{}\PYGZdq{}\PYGZdq{}Multiply two numbers.}

\PYG{l+s+sd}{    Args:}
\PYG{l+s+sd}{        a (float): The first number.}
\PYG{l+s+sd}{        b (float): The second number.}

\PYG{l+s+sd}{    Returns:}
\PYG{l+s+sd}{        float: The product of the two numbers.}
\PYG{l+s+sd}{    \PYGZdq{}\PYGZdq{}\PYGZdq{}}
    \PYG{k}{return} \PYG{n}{a} \PYG{o}{*} \PYG{n}{b}


\PYG{k}{def} \PYG{n+nf}{divide\PYGZus{}numbers}\PYG{p}{(}\PYG{n}{a}\PYG{p}{,} \PYG{n}{b}\PYG{p}{)}\PYG{p}{:}
\PYG{+w}{    }\PYG{l+s+sd}{\PYGZdq{}\PYGZdq{}\PYGZdq{}Divide one number by another.}

\PYG{l+s+sd}{    Args:}
\PYG{l+s+sd}{        a (float): The numerator.}
\PYG{l+s+sd}{        b (float): The denominator.}

\PYG{l+s+sd}{    Returns:}
\PYG{l+s+sd}{        float: The result of the division.}
\PYG{l+s+sd}{    }
\PYG{l+s+sd}{    Raises:}
\PYG{l+s+sd}{        ValueError: If the denominator is zero.}
\PYG{l+s+sd}{    \PYGZdq{}\PYGZdq{}\PYGZdq{}}
    \PYG{k}{if} \PYG{n}{b} \PYG{o}{==} \PYG{l+m+mi}{0}\PYG{p}{:}
        \PYG{k}{raise} \PYG{n+ne}{ValueError}\PYG{p}{(}\PYG{l+s+s2}{\PYGZdq{}}\PYG{l+s+s2}{The denominator cannot be zero.}\PYG{l+s+s2}{\PYGZdq{}}\PYG{p}{)}
    \PYG{k}{return} \PYG{n}{a} \PYG{o}{/} \PYG{n}{b}
\end{sphinxVerbatim}

\sphinxAtStartPar
Explicación del formato Google Style:
\begin{itemize}
\item {} 
\sphinxAtStartPar
Args: Define los argumentos que la función recibe, indicando su nombre, tipo y propósito.

\item {} 
\sphinxAtStartPar
Returns: Describe lo que devuelve la función, incluyendo el tipo de dato.

\item {} 
\sphinxAtStartPar
Raises(opcional): Especifica las excepciones que la función podría lanzar.

\end{itemize}


\subsubsection{1.2. Pasos para generar la documentación automáticamente}
\label{\detokenize{configuracion_inicial/007.Creacion_de_ficheros_de_codigo_y_generacion_automatica_de_documentacion:pasos-para-generar-la-documentacion-automaticamente}}\begin{enumerate}
\sphinxsetlistlabels{\arabic}{enumi}{enumii}{}{.}%
\item {} 
\sphinxAtStartPar
\sphinxstylestrong{Instalación de Sphinx}:\\
Hay que asegurarse de tener Sphinx instalado en el entorno de trabajo. Si no, hay que instalarlo con:

\begin{sphinxVerbatim}[commandchars=\\\{\}]
pip\PYG{+w}{ }install\PYG{+w}{ }sphinx
\end{sphinxVerbatim}

\item {} 
\sphinxAtStartPar
\sphinxstylestrong{Ejecución de sphinx\sphinxhyphen{}apidoc}:

\sphinxAtStartPar
\sphinxcode{\sphinxupquote{sphinx\sphinxhyphen{}apidoc}} es una herramienta de línea de comandos incluida en Sphinx que genera automáticamente archivos de documentación para módulos y paquetes Python.
Su propósito principal es agilizar el proceso de creación de documentación basada en docstrings, generando archivos de entrada para Sphinx a partir de la estructura del código fuente.
Esta herramienta es especialmente útil en proyectos grandes con múltiples módulos y paquetes, ya que automatiza la creación de archivos de documentación necesarios para cada módulo, en el formato configurado (por ejemplo, .rst, .md o cualquier otro formato compatible).

\begin{sphinxVerbatim}[commandchars=\\\{\}]
sphinx\PYGZhy{}apidoc\PYG{+w}{ }\PYGZhy{}o\PYG{+w}{ }docs/source\PYG{+w}{ }src/\PYG{+w}{ }\PYGZhy{}\PYGZhy{}separate\PYG{+w}{ }\PYGZhy{}\PYGZhy{}suffix\PYG{+w}{ }.md
\end{sphinxVerbatim}
\begin{itemize}
\item {} 
\sphinxAtStartPar
\sphinxstylestrong{\sphinxcode{\sphinxupquote{\sphinxhyphen{}o docs/source}}}: Indica la carpeta donde se generarán los archivos de documentación.

\item {} 
\sphinxAtStartPar
\sphinxstylestrong{\sphinxcode{\sphinxupquote{src/}}}: Especifica el directorio donde se encuentran los archivos de código Python.

\item {} 
\sphinxAtStartPar
\sphinxstylestrong{\sphinxcode{\sphinxupquote{\sphinxhyphen{}\sphinxhyphen{}separate}}}: Genera un archivo separado para cada módulo detectado en lugar de consolidar toda la documentación en un único archivo.

\item {} 
\sphinxAtStartPar
\sphinxstylestrong{\sphinxcode{\sphinxupquote{\sphinxhyphen{}\sphinxhyphen{}suffix .md}}}: Configura la extensión de los archivos generados como \sphinxcode{\sphinxupquote{.md}}. Si esta opción no se especifica, el valor predeterminado es \sphinxcode{\sphinxupquote{.rst}}.

\end{itemize}

\sphinxAtStartPar
Al ejecutar el comando sphinx\sphinxhyphen{}apidoc, en nuestro caso los archivos generados son:
\begin{itemize}
\item {} 
\sphinxAtStartPar
math\_operations

\item {} 
\sphinxAtStartPar
modules

\end{itemize}

\item {} 
\sphinxAtStartPar
\sphinxstylestrong{Compilación de la documentación}:\\
Una vez se tengan los archivos generados, hay que compilar la documentación usando:
\begin{itemize}
\item {} 
\sphinxAtStartPar
Hay que navegar al directorio de documentación (\sphinxcode{\sphinxupquote{docs/}}):

\end{itemize}

\begin{sphinxVerbatim}[commandchars=\\\{\}]
\PYG{n+nb}{cd}\PYG{+w}{ }docs
\end{sphinxVerbatim}
\begin{itemize}
\item {} 
\sphinxAtStartPar
Luego hay que ejecutar el siguiente comando para generar la documentación HTML:

\end{itemize}

\begin{sphinxVerbatim}[commandchars=\\\{\}]
sphinx\PYGZhy{}build\PYG{+w}{ }\PYGZhy{}b\PYG{+w}{ }html\PYG{+w}{ }\PYG{n+nb}{source}\PYG{+w}{ }build
\end{sphinxVerbatim}

\sphinxAtStartPar
Esto creará una carpeta \sphinxcode{\sphinxupquote{docs/build/html/}} con los archivos HTML de la documentación generada.

\end{enumerate}


\bigskip\hrule\bigskip



\subsection{2. Resultados obtenidos}
\label{\detokenize{configuracion_inicial/007.Creacion_de_ficheros_de_codigo_y_generacion_automatica_de_documentacion:resultados-obtenidos}}
\sphinxAtStartPar
Cuando ejecutas el comando \sphinxcode{\sphinxupquote{sphinx\sphinxhyphen{}apidoc}}, se generan una serie de archivos en el directorio \sphinxcode{\sphinxupquote{docs/source}} que contienen la estructura básica de la documentación para el proyecto. Este comando facilita la generación automática de documentación a partir del código fuente, especialmente de los docstrings de los módulos y clases Python.


\subsubsection{2.1. \sphinxstylestrong{Estructura de Archivos Generada}}
\label{\detokenize{configuracion_inicial/007.Creacion_de_ficheros_de_codigo_y_generacion_automatica_de_documentacion:estructura-de-archivos-generada}}
\sphinxAtStartPar
Al ejecutar \sphinxcode{\sphinxupquote{sphinx\sphinxhyphen{}apidoc \sphinxhyphen{}o docs/source src/}}, se creará una serie de archivos en la carpeta \sphinxcode{\sphinxupquote{docs/source}}:


\paragraph{a. \sphinxstylestrong{\sphinxstyleliteralintitle{\sphinxupquote{index.rst}} o \sphinxstyleliteralintitle{\sphinxupquote{index.md}}}}
\label{\detokenize{configuracion_inicial/007.Creacion_de_ficheros_de_codigo_y_generacion_automatica_de_documentacion:a-index-rst-o-index-md}}
\sphinxAtStartPar
Este es el archivo principal de la documentación. Contiene la estructura base, la cual incluirá todos los módulos generados por \sphinxcode{\sphinxupquote{sphinx\sphinxhyphen{}apidoc}}. En nuestro caso al estar usando Markdown, este archivo es \sphinxcode{\sphinxupquote{index.md}}.

\sphinxAtStartPar
Ejemplo de un archivo \sphinxcode{\sphinxupquote{index.md}}:

\begin{sphinxVerbatim}[commandchars=\\\{\}]
\PYG{g+gh}{\PYGZsh{} Documentación de T05: Sphinx y Myst\PYGZhy{}Parser}

\PYG{g+gu}{\PYGZsh{}\PYGZsh{} Índice}

```\PYGZob{}toctree\PYGZcb{}
:maxdepth: 2
:glob:

configuracion\PYGZus{}inicial/*
comandos\PYGZus{}mas\PYGZus{}usados/*
\end{sphinxVerbatim}

\sphinxAtStartPar
Este archivo contiene:
\begin{itemize}
\item {} 
\sphinxAtStartPar
Un título de la documentación.

\item {} 
\sphinxAtStartPar
Una sección de índice que lista los módulos o archivos que se incluyen en la documentación generada.

\end{itemize}


\paragraph{b. \sphinxstylestrong{Archivos \sphinxstyleliteralintitle{\sphinxupquote{.rst}} o \sphinxstyleliteralintitle{\sphinxupquote{.md}} para cada módulo Python}}
\label{\detokenize{configuracion_inicial/007.Creacion_de_ficheros_de_codigo_y_generacion_automatica_de_documentacion:b-archivos-rst-o-md-para-cada-modulo-python}}
\sphinxAtStartPar
Cada módulo de Python dentro de la carpeta \sphinxcode{\sphinxupquote{src/}} tendrá un archivo generado. Nuestro proyecto tiene el módulo
\sphinxcode{\sphinxupquote{math\_operations.py}}, se genera el archivo:
\begin{itemize}
\item {} 
\sphinxAtStartPar
\sphinxcode{\sphinxupquote{math\_operations.rst}} o \sphinxcode{\sphinxupquote{math\_operations.md}}

\end{itemize}

\sphinxAtStartPar
Este archivo contienen la documentación generada a partir de los docstrings de los módulos. Por ejemplo, si tu módulo \sphinxcode{\sphinxupquote{math\_operations.py}} tiene funciones con docstrings, el archivo generado incluirá lo siguiente:

\begin{sphinxVerbatim}[commandchars=\\\{\}]
\PYG{n}{math}\PYGZbs{}\PYG{n}{\PYGZus{}operations} \PYG{n}{module}
\PYG{o}{==}\PYG{o}{==}\PYG{o}{==}\PYG{o}{==}\PYG{o}{==}\PYG{o}{==}\PYG{o}{==}\PYG{o}{==}\PYG{o}{==}\PYG{o}{==}\PYG{o}{==}\PYG{o}{=}

\PYG{o}{.}\PYG{o}{.} \PYG{n}{automodule}\PYG{p}{:}\PYG{p}{:} \PYG{n}{math\PYGZus{}operations}
   \PYG{p}{:}\PYG{n}{members}\PYG{p}{:}
   \PYG{p}{:}\PYG{n}{undoc}\PYG{o}{\PYGZhy{}}\PYG{n}{members}\PYG{p}{:}
   \PYG{p}{:}\PYG{n}{show}\PYG{o}{\PYGZhy{}}\PYG{n}{inheritance}\PYG{p}{:}


\end{sphinxVerbatim}

\sphinxAtStartPar
Este archivo generado tiene varias secciones importantes:
\begin{itemize}
\item {} 
\sphinxAtStartPar
\sphinxstylestrong{\sphinxcode{\sphinxupquote{automodule:: math\_operations}}}: Esto indica que Sphinx debe generar la documentación automáticamente para el módulo \sphinxcode{\sphinxupquote{math\_operations}}.

\item {} 
\sphinxAtStartPar
\sphinxstylestrong{\sphinxcode{\sphinxupquote{:members:}}}: Esto genera la documentación para todas las funciones y clases dentro del módulo.

\item {} 
\sphinxAtStartPar
\sphinxstylestrong{\sphinxcode{\sphinxupquote{:undoc\sphinxhyphen{}members:}}}: Si hay funciones o clases sin docstrings, esta directiva las incluirá en la documentación.

\item {} 
\sphinxAtStartPar
\sphinxstylestrong{\sphinxcode{\sphinxupquote{:show\sphinxhyphen{}inheritance:}}}: Si hay clases, muestra también la jerarquía de herencia.

\end{itemize}

\sphinxAtStartPar
Este es el formato que Sphinx usa para incluir la documentación del código fuente, asegurándose de que los cambios en los archivos de código fuente se reflejen automáticamente en la documentación.


\paragraph{c. \sphinxstylestrong{\sphinxstyleliteralintitle{\sphinxupquote{modules.rst}} o \sphinxstyleliteralintitle{\sphinxupquote{modules.md}}}}
\label{\detokenize{configuracion_inicial/007.Creacion_de_ficheros_de_codigo_y_generacion_automatica_de_documentacion:c-modules-rst-o-modules-md}}
\sphinxAtStartPar
Este archivo lista todos los módulos generados, lo cual es útil si hay muchos módulos en el proyecto. Por ejemplo:

\begin{sphinxVerbatim}[commandchars=\\\{\}]
\PYG{n}{src}
\PYG{o}{==}\PYG{o}{=}

\PYG{o}{.}\PYG{o}{.} \PYG{n}{toctree}\PYG{p}{:}\PYG{p}{:}
   \PYG{p}{:}\PYG{n}{maxdepth}\PYG{p}{:} \PYG{l+m+mi}{4}

   \PYG{n}{math\PYGZus{}operations}


\end{sphinxVerbatim}

\sphinxAtStartPar
Este archivo crea un índice de módulos que Sphinx incluirá en la documentación final.


\subsubsection{2.2. \sphinxstylestrong{Contenido de los Archivos Generados}}
\label{\detokenize{configuracion_inicial/007.Creacion_de_ficheros_de_codigo_y_generacion_automatica_de_documentacion:contenido-de-los-archivos-generados}}

\paragraph{a. \sphinxstylestrong{Documentación Generada Automáticamente desde el Código}}
\label{\detokenize{configuracion_inicial/007.Creacion_de_ficheros_de_codigo_y_generacion_automatica_de_documentacion:a-documentacion-generada-automaticamente-desde-el-codigo}}
\sphinxAtStartPar
El objetivo principal de \sphinxcode{\sphinxupquote{sphinx\sphinxhyphen{}apidoc}} es generar documentación automáticamente a partir de los docstrings en el código Python. Para cada archivo \sphinxcode{\sphinxupquote{.py}} que contiene docstrings, se generará una representación en reStructuredText o Markdown.

\sphinxAtStartPar
Por ejemplo, nuestro archivo \sphinxcode{\sphinxupquote{math\_operations.py}} con el siguiente código:

\begin{sphinxVerbatim}[commandchars=\\\{\}]
\PYG{k}{def} \PYG{n+nf}{add\PYGZus{}numbers}\PYG{p}{(}\PYG{n}{a}\PYG{p}{,} \PYG{n}{b}\PYG{p}{)}\PYG{p}{:}
\PYG{+w}{    }\PYG{l+s+sd}{\PYGZdq{}\PYGZdq{}\PYGZdq{}Suma dos números.\PYGZdq{}\PYGZdq{}\PYGZdq{}}
    \PYG{k}{return} \PYG{n}{a} \PYG{o}{+} \PYG{n}{b}

\end{sphinxVerbatim}

\sphinxAtStartPar
El archivo \sphinxcode{\sphinxupquote{math\_operations.md}} generado tendrá algo similar a esto:

\begin{sphinxVerbatim}[commandchars=\\\{\}]
\PYG{g+gh}{math\PYGZus{}operations}
\PYG{g+gh}{===============}

.. automodule:: math\PYGZus{}operations
   :members:
   :undoc\PYGZhy{}members:
   :show\PYGZhy{}inheritance:

\PYG{g+gu}{add\PYGZus{}numbers(a, b)}
\PYG{g+gu}{\PYGZhy{}\PYGZhy{}\PYGZhy{}\PYGZhy{}\PYGZhy{}\PYGZhy{}\PYGZhy{}\PYGZhy{}\PYGZhy{}\PYGZhy{}\PYGZhy{}\PYGZhy{}\PYGZhy{}\PYGZhy{}\PYGZhy{}\PYGZhy{}\PYGZhy{}}

Suma dos números.
\end{sphinxVerbatim}

\sphinxAtStartPar
La directiva \sphinxcode{\sphinxupquote{automodule}} incluye el módulo completo, y las funciones o clases del módulo se detallan debajo con sus respectivas descripciones.


\paragraph{b. \sphinxstylestrong{Incluir Todo el Código del Proyecto}}
\label{\detokenize{configuracion_inicial/007.Creacion_de_ficheros_de_codigo_y_generacion_automatica_de_documentacion:b-incluir-todo-el-codigo-del-proyecto}}
\sphinxAtStartPar
Los archivos generados también incluyen código fuente si usamos la extensión \sphinxcode{\sphinxupquote{viewcode}} en el archivo \sphinxcode{\sphinxupquote{conf.py}} de Sphinx. Esto agrega enlaces al código fuente y puede incluir la visualización completa del código si se activa la opción correspondiente.


\paragraph{c. \sphinxstylestrong{Dependencias de Módulos}}
\label{\detokenize{configuracion_inicial/007.Creacion_de_ficheros_de_codigo_y_generacion_automatica_de_documentacion:c-dependencias-de-modulos}}
\sphinxAtStartPar
Si el proyecto tiene dependencias de otros módulos dentro de \sphinxcode{\sphinxupquote{src/}}, el sistema de documentación puede generar automáticamente enlaces a esos módulos y sus funciones.


\subsubsection{2.3. \sphinxstylestrong{Uso de los Archivos Generados}}
\label{\detokenize{configuracion_inicial/007.Creacion_de_ficheros_de_codigo_y_generacion_automatica_de_documentacion:uso-de-los-archivos-generados}}
\sphinxAtStartPar
Una vez que \sphinxcode{\sphinxupquote{sphinx\sphinxhyphen{}apidoc}} ha generado estos archivos, puedes:
\begin{itemize}
\item {} 
\sphinxAtStartPar
\sphinxstylestrong{Editar manualmente} el archivo \sphinxcode{\sphinxupquote{index.rst}} o \sphinxcode{\sphinxupquote{index.md}} para personalizar la estructura de la documentación.

\item {} 
\sphinxAtStartPar
\sphinxstylestrong{Incluir o excluir módulos} de la documentación generada editando los archivos \sphinxcode{\sphinxupquote{.rst}} o \sphinxcode{\sphinxupquote{.md}} generados.

\item {} 
\sphinxAtStartPar
\sphinxstylestrong{Ejecutar \sphinxcode{\sphinxupquote{sphinx\sphinxhyphen{}build}}} para generar la documentación final en formato HTML, PDF, etc.

\end{itemize}

\sphinxAtStartPar
Por ejemplo, se puede generar la documentación HTML con el siguiente comando:

\begin{sphinxVerbatim}[commandchars=\\\{\}]
sphinx\PYGZhy{}build\PYG{+w}{ }\PYGZhy{}b\PYG{+w}{ }html\PYG{+w}{ }\PYG{n+nb}{source}\PYG{+w}{ }build
\end{sphinxVerbatim}

\sphinxAtStartPar
Esto generará una carpeta \sphinxcode{\sphinxupquote{build}} que contendrá la documentación final en formato HTML. Se pueden abrir estos archivos en el navegador para visualizar la documentación generada.


\subsubsection{2.4. \sphinxstylestrong{Personalización de la Documentación Generada}}
\label{\detokenize{configuracion_inicial/007.Creacion_de_ficheros_de_codigo_y_generacion_automatica_de_documentacion:personalizacion-de-la-documentacion-generada}}
\sphinxAtStartPar
Aunque los archivos generados por \sphinxcode{\sphinxupquote{sphinx\sphinxhyphen{}apidoc}} contienen una buena base de documentación, se pueden personalizar aún más, por ejemplo:
\begin{itemize}
\item {} 
\sphinxAtStartPar
\sphinxstylestrong{Agregar enlaces y referencias}: Se puede usar \sphinxcode{\sphinxupquote{:ref:}} y \sphinxcode{\sphinxupquote{:doc:}} para vincular otros documentos generados por Sphinx.

\item {} 
\sphinxAtStartPar
\sphinxstylestrong{Agregar secciones adicionales}: Se puede agregar más texto explicativo o personalizar la apariencia con archivos CSS o HTML.

\end{itemize}


\subsection{3. Conclusiones}
\label{\detokenize{configuracion_inicial/007.Creacion_de_ficheros_de_codigo_y_generacion_automatica_de_documentacion:conclusiones}}\begin{itemize}
\item {} 
\sphinxAtStartPar
\sphinxstylestrong{Automatización}: El uso de \sphinxcode{\sphinxupquote{sphinx\sphinxhyphen{}apidoc}} simplifica la generación de documentación a partir de código Python, ahorrando tiempo y esfuerzo.

\item {} 
\sphinxAtStartPar
\sphinxstylestrong{Legibilidad}: El uso de docstring en formato Google Style mejora la claridad y consistencia de la documentación.

\item {} 
\sphinxAtStartPar
\sphinxstylestrong{Compatibilidad}: Los archivos \sphinxcode{\sphinxupquote{.rst}} generados por Sphinx son fácilmente convertibles a otros formatos, como \sphinxcode{\sphinxupquote{.md}}, para adaptarse a distintas necesidades de publicación.

\end{itemize}

\sphinxstepscope


\section{\sphinxstylestrong{Generación de HTML a partir de la documentación}}
\label{\detokenize{configuracion_inicial/008.Generar_HTML:generacion-de-html-a-partir-de-la-documentacion}}\label{\detokenize{configuracion_inicial/008.Generar_HTML::doc}}
\sphinxAtStartPar
La generación de documentación HTML es uno de los puntos clave en un flujo de trabajo con \sphinxstylestrong{Sphinx} y \sphinxstylestrong{MyST\sphinxhyphen{}Parser}. Este proceso convierte los archivos fuente (Markdown o reStructuredText) en una interfaz web navegable y estilizada.


\bigskip\hrule\bigskip



\subsection{1. Preparar la estructura del proyecto}
\label{\detokenize{configuracion_inicial/008.Generar_HTML:preparar-la-estructura-del-proyecto}}
\sphinxAtStartPar
Asegúrate de que la estructura de tu proyecto tiene la siguiente organización básica:

\begin{sphinxVerbatim}[commandchars=\\\{\}]
docs/
├── source/
│   ├── conf.py        \PYGZlt{}\PYGZhy{} Archivo de configuración principal
│   ├── index.md       \PYGZlt{}\PYGZhy{} Página principal (puede ser index.rst o index.md)
│   ├── \PYGZus{}static/       \PYGZlt{}\PYGZhy{} Recursos estáticos (imágenes, CSS, JS)
│   ├── \PYGZus{}templates/    \PYGZlt{}\PYGZhy{} Plantillas personalizadas para HTML
│   └── secciones/     \PYGZlt{}\PYGZhy{} Más archivos .md o .rst
└── make.bat          \PYGZlt{}\PYGZhy{} Script para construir la documentación (Windows)
└── Makefile          \PYGZlt{}\PYGZhy{} Script para construir la documentación (Linux/macOS)
\end{sphinxVerbatim}


\bigskip\hrule\bigskip



\subsection{2. Crear contenido en Markdown}
\label{\detokenize{configuracion_inicial/008.Generar_HTML:crear-contenido-en-markdown}}
\sphinxAtStartPar
En la carpeta \sphinxcode{\sphinxupquote{source}}, crea o edita el archivo \sphinxcode{\sphinxupquote{index.md}} (o \sphinxcode{\sphinxupquote{index.rst}} si trabajas con reStructuredText).


\subsubsection{2.1. Ejemplo de index.md:}
\label{\detokenize{configuracion_inicial/008.Generar_HTML:ejemplo-de-index-md}}
\begin{sphinxVerbatim}[commandchars=\\\{\}]
\PYG{g+gh}{\PYGZsh{} Documentación del Proyecto}

¡Bienvenido a la documentación oficial del proyecto!

\PYG{g+gu}{\PYGZsh{}\PYGZsh{} Introducción}

Esta documentación cubre todas las características clave.

\PYG{g+gu}{\PYGZsh{}\PYGZsh{} Secciones}

\PYG{k}{\PYGZhy{}}\PYG{+w}{ }[\PYG{n+nt}{Instalación}](\PYG{n+na}{instalacion.md})
\PYG{k}{\PYGZhy{}}\PYG{+w}{ }[\PYG{n+nt}{Uso Básico}](\PYG{n+na}{uso\PYGZus{}basico.md})

\PYG{g+gu}{\PYGZsh{}\PYGZsh{} Referencias}

Más información en [\PYG{n+nt}{referencia técnica}](\PYG{n+na}{referencia.md}).
\end{sphinxVerbatim}

\sphinxAtStartPar
⚠️ \sphinxstylestrong{Importante:} En caso de usar Markdown, asegúrate de que en \sphinxcode{\sphinxupquote{conf.py}} tengas habilitado MyST\sphinxhyphen{}Parser:

\begin{sphinxVerbatim}[commandchars=\\\{\}]
\PYG{n}{extensions} \PYG{o}{=} \PYG{p}{[}
    \PYG{l+s+s2}{\PYGZdq{}}\PYG{l+s+s2}{myst\PYGZus{}parser}\PYG{l+s+s2}{\PYGZdq{}}
\PYG{p}{]}
\end{sphinxVerbatim}


\bigskip\hrule\bigskip



\subsection{3. Configurar la salida HTML en conf.py}
\label{\detokenize{configuracion_inicial/008.Generar_HTML:configurar-la-salida-html-en-conf-py}}
\sphinxAtStartPar
Abre el archivo \sphinxcode{\sphinxupquote{conf.py}} y verifica lo siguiente:

\begin{sphinxVerbatim}[commandchars=\\\{\}]
\PYG{c+c1}{\PYGZsh{} Tema HTML}
\PYG{n}{html\PYGZus{}theme} \PYG{o}{=} \PYG{l+s+s1}{\PYGZsq{}}\PYG{l+s+s1}{sphinx\PYGZus{}rtd\PYGZus{}theme}\PYG{l+s+s1}{\PYGZsq{}}

\PYG{c+c1}{\PYGZsh{} Directorios de recursos estáticos}
\PYG{n}{html\PYGZus{}static\PYGZus{}path} \PYG{o}{=} \PYG{p}{[}\PYG{l+s+s1}{\PYGZsq{}}\PYG{l+s+s1}{\PYGZus{}static}\PYG{l+s+s1}{\PYGZsq{}}\PYG{p}{]}

\PYG{c+c1}{\PYGZsh{} Tipo de archivos fuente}
\PYG{n}{source\PYGZus{}suffix} \PYG{o}{=} \PYG{p}{\PYGZob{}}
    \PYG{l+s+s1}{\PYGZsq{}}\PYG{l+s+s1}{.rst}\PYG{l+s+s1}{\PYGZsq{}}\PYG{p}{:} \PYG{l+s+s1}{\PYGZsq{}}\PYG{l+s+s1}{restructuredtext}\PYG{l+s+s1}{\PYGZsq{}}\PYG{p}{,}
    \PYG{l+s+s1}{\PYGZsq{}}\PYG{l+s+s1}{.md}\PYG{l+s+s1}{\PYGZsq{}}\PYG{p}{:} \PYG{l+s+s1}{\PYGZsq{}}\PYG{l+s+s1}{markdown}\PYG{l+s+s1}{\PYGZsq{}}\PYG{p}{,}
\PYG{p}{\PYGZcb{}}
\end{sphinxVerbatim}

\sphinxAtStartPar
Estas configuraciones aseguran que Sphinx procese correctamente los archivos \sphinxcode{\sphinxupquote{.md}} y genere la salida en HTML.


\bigskip\hrule\bigskip



\subsection{4. Generar la documentación HTML}
\label{\detokenize{configuracion_inicial/008.Generar_HTML:generar-la-documentacion-html}}\begin{itemize}
\item {} 
\sphinxAtStartPar
\sphinxstylestrong{Windows:}
Abre una terminal en la carpeta raíz de tu proyecto (donde está \sphinxcode{\sphinxupquote{make.bat}}) y ejecuta:

\begin{sphinxVerbatim}[commandchars=\\\{\}]
make.bat html
\end{sphinxVerbatim}

\item {} 
\sphinxAtStartPar
\sphinxstylestrong{Linux/macOS:}
En terminal, usa el comando:

\begin{sphinxVerbatim}[commandchars=\\\{\}]
make\PYG{+w}{ }html
\end{sphinxVerbatim}

\end{itemize}


\subsubsection{4.1. ¿Qué sucede aquí?}
\label{\detokenize{configuracion_inicial/008.Generar_HTML:que-sucede-aqui}}\begin{itemize}
\item {} 
\sphinxAtStartPar
Sphinx lee los archivos \sphinxcode{\sphinxupquote{.md}} o \sphinxcode{\sphinxupquote{.rst}} en el directorio \sphinxcode{\sphinxupquote{source}}.

\item {} 
\sphinxAtStartPar
Convierte estos archivos en HTML.

\item {} 
\sphinxAtStartPar
El resultado se guarda en el directorio \sphinxcode{\sphinxupquote{build/html}}.

\end{itemize}


\bigskip\hrule\bigskip



\subsection{5. Explorar la salida HTML}
\label{\detokenize{configuracion_inicial/008.Generar_HTML:explorar-la-salida-html}}
\sphinxAtStartPar
Una vez finalizado el proceso, navega hasta:

\begin{sphinxVerbatim}[commandchars=\\\{\}]
docs/build/html/index.html
\end{sphinxVerbatim}
\begin{itemize}
\item {} 
\sphinxAtStartPar
Abre el archivo \sphinxcode{\sphinxupquote{index.html}} en tu navegador web preferido.

\item {} 
\sphinxAtStartPar
Deberías ver una interfaz estilizada con el contenido de tu documentación.

\end{itemize}


\subsubsection{5.1. Ejemplo de salida HTML:}
\label{\detokenize{configuracion_inicial/008.Generar_HTML:ejemplo-de-salida-html}}
\sphinxAtStartPar
\sphinxincludegraphics{{imageHTML}.png}

\sphinxstepscope


\section{\sphinxstylestrong{Generación de PDF a partir de la documentación}}
\label{\detokenize{configuracion_inicial/009.Generar_PDF:generacion-de-pdf-a-partir-de-la-documentacion}}\label{\detokenize{configuracion_inicial/009.Generar_PDF::doc}}
\sphinxAtStartPar
Generar un documento PDF a partir de tu documentación es una funcionalidad poderosa que ofrece \sphinxstylestrong{Sphinx} gracias a su integración con \sphinxstylestrong{LaTeX}. Sin embargo, este proceso requiere una configuración adecuada y algunas dependencias instaladas previamente.


\bigskip\hrule\bigskip



\subsection{1. Requisitos previos}
\label{\detokenize{configuracion_inicial/009.Generar_PDF:requisitos-previos}}
\sphinxAtStartPar
Antes de generar archivos PDF, asegúrate de cumplir con los siguientes requisitos:
\begin{enumerate}
\sphinxsetlistlabels{\arabic}{enumi}{enumii}{}{.}%
\item {} 
\sphinxAtStartPar
\sphinxstylestrong{Instalar LaTeX:}
\begin{itemize}
\item {} 
\sphinxAtStartPar
En \sphinxstylestrong{Windows:} Descarga e instala \sphinxstylestrong{MiKTeX} desde \sphinxurl{https://miktex.org/download}.

\item {} 
\sphinxAtStartPar
En \sphinxstylestrong{Linux:} Instala \sphinxcode{\sphinxupquote{TeX Live}} con:

\begin{sphinxVerbatim}[commandchars=\\\{\}]
sudo\PYG{+w}{ }apt\PYG{+w}{ }install\PYG{+w}{ }texlive\PYGZhy{}latex\PYGZhy{}base\PYG{+w}{ }texlive\PYGZhy{}fonts\PYGZhy{}recommended\PYG{+w}{ }texlive\PYGZhy{}latex\PYGZhy{}extra
\end{sphinxVerbatim}

\item {} 
\sphinxAtStartPar
En \sphinxstylestrong{macOS:} Instala \sphinxstylestrong{MacTeX} desde https://tug.org/mactex/.

\end{itemize}

\item {} 
\sphinxAtStartPar
\sphinxstylestrong{Verificar la instalación:}\\
Abre una terminal y verifica que \sphinxcode{\sphinxupquote{pdflatex}} está disponible:

\begin{sphinxVerbatim}[commandchars=\\\{\}]
pdflatex\PYG{+w}{ }\PYGZhy{}\PYGZhy{}version
\end{sphinxVerbatim}

\sphinxAtStartPar
Deberías ver información sobre la versión instalada.

\item {} 
\sphinxAtStartPar
\sphinxstylestrong{Paquetes adicionales de LaTeX:}\\
Es posible que necesites algunos paquetes adicionales. Puedes instalarlos manualmente o permitir que MiKTeX los descargue automáticamente cuando sea necesario.

\end{enumerate}


\bigskip\hrule\bigskip



\subsection{2. Configurar el archivo conf.py}
\label{\detokenize{configuracion_inicial/009.Generar_PDF:configurar-el-archivo-conf-py}}
\sphinxAtStartPar
Abre el archivo \sphinxcode{\sphinxupquote{conf.py}} y asegúrate de tener la siguiente configuración para LaTeX:

\begin{sphinxVerbatim}[commandchars=\\\{\}]
\PYG{c+c1}{\PYGZsh{} Usar pdflatex como motor de compilación}
\PYG{n}{latex\PYGZus{}engine} \PYG{o}{=} \PYG{l+s+s1}{\PYGZsq{}}\PYG{l+s+s1}{pdflatex}\PYG{l+s+s1}{\PYGZsq{}}

\PYG{c+c1}{\PYGZsh{} Configurar opciones para el documento PDF}
\PYG{n}{latex\PYGZus{}elements} \PYG{o}{=} \PYG{p}{\PYGZob{}}
    \PYG{l+s+s1}{\PYGZsq{}}\PYG{l+s+s1}{papersize}\PYG{l+s+s1}{\PYGZsq{}}\PYG{p}{:} \PYG{l+s+s1}{\PYGZsq{}}\PYG{l+s+s1}{a4paper}\PYG{l+s+s1}{\PYGZsq{}}\PYG{p}{,}          \PYG{c+c1}{\PYGZsh{} Tamaño del papel}
    \PYG{l+s+s1}{\PYGZsq{}}\PYG{l+s+s1}{pointsize}\PYG{l+s+s1}{\PYGZsq{}}\PYG{p}{:} \PYG{l+s+s1}{\PYGZsq{}}\PYG{l+s+s1}{10pt}\PYG{l+s+s1}{\PYGZsq{}}\PYG{p}{,}             \PYG{c+c1}{\PYGZsh{} Tamaño de fuente principal}
    \PYG{l+s+s1}{\PYGZsq{}}\PYG{l+s+s1}{preamble}\PYG{l+s+s1}{\PYGZsq{}}\PYG{p}{:} \PYG{l+s+sa}{r}\PYG{l+s+s1}{\PYGZsq{}}\PYG{l+s+s1}{\PYGZbs{}}\PYG{l+s+s1}{usepackage[utf8]}\PYG{l+s+si}{\PYGZob{}inputenc\PYGZcb{}}\PYG{l+s+s1}{\PYGZsq{}}\PYG{p}{,}  \PYG{c+c1}{\PYGZsh{} Codificación de entrada}
    \PYG{l+s+s1}{\PYGZsq{}}\PYG{l+s+s1}{figure\PYGZus{}align}\PYG{l+s+s1}{\PYGZsq{}}\PYG{p}{:} \PYG{l+s+s1}{\PYGZsq{}}\PYG{l+s+s1}{htbp}\PYG{l+s+s1}{\PYGZsq{}}           \PYG{c+c1}{\PYGZsh{} Alineación de figuras}
\PYG{p}{\PYGZcb{}}

\PYG{c+c1}{\PYGZsh{} Documentos PDF a generar}
\PYG{n}{latex\PYGZus{}documents} \PYG{o}{=} \PYG{p}{[}
    \PYG{p}{(}\PYG{l+s+s1}{\PYGZsq{}}\PYG{l+s+s1}{index}\PYG{l+s+s1}{\PYGZsq{}}\PYG{p}{,} \PYG{l+s+s1}{\PYGZsq{}}\PYG{l+s+s1}{documento.tex}\PYG{l+s+s1}{\PYGZsq{}}\PYG{p}{,} \PYG{l+s+s1}{\PYGZsq{}}\PYG{l+s+s1}{Título del Documento}\PYG{l+s+s1}{\PYGZsq{}}\PYG{p}{,}
     \PYG{l+s+s1}{\PYGZsq{}}\PYG{l+s+s1}{Autor del Documento}\PYG{l+s+s1}{\PYGZsq{}}\PYG{p}{,} \PYG{l+s+s1}{\PYGZsq{}}\PYG{l+s+s1}{manual}\PYG{l+s+s1}{\PYGZsq{}}\PYG{p}{)}\PYG{p}{,}
\PYG{p}{]}
\end{sphinxVerbatim}


\subsubsection{2.1 Explicación de las opciones clave:}
\label{\detokenize{configuracion_inicial/009.Generar_PDF:explicacion-de-las-opciones-clave}}\begin{itemize}
\item {} 
\sphinxAtStartPar
\sphinxcode{\sphinxupquote{papersize}}: Define el tamaño de la página (\sphinxcode{\sphinxupquote{a4paper}} o \sphinxcode{\sphinxupquote{letterpaper}}).

\item {} 
\sphinxAtStartPar
\sphinxcode{\sphinxupquote{pointsize}}: Tamaño de la fuente en puntos (\sphinxcode{\sphinxupquote{10pt}}, \sphinxcode{\sphinxupquote{11pt}}, \sphinxcode{\sphinxupquote{12pt}}).

\item {} 
\sphinxAtStartPar
\sphinxcode{\sphinxupquote{preamble}}: Permite agregar configuraciones adicionales de LaTeX.

\item {} 
\sphinxAtStartPar
\sphinxcode{\sphinxupquote{figure\_align}}: Controla la alineación de las figuras.

\end{itemize}


\bigskip\hrule\bigskip



\subsection{3. Generar archivos LaTeX}
\label{\detokenize{configuracion_inicial/009.Generar_PDF:generar-archivos-latex}}
\sphinxAtStartPar
En la terminal, dirígete a la carpeta raíz donde se encuentra el archivo \sphinxcode{\sphinxupquote{make.bat}} (en Windows) o \sphinxcode{\sphinxupquote{Makefile}} (en Linux/macOS).
\begin{itemize}
\item {} 
\sphinxAtStartPar
\sphinxstylestrong{Windows:}

\begin{sphinxVerbatim}[commandchars=\\\{\}]
make.bat latex
\end{sphinxVerbatim}

\item {} 
\sphinxAtStartPar
\sphinxstylestrong{Linux/macOS:}

\begin{sphinxVerbatim}[commandchars=\\\{\}]
make\PYG{+w}{ }latex
\end{sphinxVerbatim}

\end{itemize}


\subsubsection{3.1. Resultado esperado:}
\label{\detokenize{configuracion_inicial/009.Generar_PDF:resultado-esperado}}
\sphinxAtStartPar
Esto generará una carpeta \sphinxcode{\sphinxupquote{build/latex}} con varios archivos, incluyendo un archivo \sphinxcode{\sphinxupquote{.tex}}.

\sphinxAtStartPar
📸 \sphinxstylestrong{Ejemplo de resultado:}

\sphinxAtStartPar
\sphinxincludegraphics{{imagePDF1}.png}


\bigskip\hrule\bigskip



\subsection{4. Compilar el archivo .tex a PDF}
\label{\detokenize{configuracion_inicial/009.Generar_PDF:compilar-el-archivo-tex-a-pdf}}
\sphinxAtStartPar
Dirígete a la carpeta \sphinxcode{\sphinxupquote{build/latex}} y compila el archivo \sphinxcode{\sphinxupquote{.tex}}:

\begin{sphinxVerbatim}[commandchars=\\\{\}]
pdflatex\PYG{+w}{ }documento.tex
\end{sphinxVerbatim}
\begin{itemize}
\item {} 
\sphinxAtStartPar
Si aparece un diálogo de instalación de paquetes (en MiKTeX), selecciona un servidor local (por ejemplo, España) y permite la instalación de los paquetes requeridos.

\item {} 
\sphinxAtStartPar
Es posible que necesites ejecutar el comando varias veces para generar correctamente el índice y las referencias cruzadas.

\end{itemize}


\subsubsection{4.1. Resultado esperado:}
\label{\detokenize{configuracion_inicial/009.Generar_PDF:id1}}
\sphinxAtStartPar
Al finalizar, verás un archivo \sphinxcode{\sphinxupquote{documento.pdf}} en la misma carpeta.

\sphinxAtStartPar
\sphinxstylestrong{Ejemplo de resultado final:}

\sphinxAtStartPar
\sphinxincludegraphics{{imagePDF2}.png}


\bigskip\hrule\bigskip



\subsection{5. Verificar y revisar el PDF}
\label{\detokenize{configuracion_inicial/009.Generar_PDF:verificar-y-revisar-el-pdf}}
\sphinxAtStartPar
Abre el archivo PDF con tu lector de PDF preferido y verifica que:
\begin{itemize}
\item {} 
\sphinxAtStartPar
La tabla de contenidos esté correctamente generada.

\item {} 
\sphinxAtStartPar
Las imágenes y tablas estén alineadas adecuadamente.

\item {} 
\sphinxAtStartPar
Las referencias cruzadas y enlaces funcionen correctamente.

\end{itemize}


\subsection{6. Solución de problemas comunes}
\label{\detokenize{configuracion_inicial/009.Generar_PDF:solucion-de-problemas-comunes}}

\subsubsection{6.1. Error: Paquete LaTeX faltante}
\label{\detokenize{configuracion_inicial/009.Generar_PDF:error-paquete-latex-faltante}}\begin{itemize}
\item {} 
\sphinxAtStartPar
Si ves mensajes de error sobre paquetes faltantes, instala los paquetes manualmente o permite que MiKTeX los descargue automáticamente.

\end{itemize}


\subsubsection{6.2. Error de codificación UTF\sphinxhyphen{}8}
\label{\detokenize{configuracion_inicial/009.Generar_PDF:error-de-codificacion-utf-8}}\begin{itemize}
\item {} 
\sphinxAtStartPar
Agrega esta línea en \sphinxcode{\sphinxupquote{conf.py}} dentro de \sphinxcode{\sphinxupquote{latex\_elements}}:

\begin{sphinxVerbatim}[commandchars=\\\{\}]
\PYG{l+s+s1}{\PYGZsq{}}\PYG{l+s+s1}{preamble}\PYG{l+s+s1}{\PYGZsq{}}\PYG{p}{:} \PYG{l+s+sa}{r}\PYG{l+s+s1}{\PYGZsq{}}\PYG{l+s+s1}{\PYGZbs{}}\PYG{l+s+s1}{usepackage[utf8]}\PYG{l+s+si}{\PYGZob{}inputenc\PYGZcb{}}\PYG{l+s+s1}{\PYGZsq{}} 
\end{sphinxVerbatim}

\end{itemize}


\subsubsection{6.3. Error de compilación en tablas o figuras}
\label{\detokenize{configuracion_inicial/009.Generar_PDF:error-de-compilacion-en-tablas-o-figuras}}\begin{itemize}
\item {} 
\sphinxAtStartPar
Asegúrate de que las tablas y figuras estén bien formateadas y no excedan los márgenes de la página.

\end{itemize}

\sphinxstepscope

\sphinxstepscope


\section{\sphinxstylestrong{Guía de uso MyST \textendash{} Parser}}
\label{\detokenize{configuracion_inicial/013.guia_de_myst_parser:guia-de-uso-myst-parser}}\label{\detokenize{configuracion_inicial/013.guia_de_myst_parser::doc}}

\subsection{1. Typography}
\label{\detokenize{configuracion_inicial/013.guia_de_myst_parser:typography}}
\sphinxAtStartPar
El soporte de tipografía en MyST\sphinxhyphen{}Parser permite a los usuarios crear contenido estilizado y bien formateado en documentos escritos en Markdown. MyST\sphinxhyphen{}Parser extiende Markdown estándar con características avanzadas que incluyen símbolos especiales, estilos de texto enriquecidos y más.


\subsubsection{Características principales de la tipografía en MyST\sphinxhyphen{}Parser}
\label{\detokenize{configuracion_inicial/013.guia_de_myst_parser:caracteristicas-principales-de-la-tipografia-en-myst-parser}}

\paragraph{1. \sphinxstylestrong{Énfasis de texto}}
\label{\detokenize{configuracion_inicial/013.guia_de_myst_parser:enfasis-de-texto}}
\sphinxAtStartPar
MyST\sphinxhyphen{}Parser utiliza una sintaxis similar a Markdown para resaltar texto con \sphinxstyleemphasis{énfasis} (cursiva) y \sphinxstylestrong{fuerte énfasis} (negrita).
\begin{itemize}
\item {} 
\sphinxAtStartPar
\sphinxstylestrong{Cursiva:} Usa un solo asterisco (\sphinxcode{\sphinxupquote{*}}) o guion bajo (\sphinxcode{\sphinxupquote{\_}}):

\begin{sphinxVerbatim}[commandchars=\\\{\}]
\PYG{g+ge}{*Texto en cursiva*} o \PYG{g+ge}{\PYGZus{}Texto en cursiva\PYGZus{}}
\end{sphinxVerbatim}

\sphinxAtStartPar
Resultado: \sphinxstyleemphasis{Texto en cursiva}

\item {} 
\sphinxAtStartPar
\sphinxstylestrong{Negrita:} Usa dos asteriscos (\sphinxcode{\sphinxupquote{**}}) o guiones bajos dobles (\sphinxcode{\sphinxupquote{\_\_}}):

\begin{sphinxVerbatim}[commandchars=\\\{\}]
\PYG{g+gs}{**Texto en negrita**} o \PYG{g+gs}{\PYGZus{}\PYGZus{}Texto en negrita\PYGZus{}\PYGZus{}}
\end{sphinxVerbatim}

\sphinxAtStartPar
Resultado: \sphinxstylestrong{Texto en negrita}

\item {} 
\sphinxAtStartPar
\sphinxstylestrong{Combinación de estilos:} Puedes combinar cursiva y negrita:

\begin{sphinxVerbatim}[commandchars=\\\{\}]
\PYG{g+gs}{**\PYGZus{}Texto combinado\PYGZus{}**}
\end{sphinxVerbatim}

\sphinxAtStartPar
Resultado: \sphinxstylestrong{\sphinxstyleemphasis{Texto combinado}}

\end{itemize}


\paragraph{2. \sphinxstylestrong{Texto tachado}}
\label{\detokenize{configuracion_inicial/013.guia_de_myst_parser:texto-tachado}}
\sphinxAtStartPar
Para tachar texto, usa dos tildes (\sphinxcode{\sphinxupquote{\textasciitilde{}\textasciitilde{}}}):

\begin{sphinxVerbatim}[commandchars=\\\{\}]
\PYG{g+gd}{\PYGZti{}\PYGZti{}Texto tachado\PYGZti{}\PYGZti{}}
\end{sphinxVerbatim}

\sphinxAtStartPar
Resultado: \textasciitilde{}\textasciitilde{}Texto tachado\textasciitilde{}\textasciitilde{}


\paragraph{3. \sphinxstylestrong{Citas tipográficas}}
\label{\detokenize{configuracion_inicial/013.guia_de_myst_parser:citas-tipograficas}}
\sphinxAtStartPar
Puedes usar \sphinxcode{\sphinxupquote{\textgreater{}}} para indicar una cita en bloque:

\begin{sphinxVerbatim}[commandchars=\\\{\}]
\PYG{k}{\PYGZgt{} }\PYG{g+ge}{Este es un bloque de cita.}
\end{sphinxVerbatim}

\sphinxAtStartPar
Resultado:
\begin{quote}

\sphinxAtStartPar
Este es un bloque de cita.
\end{quote}


\paragraph{4. \sphinxstylestrong{Subíndice y superíndice}}
\label{\detokenize{configuracion_inicial/013.guia_de_myst_parser:subindice-y-superindice}}
\sphinxAtStartPar
MyST\sphinxhyphen{}Parser incluye soporte para subíndices y superíndices utilizando notación inline.
\begin{itemize}
\item {} 
\sphinxAtStartPar
\sphinxstylestrong{Subíndice:} Usa \sphinxcode{\sphinxupquote{\textasciitilde{}}} para envolver el texto del subíndice:

\begin{sphinxVerbatim}[commandchars=\\\{\}]
H\PYGZti{}2\PYGZti{}O
\end{sphinxVerbatim}

\sphinxAtStartPar
Resultado: H\(\sb{\text{2}}\)O

\item {} 
\sphinxAtStartPar
\sphinxstylestrong{Superíndice:} Usa \sphinxcode{\sphinxupquote{\textasciicircum{}}} para envolver el texto del superíndice:

\begin{sphinxVerbatim}[commandchars=\\\{\}]
E = mc\PYGZca{}2\PYGZca{}
\end{sphinxVerbatim}

\sphinxAtStartPar
Resultado: E = mc\(\sp{\text{2}}\)

\end{itemize}


\paragraph{5. \sphinxstylestrong{Carácter de escape}}
\label{\detokenize{configuracion_inicial/013.guia_de_myst_parser:caracter-de-escape}}
\sphinxAtStartPar
Para incluir caracteres literales que podrían ser interpretados como sintaxis (por ejemplo, \sphinxcode{\sphinxupquote{*}}, \sphinxcode{\sphinxupquote{\_}} o \sphinxcode{\sphinxupquote{\textasciitilde{}}}), precede el carácter con una barra invertida (\sphinxcode{\sphinxupquote{\textbackslash{}}}):

\begin{sphinxVerbatim}[commandchars=\\\{\}]
Este es un asterisco literal: \PYGZbs{}*
\end{sphinxVerbatim}

\sphinxAtStartPar
Resultado: Este es un asterisco literal: *


\paragraph{6. \sphinxstylestrong{Separadores horizontales}}
\label{\detokenize{configuracion_inicial/013.guia_de_myst_parser:separadores-horizontales}}
\sphinxAtStartPar
Crea líneas horizontales para dividir contenido con tres o más guiones (\sphinxcode{\sphinxupquote{\sphinxhyphen{}\sphinxhyphen{}\sphinxhyphen{}}}), asteriscos (\sphinxcode{\sphinxupquote{***}}) o guiones bajos (\sphinxcode{\sphinxupquote{\_\_\_}}):

\begin{sphinxVerbatim}[commandchars=\\\{\}]
\PYGZhy{}\PYGZhy{}\PYGZhy{}
\end{sphinxVerbatim}

\sphinxAtStartPar
Resultado:


\bigskip\hrule\bigskip



\paragraph{7. \sphinxstylestrong{Letras y símbolos especiales}}
\label{\detokenize{configuracion_inicial/013.guia_de_myst_parser:letras-y-simbolos-especiales}}
\sphinxAtStartPar
MyST\sphinxhyphen{}Parser soporta caracteres Unicode directamente en los archivos Markdown. Además, puedes utilizar entidades HTML si lo necesitas:

\begin{sphinxVerbatim}[commandchars=\\\{\}]
Corazón: ❤️ o \PYGZam{}\PYGZsh{}x2764;
\end{sphinxVerbatim}

\sphinxAtStartPar
Resultado: Corazón: ❤️ o ❤


\bigskip\hrule\bigskip



\subsection{2. Admonitions}
\label{\detokenize{configuracion_inicial/013.guia_de_myst_parser:admonitions}}
\sphinxAtStartPar
Las admonitions son bloques especiales utilizados para resaltar contenido importante o específico, como notas, advertencias o ejemplos. MyST\sphinxhyphen{}Parser permite crear una amplia variedad de admonitions mediante una sintaxis intuitiva que mejora la claridad y el formato de los documentos.


\subsubsection{Características de las admonitions en MyST\sphinxhyphen{}Parser}
\label{\detokenize{configuracion_inicial/013.guia_de_myst_parser:caracteristicas-de-las-admonitions-en-myst-parser}}

\paragraph{1. \sphinxstylestrong{Sintaxis básica}}
\label{\detokenize{configuracion_inicial/013.guia_de_myst_parser:sintaxis-basica}}
\sphinxAtStartPar
Para crear una admonition, usa la siguiente sintaxis:

\begin{sphinxVerbatim}[commandchars=\\\{\}]
    ```\PYGZob{}admonition\PYGZcb{} Título opcional :class: type Contenido de la admonition. ```
\end{sphinxVerbatim}

\sphinxAtStartPar
Resultado:

\begin{sphinxadmonition}{note}{Título opcional}

\sphinxAtStartPar
Contenido de la admonition.
\end{sphinxadmonition}


\paragraph{2. \sphinxstylestrong{Tipos predefinidos de admonitions}}
\label{\detokenize{configuracion_inicial/013.guia_de_myst_parser:tipos-predefinidos-de-admonitions}}
\sphinxAtStartPar
MyST\sphinxhyphen{}Parser incluye varios tipos de admonitions predefinidas para diferentes propósitos. La sintaxis es similar a la básica, pero especificando el tipo de admonition.
\begin{itemize}
\item {} 
\sphinxAtStartPar
\sphinxstylestrong{Nota (\sphinxcode{\sphinxupquote{note}})}:

\begin{sphinxVerbatim}[commandchars=\\\{\}]
  ```\PYGZob{}admonition\PYGZcb{} Nota :class: note Esta es una nota informativa.```
\end{sphinxVerbatim}
\begin{itemize}
\item {} 
\sphinxAtStartPar
Así se vería:

\begin{sphinxadmonition}{note}{Nota}

\sphinxAtStartPar
Esta es una nota informativa.
\end{sphinxadmonition}

\end{itemize}

\item {} 
\sphinxAtStartPar
\sphinxstylestrong{Advertencia (\sphinxcode{\sphinxupquote{warning}})}:

\begin{sphinxVerbatim}[commandchars=\\\{\}]
  ```\PYGZob{}admonition\PYGZcb{} Advertencia :class: warning Ten cuidado con esta advertencia.```
\end{sphinxVerbatim}
\begin{itemize}
\item {} 
\sphinxAtStartPar
Así se vería:

\begin{sphinxadmonition}{note}{Advertencia}

\sphinxAtStartPar
Ten cuidado con esta advertencia.
\end{sphinxadmonition}

\end{itemize}

\item {} 
\sphinxAtStartPar
\sphinxstylestrong{Consejo (\sphinxcode{\sphinxupquote{tip}})}:

\begin{sphinxVerbatim}[commandchars=\\\{\}]
  ```\PYGZob{}admonition\PYGZcb{} Consejo :class: tip Aquí tienes un consejo útil. ```
\end{sphinxVerbatim}
\begin{itemize}
\item {} 
\sphinxAtStartPar
Así se vería:

\begin{sphinxadmonition}{note}{Consejo}

\sphinxAtStartPar
Aquí tienes un consejo útil.
\end{sphinxadmonition}

\end{itemize}

\item {} 
\sphinxAtStartPar
\sphinxstylestrong{Importante (\sphinxcode{\sphinxupquote{important}})}:

\begin{sphinxVerbatim}[commandchars=\\\{\}]
  ```\PYGZob{}admonition\PYGZcb{} Importante :class: important Esta información es crucial. ```
\end{sphinxVerbatim}
\begin{itemize}
\item {} 
\sphinxAtStartPar
Así se vería:

\begin{sphinxadmonition}{note}{Importante}

\sphinxAtStartPar
Esta información es crucial.
\end{sphinxadmonition}

\end{itemize}

\item {} 
\sphinxAtStartPar
\sphinxstylestrong{Peligro (\sphinxcode{\sphinxupquote{danger}})}:

\begin{sphinxVerbatim}[commandchars=\\\{\}]
  ```\PYGZob{}admonition\PYGZcb{} Peligroso :class: danger ¡Atención! Esta acción puede ser peligrosa. ```
\end{sphinxVerbatim}
\begin{itemize}
\item {} 
\sphinxAtStartPar
Así se vería:

\begin{sphinxadmonition}{note}{Peligroso}

\sphinxAtStartPar
¡Atención! Esta acción puede ser peligrosa.
\end{sphinxadmonition}

\end{itemize}

\item {} 
\sphinxAtStartPar
\sphinxstylestrong{Precaución (\sphinxcode{\sphinxupquote{caution}})}:

\begin{sphinxVerbatim}[commandchars=\\\{\}]
  ```\PYGZob{}admonition\PYGZcb{} Precaución :class: caution Ten cuidado al realizar esta acción, podría haber consecuencias inesperadas. ```
\end{sphinxVerbatim}
\begin{itemize}
\item {} 
\sphinxAtStartPar
Así se vería:

\begin{sphinxadmonition}{note}{Precaución}

\sphinxAtStartPar
Ten cuidado al realizar esta acción, podría haber consecuencias inesperadas.
\end{sphinxadmonition}

\end{itemize}

\item {} 
\sphinxAtStartPar
\sphinxstylestrong{Atención (\sphinxcode{\sphinxupquote{attention}})}:

\begin{sphinxVerbatim}[commandchars=\\\{\}]
  ```\PYGZob{}admonition\PYGZcb{} Atención :class: attetion Esta sección requiere tu atención inmediata para evitar errores críticos. ```
\end{sphinxVerbatim}
\begin{itemize}
\item {} 
\sphinxAtStartPar
Así se vería:

\begin{sphinxadmonition}{note}{Atención}

\sphinxAtStartPar
Esta sección requiere tu atención inmediata para evitar errores críticos.
\end{sphinxadmonition}

\end{itemize}

\item {} 
\sphinxAtStartPar
\sphinxstylestrong{Pista (\sphinxcode{\sphinxupquote{hint}})}:

\begin{sphinxVerbatim}[commandchars=\\\{\}]
  ```\PYGZob{}admonition\PYGZcb{} Pista :class: hint Aquí hay una sugerencia útil para facilitar tu tarea. ```
\end{sphinxVerbatim}
\begin{itemize}
\item {} 
\sphinxAtStartPar
Así se vería:

\begin{sphinxadmonition}{note}{Pista}

\sphinxAtStartPar
Aquí hay una sugerencia útil para facilitar tu tarea.
\end{sphinxadmonition}

\end{itemize}

\item {} 
\sphinxAtStartPar
\sphinxstylestrong{Precaución (\sphinxcode{\sphinxupquote{seealso }})}:

\begin{sphinxVerbatim}[commandchars=\\\{\}]
  ```\PYGZob{}admonition\PYGZcb{} Véase también :class: seealso  Consulta la sección anterior para obtener más contexto sobre este tema. ```
\end{sphinxVerbatim}
\begin{itemize}
\item {} 
\sphinxAtStartPar
Así se vería:

\begin{sphinxadmonition}{note}{Véase también}

\sphinxAtStartPar
Consulta la sección anterior para obtener más contexto sobre este tema.
\end{sphinxadmonition}

\end{itemize}

\end{itemize}


\paragraph{3. \sphinxstylestrong{Admonitions con títulos personalizados}}
\label{\detokenize{configuracion_inicial/013.guia_de_myst_parser:admonitions-con-titulos-personalizados}}
\sphinxAtStartPar
Puedes añadir un título personalizado a cualquier admonition especificando el título después del tipo:

\begin{sphinxVerbatim}[commandchars=\\\{\}]
  ```\PYGZob{}admonition\PYGZcb{} Título personalizado :class: tip Contenido con un título personalizado. ```
\end{sphinxVerbatim}

\sphinxAtStartPar
Resultado:

\begin{sphinxadmonition}{note}{Título personalizado}

\sphinxAtStartPar
Contenido con un título personalizado.
\end{sphinxadmonition}


\paragraph{4. \sphinxstylestrong{Admonitions anidadas}}
\label{\detokenize{configuracion_inicial/013.guia_de_myst_parser:admonitions-anidadas}}
\sphinxAtStartPar
Puedes anidar admonitions dentro de otras para mayor claridad. Por ejemplo:

\begin{sphinxVerbatim}[commandchars=\\\{\}]
    ```\PYGZob{}admonition\PYGZcb{} Nota :class: note Esto es un ejemplo de nota. 
      ```\PYGZob{}admonition\PYGZcb{} Tip :class: tip Esto es un ejemplo de tip anidado. ```
    ```
\end{sphinxVerbatim}

\sphinxAtStartPar
Resultado:

\begin{sphinxadmonition}{note}{Nota}

\sphinxAtStartPar
Esto es un ejemplo de nota.

\begin{sphinxadmonition}{note}{Tip}

\sphinxAtStartPar
Esto es un ejemplo de tip anidado. \textasciigrave{}\textasciigrave{}\textasciigrave{}
\end{sphinxadmonition}
\end{sphinxadmonition}


\paragraph{5. \sphinxstylestrong{Admonitions con bloques de código}}
\label{\detokenize{configuracion_inicial/013.guia_de_myst_parser:admonitions-con-bloques-de-codigo}}
\sphinxAtStartPar
Puedes incluir bloques de código dentro de una admonition:

\begin{sphinxVerbatim}[commandchars=\\\{\}]
```\PYGZob{}admonition\PYGZcb{} Ejemplo de código
:class: tip

Aquí tienes un bloque de código:

```python
print(\PYGZdq{}Hola, mundo!\PYGZdq{})``` 
\end{sphinxVerbatim}

\begin{sphinxVerbatim}[commandchars=\\\{\}]

Resultado:

``` \PYGZob{}admonition\PYGZcb{} Ejemplo de código
:class: tip

Aquí tienes un bloque de código:

    ```python
    print(\PYGZdq{}Hola, mundo!\PYGZdq{})```
\end{sphinxVerbatim}


\subsubsection{6. Personalización de admonitions}
\label{\detokenize{configuracion_inicial/013.guia_de_myst_parser:personalizacion-de-admonitions}}

\paragraph{\sphinxstylestrong{6.1. Entender la Estructura de Admonitions en Sphinx}}
\label{\detokenize{configuracion_inicial/013.guia_de_myst_parser:entender-la-estructura-de-admonitions-en-sphinx}}
\sphinxAtStartPar
En Sphinx, las admonitions son bloques especiales que resaltan contenido específico, como notas, advertencias o ejemplos. Estos bloques se generan en el HTML final con clases CSS específicas que permiten su personalización.

\sphinxAtStartPar
Algunos tipos comunes explicados anteriormente son:
\begin{itemize}
\item {} 
\sphinxAtStartPar
\sphinxcode{\sphinxupquote{note}}

\item {} 
\sphinxAtStartPar
\sphinxcode{\sphinxupquote{warning}}

\item {} 
\sphinxAtStartPar
\sphinxcode{\sphinxupquote{important}}

\item {} 
\sphinxAtStartPar
\sphinxcode{\sphinxupquote{hint}}

\item {} 
\sphinxAtStartPar
\sphinxcode{\sphinxupquote{caution}}

\item {} 
\sphinxAtStartPar
\sphinxcode{\sphinxupquote{attention}}

\item {} 
\sphinxAtStartPar
\sphinxcode{\sphinxupquote{seealso}}

\end{itemize}


\subparagraph{\sphinxstylestrong{6.1.1. Estructura HTML Generada por una Admonition}}
\label{\detokenize{configuracion_inicial/013.guia_de_myst_parser:estructura-html-generada-por-una-admonition}}
\sphinxAtStartPar
Al compilar la documentación, una admonition podría generarse como:

\begin{sphinxVerbatim}[commandchars=\\\{\}]
\PYG{p}{\PYGZlt{}}\PYG{n+nt}{div} \PYG{n+na}{class}\PYG{o}{=}\PYG{l+s}{\PYGZdq{}admonition note\PYGZdq{}}\PYG{p}{\PYGZgt{}}
  \PYG{p}{\PYGZlt{}}\PYG{n+nt}{p} \PYG{n+na}{class}\PYG{o}{=}\PYG{l+s}{\PYGZdq{}admonition\PYGZhy{}title\PYGZdq{}}\PYG{p}{\PYGZgt{}}Nota\PYG{p}{\PYGZlt{}}\PYG{p}{/}\PYG{n+nt}{p}\PYG{p}{\PYGZgt{}}
  \PYG{p}{\PYGZlt{}}\PYG{n+nt}{p}\PYG{p}{\PYGZgt{}}Este es un contenido dentro de una admonition de tipo \PYGZdq{}note\PYGZdq{}.\PYG{p}{\PYGZlt{}}\PYG{p}{/}\PYG{n+nt}{p}\PYG{p}{\PYGZgt{}}
\PYG{p}{\PYGZlt{}}\PYG{p}{/}\PYG{n+nt}{div}\PYG{p}{\PYGZgt{}}
\end{sphinxVerbatim}


\bigskip\hrule\bigskip



\paragraph{\sphinxstylestrong{6.2. Crear un Archivo de Estilos Personalizado en Sphinx}}
\label{\detokenize{configuracion_inicial/013.guia_de_myst_parser:crear-un-archivo-de-estilos-personalizado-en-sphinx}}

\subparagraph{\sphinxstylestrong{6.2.1. Estructura del Proyecto}}
\label{\detokenize{configuracion_inicial/013.guia_de_myst_parser:estructura-del-proyecto}}
\sphinxAtStartPar
Asegúrate de tener la siguiente estructura en tu proyecto Sphinx:

\begin{sphinxVerbatim}[commandchars=\\\{\}]
docs/
├──\PYG{+w}{ }\PYGZus{}static/
│\PYG{+w}{   }├──\PYG{+w}{ }custom.css\PYG{+w}{  }\PYG{c+c1}{\PYGZsh{} Archivo CSS personalizado}
├──\PYG{+w}{ }conf.py
├──\PYG{+w}{ }index.rst
└──\PYG{+w}{ }...
\end{sphinxVerbatim}


\subparagraph{\sphinxstylestrong{6.2.2. Editar el Archivo \sphinxstyleliteralintitle{\sphinxupquote{custom.css}}}}
\label{\detokenize{configuracion_inicial/013.guia_de_myst_parser:editar-el-archivo-custom-css}}
\sphinxAtStartPar
Abre (o crea) el archivo \sphinxcode{\sphinxupquote{\_static/custom.css}} y agrega estilos personalizados para las admonitions:

\begin{sphinxVerbatim}[commandchars=\\\{\}]
\PYG{c}{/* === Estilos Generales para todas las Admonitions === */}
\PYG{p}{.}\PYG{n+nc}{admonition}\PYG{+w}{ }\PYG{p}{\PYGZob{}}
\PYG{+w}{    }\PYG{k}{border\PYGZhy{}left}\PYG{p}{:}\PYG{+w}{ }\PYG{l+m+mi}{5}\PYG{k+kt}{px}\PYG{+w}{ }\PYG{k+kc}{solid}\PYG{+w}{ }\PYG{l+m+mh}{\PYGZsh{}444}\PYG{p}{;}
\PYG{+w}{    }\PYG{k}{border\PYGZhy{}radius}\PYG{p}{:}\PYG{+w}{ }\PYG{l+m+mi}{5}\PYG{k+kt}{px}\PYG{p}{;}
\PYG{+w}{    }\PYG{k}{margin}\PYG{p}{:}\PYG{+w}{ }\PYG{l+m+mi}{1}\PYG{k+kt}{em}\PYG{+w}{ }\PYG{l+m+mi}{0}\PYG{p}{;}
\PYG{+w}{    }\PYG{k}{padding}\PYG{p}{:}\PYG{+w}{ }\PYG{l+m+mi}{1}\PYG{k+kt}{em}\PYG{p}{;}
\PYG{+w}{    }\PYG{k}{background\PYGZhy{}color}\PYG{p}{:}\PYG{+w}{ }\PYG{l+m+mh}{\PYGZsh{}f9f9f9}\PYG{p}{;}
\PYG{p}{\PYGZcb{}}

\PYG{c}{/* === Título de las Admonitions === */}
\PYG{p}{.}\PYG{n+nc}{admonition}\PYG{+w}{ }\PYG{p}{.}\PYG{n+nc}{admonition\PYGZhy{}title}\PYG{+w}{ }\PYG{p}{\PYGZob{}}
\PYG{+w}{    }\PYG{k}{font\PYGZhy{}weight}\PYG{p}{:}\PYG{+w}{ }\PYG{k+kc}{bold}\PYG{p}{;}
\PYG{+w}{    }\PYG{k}{font\PYGZhy{}size}\PYG{p}{:}\PYG{+w}{ }\PYG{l+m+mf}{1.1}\PYG{k+kt}{em}\PYG{p}{;}
\PYG{+w}{    }\PYG{k}{margin\PYGZhy{}bottom}\PYG{p}{:}\PYG{+w}{ }\PYG{l+m+mf}{0.5}\PYG{k+kt}{em}\PYG{p}{;}
\PYG{p}{\PYGZcb{}}

\PYG{c}{/* === Personalización por Tipo de Admonition === */}

\PYG{c}{/* NOTE */}
\PYG{p}{.}\PYG{n+nc}{admonition}\PYG{p}{.}\PYG{n+nc}{note}\PYG{+w}{ }\PYG{p}{\PYGZob{}}
\PYG{+w}{    }\PYG{k}{border\PYGZhy{}left\PYGZhy{}color}\PYG{p}{:}\PYG{+w}{ }\PYG{l+m+mh}{\PYGZsh{}2196F3}\PYG{p}{;}
\PYG{+w}{    }\PYG{k}{background\PYGZhy{}color}\PYG{p}{:}\PYG{+w}{ }\PYG{l+m+mh}{\PYGZsh{}E3F2FD}\PYG{p}{;}
\PYG{p}{\PYGZcb{}}
\PYG{p}{.}\PYG{n+nc}{admonition}\PYG{p}{.}\PYG{n+nc}{note}\PYG{+w}{ }\PYG{p}{.}\PYG{n+nc}{admonition\PYGZhy{}title}\PYG{+w}{ }\PYG{p}{\PYGZob{}}
\PYG{+w}{    }\PYG{k}{color}\PYG{p}{:}\PYG{+w}{ }\PYG{l+m+mh}{\PYGZsh{}0D47A1}\PYG{p}{;}
\PYG{p}{\PYGZcb{}}

\PYG{c}{/* WARNING */}
\PYG{p}{.}\PYG{n+nc}{admonition}\PYG{p}{.}\PYG{n+nc}{warning}\PYG{+w}{ }\PYG{p}{\PYGZob{}}
\PYG{+w}{    }\PYG{k}{border\PYGZhy{}left\PYGZhy{}color}\PYG{p}{:}\PYG{+w}{ }\PYG{l+m+mh}{\PYGZsh{}FF9800}\PYG{p}{;}
\PYG{+w}{    }\PYG{k}{background\PYGZhy{}color}\PYG{p}{:}\PYG{+w}{ }\PYG{l+m+mh}{\PYGZsh{}FFF3E0}\PYG{p}{;}
\PYG{p}{\PYGZcb{}}
\PYG{p}{.}\PYG{n+nc}{admonition}\PYG{p}{.}\PYG{n+nc}{warning}\PYG{+w}{ }\PYG{p}{.}\PYG{n+nc}{admonition\PYGZhy{}title}\PYG{+w}{ }\PYG{p}{\PYGZob{}}
\PYG{+w}{    }\PYG{k}{color}\PYG{p}{:}\PYG{+w}{ }\PYG{l+m+mh}{\PYGZsh{}E65100}\PYG{p}{;}
\PYG{p}{\PYGZcb{}}

\PYG{c}{/* IMPORTANT */}
\PYG{p}{.}\PYG{n+nc}{admonition}\PYG{p}{.}\PYG{n+nc}{important}\PYG{+w}{ }\PYG{p}{\PYGZob{}}
\PYG{+w}{    }\PYG{k}{border\PYGZhy{}left\PYGZhy{}color}\PYG{p}{:}\PYG{+w}{ }\PYG{l+m+mh}{\PYGZsh{}4CAF50}\PYG{p}{;}
\PYG{+w}{    }\PYG{k}{background\PYGZhy{}color}\PYG{p}{:}\PYG{+w}{ }\PYG{l+m+mh}{\PYGZsh{}E8F5E9}\PYG{p}{;}
\PYG{p}{\PYGZcb{}}
\PYG{p}{.}\PYG{n+nc}{admonition}\PYG{p}{.}\PYG{n+nc}{important}\PYG{+w}{ }\PYG{p}{.}\PYG{n+nc}{admonition\PYGZhy{}title}\PYG{+w}{ }\PYG{p}{\PYGZob{}}
\PYG{+w}{    }\PYG{k}{color}\PYG{p}{:}\PYG{+w}{ }\PYG{l+m+mh}{\PYGZsh{}1B5E20}\PYG{p}{;}
\PYG{p}{\PYGZcb{}}

\PYG{c}{/* HINT */}
\PYG{p}{.}\PYG{n+nc}{admonition}\PYG{p}{.}\PYG{n+nc}{hint}\PYG{+w}{ }\PYG{p}{\PYGZob{}}
\PYG{+w}{    }\PYG{k}{border\PYGZhy{}left\PYGZhy{}color}\PYG{p}{:}\PYG{+w}{ }\PYG{l+m+mh}{\PYGZsh{}9C27B0}\PYG{p}{;}
\PYG{+w}{    }\PYG{k}{background\PYGZhy{}color}\PYG{p}{:}\PYG{+w}{ }\PYG{l+m+mh}{\PYGZsh{}F3E5F5}\PYG{p}{;}
\PYG{p}{\PYGZcb{}}
\PYG{p}{.}\PYG{n+nc}{admonition}\PYG{p}{.}\PYG{n+nc}{hint}\PYG{+w}{ }\PYG{p}{.}\PYG{n+nc}{admonition\PYGZhy{}title}\PYG{+w}{ }\PYG{p}{\PYGZob{}}
\PYG{+w}{    }\PYG{k}{color}\PYG{p}{:}\PYG{+w}{ }\PYG{l+m+mh}{\PYGZsh{}6A1B9A}\PYG{p}{;}
\PYG{p}{\PYGZcb{}}
\end{sphinxVerbatim}


\subparagraph{\sphinxstylestrong{6.2.3. Explicación de los Estilos:}}
\label{\detokenize{configuracion_inicial/013.guia_de_myst_parser:explicacion-de-los-estilos}}\begin{enumerate}
\sphinxsetlistlabels{\arabic}{enumi}{enumii}{}{.}%
\item {} 
\sphinxAtStartPar
\sphinxstylestrong{\sphinxcode{\sphinxupquote{.admonition}}}: Define un estilo base para todas las admonitions.

\item {} 
\sphinxAtStartPar
\sphinxstylestrong{\sphinxcode{\sphinxupquote{.admonition\sphinxhyphen{}title}}}: Estiliza los títulos dentro de las admonitions.

\item {} 
\sphinxAtStartPar
\sphinxstylestrong{\sphinxcode{\sphinxupquote{.admonition.note}}, \sphinxcode{\sphinxupquote{.admonition.warning}}, etc.}: Estiliza cada tipo de admonition con colores específicos para los bordes y el fondo.

\end{enumerate}


\bigskip\hrule\bigskip



\paragraph{\sphinxstylestrong{6.3. Configurar \sphinxstyleliteralintitle{\sphinxupquote{conf.py}} para Incluir CSS Personalizado}}
\label{\detokenize{configuracion_inicial/013.guia_de_myst_parser:configurar-conf-py-para-incluir-css-personalizado}}
\sphinxAtStartPar
Abre tu archivo \sphinxcode{\sphinxupquote{conf.py}} y verifica que la configuración sea correcta:

\begin{sphinxVerbatim}[commandchars=\\\{\}]
\PYG{c+c1}{\PYGZsh{} Habilitar myst\PYGZus{}parser para soporte Markdown}
\PYG{n}{extensions} \PYG{o}{=} \PYG{p}{[}
    \PYG{l+s+s1}{\PYGZsq{}}\PYG{l+s+s1}{myst\PYGZus{}parser}\PYG{l+s+s1}{\PYGZsq{}}\PYG{p}{,}
    \PYG{l+s+s1}{\PYGZsq{}}\PYG{l+s+s1}{sphinx.ext.autodoc}\PYG{l+s+s1}{\PYGZsq{}}\PYG{p}{,}
    \PYG{l+s+s1}{\PYGZsq{}}\PYG{l+s+s1}{sphinx.ext.napoleon}\PYG{l+s+s1}{\PYGZsq{}}\PYG{p}{,}
    \PYG{l+s+s1}{\PYGZsq{}}\PYG{l+s+s1}{sphinx.ext.viewcode}\PYG{l+s+s1}{\PYGZsq{}}\PYG{p}{,}
    \PYG{l+s+s1}{\PYGZsq{}}\PYG{l+s+s1}{sphinx\PYGZus{}rtd\PYGZus{}theme}\PYG{l+s+s1}{\PYGZsq{}}\PYG{p}{,}
\PYG{p}{]}

\PYG{c+c1}{\PYGZsh{} Añadir estilos CSS personalizados}
\PYG{n}{html\PYGZus{}static\PYGZus{}path} \PYG{o}{=} \PYG{p}{[}\PYG{l+s+s1}{\PYGZsq{}}\PYG{l+s+s1}{\PYGZus{}static}\PYG{l+s+s1}{\PYGZsq{}}\PYG{p}{]}
\PYG{n}{html\PYGZus{}css\PYGZus{}files} \PYG{o}{=} \PYG{p}{[}
    \PYG{l+s+s1}{\PYGZsq{}}\PYG{l+s+s1}{custom.css}\PYG{l+s+s1}{\PYGZsq{}}\PYG{p}{,}  \PYG{c+c1}{\PYGZsh{} Asegúrate de que custom.css esté en la carpeta \PYGZus{}static}
\PYG{p}{]}
\end{sphinxVerbatim}


\paragraph{\sphinxstylestrong{6.4. Configurar el tema (opcional)}}
\label{\detokenize{configuracion_inicial/013.guia_de_myst_parser:configurar-el-tema-opcional}}
\sphinxAtStartPar
html\_theme = “sphinx\_rtd\_theme”\textasciigrave{}
\begin{itemize}
\item {} 
\sphinxAtStartPar
\sphinxcode{\sphinxupquote{html\_static\_path}}: Especifica dónde buscar archivos estáticos.

\item {} 
\sphinxAtStartPar
\sphinxcode{\sphinxupquote{html\_css\_files}}: Carga el archivo \sphinxcode{\sphinxupquote{custom.css}} en la salida HTML.

\end{itemize}


\subsection{3. Images and Figures}
\label{\detokenize{configuracion_inicial/013.guia_de_myst_parser:images-and-figures}}
\sphinxAtStartPar
El manejo de \sphinxstylestrong{imágenes} y \sphinxstylestrong{figuras} en MyST\sphinxhyphen{}Parser permite enriquecer la documentación con contenido visual, facilitando la comprensión de conceptos complejos y mejorando la presentación general de tus documentos.

\sphinxAtStartPar
En esta guía, aprenderás:
\begin{enumerate}
\sphinxsetlistlabels{\arabic}{enumi}{enumii}{}{.}%
\item {} 
\sphinxAtStartPar
Cómo insertar imágenes básicas.

\item {} 
\sphinxAtStartPar
Cómo usar figuras con títulos y referencias.

\item {} 
\sphinxAtStartPar
Configurar opciones avanzadas para imágenes.

\item {} 
\sphinxAtStartPar
Personalizar estilos de imágenes y figuras.

\end{enumerate}


\subsubsection{\sphinxstylestrong{1. Imágenes Básicas en MyST\sphinxhyphen{}Parser}}
\label{\detokenize{configuracion_inicial/013.guia_de_myst_parser:imagenes-basicas-en-myst-parser}}

\paragraph{\sphinxstylestrong{1.1. Sintaxis Básica para Insertar una Imagen}}
\label{\detokenize{configuracion_inicial/013.guia_de_myst_parser:sintaxis-basica-para-insertar-una-imagen}}
\sphinxAtStartPar
En Markdown estándar, puedes insertar imágenes usando la siguiente sintaxis:

\begin{sphinxVerbatim}[commandchars=\\\{\}]
![\PYG{n+nt}{Texto alternativo}](\PYG{n+na}{ruta/a/la/imagen.png})
\end{sphinxVerbatim}
\begin{itemize}
\item {} 
\sphinxAtStartPar
\sphinxstylestrong{Texto alternativo:} Describe la imagen para accesibilidad y en caso de que no se cargue correctamente.

\item {} 
\sphinxAtStartPar
\sphinxstylestrong{Ruta de la imagen:} Puede ser una ruta relativa o absoluta.

\end{itemize}


\subparagraph{\sphinxstylestrong{Ejemplo:}}
\label{\detokenize{configuracion_inicial/013.guia_de_myst_parser:ejemplo}}
\begin{sphinxVerbatim}[commandchars=\\\{\}]
![\PYG{n+nt}{Logo de MyST}](\PYG{n+na}{../\PYGZus{}static/images/logo.png})
\end{sphinxVerbatim}

\sphinxAtStartPar
\sphinxstylestrong{Resultado:}

\sphinxAtStartPar
\sphinxincludegraphics{{logo}.png}


\bigskip\hrule\bigskip



\subsubsection{\sphinxstylestrong{2. Figuras con Título y Referencia}}
\label{\detokenize{configuracion_inicial/013.guia_de_myst_parser:figuras-con-titulo-y-referencia}}

\paragraph{\sphinxstylestrong{2.1. Insertar una Figura con Título}}
\label{\detokenize{configuracion_inicial/013.guia_de_myst_parser:insertar-una-figura-con-titulo}}
\sphinxAtStartPar
Una \sphinxstylestrong{figura} es una imagen con un título opcional y, en algunos casos, con referencias cruzadas para enlazarla desde otras partes del documento.

\begin{sphinxVerbatim}[commandchars=\\\{\}]
  ```\PYGZob{}figure\PYGZcb{} ../\PYGZus{}static/images/logo.png
  :alt: Logo de MyST
  :width: 200px
  :align: center

  Este es el logo oficial de MyST\PYGZhy{}Parser.```
\end{sphinxVerbatim}

\sphinxAtStartPar
\sphinxstylestrong{Resultado:}

\begin{figure}[htbp]
\centering
\capstart

\noindent\sphinxincludegraphics[width=200\sphinxpxdimen]{{logo}.png}
\caption{Este es el logo oficial de MyST\sphinxhyphen{}Parser.}\label{\detokenize{configuracion_inicial/013.guia_de_myst_parser:fig-logo}}\end{figure}


\subparagraph{\sphinxstylestrong{Explicación:}}
\label{\detokenize{configuracion_inicial/013.guia_de_myst_parser:explicacion}}\begin{itemize}
\item {} 
\sphinxAtStartPar
\sphinxcode{\sphinxupquote{:alt:}} Proporciona texto alternativo para la imagen.

\item {} 
\sphinxAtStartPar
\sphinxcode{\sphinxupquote{:width:}} Ajusta el ancho de la imagen (en píxeles o porcentaje).

\item {} 
\sphinxAtStartPar
\sphinxcode{\sphinxupquote{:align:}} Alinea la imagen (\sphinxcode{\sphinxupquote{center}}, \sphinxcode{\sphinxupquote{left}}, \sphinxcode{\sphinxupquote{right}}).

\item {} 
\sphinxAtStartPar
El contenido debajo de la directiva actúa como el \sphinxstylestrong{título de la figura}.

\end{itemize}


\paragraph{\sphinxstylestrong{2.2. Referencia Cruzada a una Figura}}
\label{\detokenize{configuracion_inicial/013.guia_de_myst_parser:referencia-cruzada-a-una-figura}}
\sphinxAtStartPar
Para referenciar la figura en otro punto del documento, añade la etiqueta \sphinxcode{\sphinxupquote{:name: Referencia}}:

\begin{sphinxVerbatim}[commandchars=\\\{\}]
  ```\PYGZob{}figure\PYGZcb{} ../\PYGZus{}static/images/logo.png
  :alt: Logo de MyST
  :width: 200px
  :align: center

  Este es el logo oficial de MyST\PYGZhy{}Parser.```
\end{sphinxVerbatim}

\sphinxAtStartPar
Puedes consultar el logo en la {\hyperref[\detokenize{configuracion_inicial/013.guia_de_myst_parser:fig-logo}]{\sphinxcrossref{Figura 1}}}.\textasciigrave{}

\sphinxAtStartPar
\sphinxstylestrong{Resultado:}

\sphinxAtStartPar
\sphinxincludegraphics{{logo}.png}

\sphinxAtStartPar
\sphinxstylestrong{Figura 1:} Este es el logo oficial de MyST\sphinxhyphen{}Parser.


\bigskip\hrule\bigskip



\subsubsection{\sphinxstylestrong{3. Opciones Avanzadas para Imágenes y Figuras}}
\label{\detokenize{configuracion_inicial/013.guia_de_myst_parser:opciones-avanzadas-para-imagenes-y-figuras}}
\sphinxAtStartPar
MyST\sphinxhyphen{}Parser ofrece más opciones para controlar el comportamiento de imágenes y figuras.


\paragraph{\sphinxstylestrong{3.1. Escalar Imágenes}}
\label{\detokenize{configuracion_inicial/013.guia_de_myst_parser:escalar-imagenes}}
\begin{sphinxVerbatim}[commandchars=\\\{\}]
  ```\PYGZob{}figure\PYGZcb{} ../\PYGZus{}static/images/logo.png
  :alt: Logo de MyST
  :scale: 50\PYGZpc{}

  Logo escalado al 50\PYGZpc{} de su tamaño original.```
\end{sphinxVerbatim}
\begin{itemize}
\item {} 
\sphinxAtStartPar
\sphinxstylestrong{\sphinxcode{\sphinxupquote{:scale:}}} Ajusta el tamaño de la imagen como un porcentaje de su tamaño original.

\end{itemize}


\bigskip\hrule\bigskip



\paragraph{\sphinxstylestrong{3.2. Alinear Imágenes}}
\label{\detokenize{configuracion_inicial/013.guia_de_myst_parser:alinear-imagenes}}
\begin{sphinxVerbatim}[commandchars=\\\{\}]
  ```\PYGZob{}figure\PYGZcb{} ../\PYGZus{}static/images/logo.png
  :alt: Logo de MyST
  :align: right

  Imagen alineada a la derecha.```
\end{sphinxVerbatim}
\begin{itemize}
\item {} 
\sphinxAtStartPar
\sphinxcode{\sphinxupquote{:align: center}} (centro)

\item {} 
\sphinxAtStartPar
\sphinxcode{\sphinxupquote{:align: left}} (izquierda)

\item {} 
\sphinxAtStartPar
\sphinxcode{\sphinxupquote{:align: right}} (derecha)

\end{itemize}


\bigskip\hrule\bigskip



\paragraph{\sphinxstylestrong{3.3. Enlazar Imágenes}}
\label{\detokenize{configuracion_inicial/013.guia_de_myst_parser:enlazar-imagenes}}
\sphinxAtStartPar
Puedes hacer que una imagen sea un enlace:

\begin{sphinxVerbatim}[commandchars=\\\{\}]
[\PYG{n+nt}{![Logo de MyST}](\PYG{n+na}{../\PYGZus{}static/images/logo.png})](https://myst\PYGZhy{}parser.readthedocs.io)
\end{sphinxVerbatim}

\sphinxAtStartPar
\sphinxstylestrong{Resultado:}

\sphinxAtStartPar
\sphinxhref{https://myst-parser.readthedocs.io}{\sphinxincludegraphics{{logo}.png}}


\subsection{4. Tables}
\label{\detokenize{configuracion_inicial/013.guia_de_myst_parser:tables}}
\sphinxAtStartPar
Las \sphinxstylestrong{tablas} son una herramienta esencial en la documentación técnica, ya que permiten organizar y presentar datos de forma clara y estructurada. \sphinxstylestrong{MyST\sphinxhyphen{}Parser} extiende la funcionalidad de Markdown para ofrecer opciones avanzadas de creación y personalización de tablas en \sphinxstylestrong{Sphinx}.


\bigskip\hrule\bigskip



\subsubsection{\sphinxstylestrong{1. Tablas Básicas en MyST\sphinxhyphen{}Parser}}
\label{\detokenize{configuracion_inicial/013.guia_de_myst_parser:tablas-basicas-en-myst-parser}}

\paragraph{\sphinxstylestrong{1.1. Sintaxis Básica de Tablas Markdown}}
\label{\detokenize{configuracion_inicial/013.guia_de_myst_parser:sintaxis-basica-de-tablas-markdown}}
\sphinxAtStartPar
Puedes crear tablas simples utilizando la sintaxis básica de Markdown:

\begin{sphinxVerbatim}[commandchars=\\\{\}]
| Encabezado 1 | Encabezado 2 | Encabezado 3 |
|\PYGZhy{}\PYGZhy{}\PYGZhy{}\PYGZhy{}\PYGZhy{}\PYGZhy{}\PYGZhy{}\PYGZhy{}\PYGZhy{}\PYGZhy{}\PYGZhy{}\PYGZhy{}\PYGZhy{}\PYGZhy{}|\PYGZhy{}\PYGZhy{}\PYGZhy{}\PYGZhy{}\PYGZhy{}\PYGZhy{}\PYGZhy{}\PYGZhy{}\PYGZhy{}\PYGZhy{}\PYGZhy{}\PYGZhy{}\PYGZhy{}\PYGZhy{}|\PYGZhy{}\PYGZhy{}\PYGZhy{}\PYGZhy{}\PYGZhy{}\PYGZhy{}\PYGZhy{}\PYGZhy{}\PYGZhy{}\PYGZhy{}\PYGZhy{}\PYGZhy{}\PYGZhy{}\PYGZhy{}|
| Celda 1     | Celda 2     | Celda 3     |
| Celda 4     | Celda 5     | Celda 6     |
\end{sphinxVerbatim}


\subparagraph{🛠️ \sphinxstylestrong{Explicación:}}
\label{\detokenize{configuracion_inicial/013.guia_de_myst_parser:id1}}\begin{itemize}
\item {} 
\sphinxAtStartPar
Las tuberías (\sphinxcode{\sphinxupquote{|}}) delimitan columnas.

\item {} 
\sphinxAtStartPar
La segunda línea (\sphinxcode{\sphinxupquote{\sphinxhyphen{}\sphinxhyphen{}\sphinxhyphen{}}}) define los encabezados.

\item {} 
\sphinxAtStartPar
Cada fila se define en una nueva línea.

\end{itemize}

\sphinxAtStartPar
\sphinxstylestrong{Resultado:}


\begin{savenotes}\sphinxattablestart
\sphinxthistablewithglobalstyle
\centering
\begin{tabulary}{\linewidth}[t]{TTT}
\sphinxtoprule
\sphinxstyletheadfamily 
\sphinxAtStartPar
Encabezado 1
&\sphinxstyletheadfamily 
\sphinxAtStartPar
Encabezado 2
&\sphinxstyletheadfamily 
\sphinxAtStartPar
Encabezado 3
\\
\sphinxmidrule
\sphinxtableatstartofbodyhook
\sphinxAtStartPar
Celda 1
&
\sphinxAtStartPar
Celda 2
&
\sphinxAtStartPar
Celda 3
\\
\sphinxhline
\sphinxAtStartPar
Celda 4
&
\sphinxAtStartPar
Celda 5
&
\sphinxAtStartPar
Celda 6
\\
\sphinxbottomrule
\end{tabulary}
\sphinxtableafterendhook\par
\sphinxattableend\end{savenotes}


\bigskip\hrule\bigskip



\subsubsection{\sphinxstylestrong{2. Tablas con Alineación}}
\label{\detokenize{configuracion_inicial/013.guia_de_myst_parser:tablas-con-alineacion}}

\paragraph{📌 \sphinxstylestrong{2.1. Alinear el Contenido de las Columnas}}
\label{\detokenize{configuracion_inicial/013.guia_de_myst_parser:alinear-el-contenido-de-las-columnas}}
\sphinxAtStartPar
Puedes controlar la alineación del texto dentro de las columnas usando los símbolos \sphinxcode{\sphinxupquote{:}}:

\begin{sphinxVerbatim}[commandchars=\\\{\}]
| Izquierda   | Centro     | Derecha   |
|:\PYGZhy{}\PYGZhy{}\PYGZhy{}\PYGZhy{}\PYGZhy{}\PYGZhy{}\PYGZhy{}\PYGZhy{}\PYGZhy{}\PYGZhy{}\PYGZhy{} |:\PYGZhy{}\PYGZhy{}\PYGZhy{}\PYGZhy{}\PYGZhy{}\PYGZhy{}\PYGZhy{}\PYGZhy{}\PYGZhy{}: |\PYGZhy{}\PYGZhy{}\PYGZhy{}\PYGZhy{}\PYGZhy{}\PYGZhy{}\PYGZhy{}\PYGZhy{}\PYGZhy{}: |
| Celda 1     | Celda 2    | Celda 3   |
| Celda 4     | Celda 5    | Celda 6   |
\end{sphinxVerbatim}


\subparagraph{🛠️ \sphinxstylestrong{Explicación:}}
\label{\detokenize{configuracion_inicial/013.guia_de_myst_parser:id2}}\begin{itemize}
\item {} 
\sphinxAtStartPar
\sphinxcode{\sphinxupquote{:\sphinxhyphen{}\sphinxhyphen{}\sphinxhyphen{}}} \(\rightarrow\) Alineación a la izquierda.

\item {} 
\sphinxAtStartPar
\sphinxcode{\sphinxupquote{:\sphinxhyphen{}\sphinxhyphen{}\sphinxhyphen{}:}} \(\rightarrow\) Alineación centrada.

\item {} 
\sphinxAtStartPar
\sphinxcode{\sphinxupquote{\sphinxhyphen{}\sphinxhyphen{}\sphinxhyphen{}:}} \(\rightarrow\) Alineación a la derecha.

\end{itemize}

\sphinxAtStartPar
\sphinxstylestrong{Resultado:}


\begin{savenotes}\sphinxattablestart
\sphinxthistablewithglobalstyle
\centering
\begin{tabulary}{\linewidth}[t]{TTT}
\sphinxtoprule
\sphinxstyletheadfamily 
\sphinxAtStartPar
Izquierda
&\sphinxstyletheadfamily 
\sphinxAtStartPar
Centro
&\sphinxstyletheadfamily 
\sphinxAtStartPar
Derecha
\\
\sphinxmidrule
\sphinxtableatstartofbodyhook
\sphinxAtStartPar
Celda 1
&
\sphinxAtStartPar
Celda 2
&
\sphinxAtStartPar
Celda 3
\\
\sphinxhline
\sphinxAtStartPar
Celda 4
&
\sphinxAtStartPar
Celda 5
&
\sphinxAtStartPar
Celda 6
\\
\sphinxbottomrule
\end{tabulary}
\sphinxtableafterendhook\par
\sphinxattableend\end{savenotes}


\bigskip\hrule\bigskip



\subsubsection{\sphinxstylestrong{3. Tablas Complejas con Formato Avanzado}}
\label{\detokenize{configuracion_inicial/013.guia_de_myst_parser:tablas-complejas-con-formato-avanzado}}

\paragraph{📌 \sphinxstylestrong{3.1. Crear Tablas con MyST Directivas}}
\label{\detokenize{configuracion_inicial/013.guia_de_myst_parser:crear-tablas-con-myst-directivas}}
\sphinxAtStartPar
MyST\sphinxhyphen{}Parser admite el uso de directivas para crear tablas más avanzadas con opciones adicionales:

\begin{sphinxVerbatim}[commandchars=\\\{\}]
  ```\PYGZob{}list\PYGZhy{}table\PYGZcb{} Título de la Tabla
  :header\PYGZhy{}rows: 1
  :widths: 20 40 40

\PYG{k}{*}\PYG{+w}{ }  \PYGZhy{} Treat
\PYG{+w}{    }\PYG{k}{\PYGZhy{}}\PYG{+w}{ }Quantity
\PYG{+w}{    }\PYG{k}{\PYGZhy{}}\PYG{+w}{ }Description
\PYG{k}{*}\PYG{+w}{ }  \PYGZhy{} Albatross
\PYG{+w}{    }\PYG{k}{\PYGZhy{}}\PYG{+w}{ }2.99
\PYG{+w}{    }\PYG{k}{\PYGZhy{}}\PYG{+w}{ }On a stick!
\PYG{k}{*}\PYG{+w}{ }  \PYGZhy{} Crunchy Frog
\PYG{+w}{    }\PYG{k}{\PYGZhy{}}\PYG{+w}{ }1.49
\PYG{+w}{    }\PYG{k}{\PYGZhy{}}\PYG{+w}{ }If we took the bones out, it wouldn\PYGZsq{}t be
 crunchy, now would it?
\PYG{k}{*}\PYG{+w}{ }  \PYGZhy{} Gannet Ripple
\PYG{+w}{    }\PYG{k}{\PYGZhy{}}\PYG{+w}{ }1.99
\PYG{+w}{    }\PYG{k}{\PYGZhy{}}\PYG{+w}{ }On a stick! ```
\end{sphinxVerbatim}


\subparagraph{\sphinxstylestrong{Explicación:}}
\label{\detokenize{configuracion_inicial/013.guia_de_myst_parser:id3}}\begin{itemize}
\item {} 
\sphinxAtStartPar
\sphinxcode{\sphinxupquote{:header\sphinxhyphen{}rows:}} \(\rightarrow\) Define cuántas filas son para encabezados (en este caso, 1).

\item {} 
\sphinxAtStartPar
\sphinxcode{\sphinxupquote{:widths:}} \(\rightarrow\) Define el ancho de cada columna (en porcentaje).

\item {} 
\sphinxAtStartPar
Cada fila de la tabla se define con \sphinxcode{\sphinxupquote{* \sphinxhyphen{}}}.

\end{itemize}

\sphinxAtStartPar
\sphinxstylestrong{Resultado:}


\begin{savenotes}\sphinxattablestart
\sphinxthistablewithglobalstyle
\centering
\sphinxcapstartof{table}
\sphinxthecaptionisattop
\sphinxcaption{Título de la Tabla}\label{\detokenize{configuracion_inicial/013.guia_de_myst_parser:id7}}
\sphinxaftertopcaption
\begin{tabular}[t]{\X{20}{100}\X{40}{100}\X{40}{100}}
\sphinxtoprule
\sphinxstyletheadfamily 
\sphinxAtStartPar
Treat
&\sphinxstyletheadfamily 
\sphinxAtStartPar
Quantity
&\sphinxstyletheadfamily 
\sphinxAtStartPar
Description
\\
\sphinxmidrule
\sphinxtableatstartofbodyhook
\sphinxAtStartPar
Albatross
&
\sphinxAtStartPar
2.99
&
\sphinxAtStartPar
On a stick!
\\
\sphinxhline
\sphinxAtStartPar
Crunchy Frog
&
\sphinxAtStartPar
1.49
&
\sphinxAtStartPar
If we took the bones out, it wouldn’t be
crunchy, now would it?
\\
\sphinxhline
\sphinxAtStartPar
Gannet Ripple
&
\sphinxAtStartPar
1.99
&
\sphinxAtStartPar
On a stick!
\\
\sphinxbottomrule
\end{tabular}
\sphinxtableafterendhook\par
\sphinxattableend\end{savenotes}


\bigskip\hrule\bigskip



\subsubsection{\sphinxstylestrong{4. Tablas con Referencias y Nombres}}
\label{\detokenize{configuracion_inicial/013.guia_de_myst_parser:tablas-con-referencias-y-nombres}}

\paragraph{\sphinxstylestrong{4.1. Agregar una Etiqueta a una Tabla}}
\label{\detokenize{configuracion_inicial/013.guia_de_myst_parser:agregar-una-etiqueta-a-una-tabla}}
\sphinxAtStartPar
Puedes etiquetar una tabla para referenciarla más adelante:

\begin{sphinxVerbatim}[commandchars=\\\{\}]
```\PYGZob{}list\PYGZhy{}table\PYGZcb{} Título de la Tabla
:header\PYGZhy{}rows: 1
:name: tabla\PYGZhy{}ejemplo

\PYG{k}{*}\PYG{+w}{ }\PYGZhy{} Encabezado A
\PYG{+w}{ }\PYG{k}{\PYGZhy{}}\PYG{+w}{ }Encabezado B
\PYG{k}{*}\PYG{+w}{ }\PYGZhy{} Valor 1
\PYG{+w}{ }\PYG{k}{\PYGZhy{}}\PYG{+w}{ }Valor 2```
\end{sphinxVerbatim}


\subparagraph{\sphinxstylestrong{Explicación:}}
\label{\detokenize{configuracion_inicial/013.guia_de_myst_parser:id4}}\begin{itemize}
\item {} 
\sphinxAtStartPar
\sphinxcode{\sphinxupquote{:name:}} Asigna un identificador único a la tabla.

\item {} 
\sphinxAtStartPar
Puedes referenciar la tabla utilizando \sphinxcode{\sphinxupquote{\#nombre\sphinxhyphen{}de\sphinxhyphen{}la\sphinxhyphen{}tabla}}.

\end{itemize}


\subsection{5. Source code and APIs}
\label{\detokenize{configuracion_inicial/013.guia_de_myst_parser:source-code-and-apis}}
\sphinxAtStartPar
El uso de código fuente y APIs en la documentación técnica es esencial para proporcionar ejemplos prácticos, fragmentos reutilizables y una referencia clara para desarrolladores. MyST\sphinxhyphen{}Parser, combinado con Sphinx, ofrece herramientas avanzadas para formatear, resaltar y documentar código fuente y APIs de forma clara y organizada.


\subsubsection{\sphinxstylestrong{1. Mostrar Código Fuente en Markdown}}
\label{\detokenize{configuracion_inicial/013.guia_de_myst_parser:mostrar-codigo-fuente-en-markdown}}

\paragraph{\sphinxstylestrong{1.1. Bloques de Código (Code Blocks)}}
\label{\detokenize{configuracion_inicial/013.guia_de_myst_parser:bloques-de-codigo-code-blocks}}
\sphinxAtStartPar
Para mostrar bloques de código en MyST\sphinxhyphen{}Parser, puedes usar la sintaxis estándar de Markdown con tres acentos graves (  ):

\begin{sphinxVerbatim}[commandchars=\\\{\}]
	```python
	def hola\PYGZus{}mundo():
	    print(\PYGZdq{}¡Hola, mundo!\PYGZdq{})```
\end{sphinxVerbatim}

\sphinxAtStartPar
\sphinxstylestrong{Resultado:}

\begin{sphinxVerbatim}[commandchars=\\\{\}]
\PYG{k}{def} \PYG{n+nf}{hola\PYGZus{}mundo}\PYG{p}{(}\PYG{p}{)}\PYG{p}{:}
    \PYG{n+nb}{print}\PYG{p}{(}\PYG{l+s+s2}{\PYGZdq{}}\PYG{l+s+s2}{¡Hola, mundo!}\PYG{l+s+s2}{\PYGZdq{}}\PYG{p}{)}
\end{sphinxVerbatim}
\begin{itemize}
\item {} 
\sphinxAtStartPar
Puedes usar otros lenguajes como \sphinxcode{\sphinxupquote{javascript}}, \sphinxcode{\sphinxupquote{html}}, \sphinxcode{\sphinxupquote{css}}, etc.

\end{itemize}


\bigskip\hrule\bigskip



\subsubsection{\sphinxstylestrong{2. Incrustar Código en Línea}}
\label{\detokenize{configuracion_inicial/013.guia_de_myst_parser:incrustar-codigo-en-linea}}
\sphinxAtStartPar
Puedes incrustar fragmentos de código dentro de una línea de texto utilizando acentos graves simples (\textasciigrave{}):

\begin{sphinxVerbatim}[commandchars=\\\{\}]
El comando \PYG{l+s+sb}{`print(\PYGZdq{}Hola\PYGZdq{})`} muestra un mensaje en la consola.
\end{sphinxVerbatim}

\sphinxAtStartPar
\sphinxstylestrong{Resultado:}\\
El comando \sphinxcode{\sphinxupquote{print("Hola")}} muestra un mensaje en la consola.


\bigskip\hrule\bigskip



\subsubsection{\sphinxstylestrong{3. Resaltar Fragmentos de Código (Highlighting)}}
\label{\detokenize{configuracion_inicial/013.guia_de_myst_parser:resaltar-fragmentos-de-codigo-highlighting}}
\sphinxAtStartPar
Puedes resaltar partes específicas de un bloque de código utilizando la directiva \sphinxstylestrong{\sphinxcode{\sphinxupquote{\{code\sphinxhyphen{}block\}}}} de MyST\sphinxhyphen{}Parser:

\begin{sphinxVerbatim}[commandchars=\\\{\}]
	```\PYGZob{}code\PYGZhy{}block\PYGZcb{} python
	:emphasize\PYGZhy{}lines: 2

	def suma(a, b):
	 resultado = a + b  \PYGZsh{} Esta línea está resaltada
	 return resultado```
\end{sphinxVerbatim}

\sphinxAtStartPar
\sphinxstylestrong{Resultado:}

\fvset{hllines={, 2,}}%
\begin{sphinxVerbatim}[commandchars=\\\{\}]
\PYG{k}{def} \PYG{n+nf}{suma}\PYG{p}{(}\PYG{n}{a}\PYG{p}{,} \PYG{n}{b}\PYG{p}{)}\PYG{p}{:}  
\PYG{n}{resultado} \PYG{o}{=} \PYG{n}{a} \PYG{o}{+} \PYG{n}{b} \PYG{c+c1}{\PYGZsh{} Esta línea está resaltada  }
\PYG{k}{return} \PYG{n}{resultado}  
\end{sphinxVerbatim}
\sphinxresetverbatimhllines


\paragraph{\sphinxstylestrong{Explicación:}}
\label{\detokenize{configuracion_inicial/013.guia_de_myst_parser:id5}}\begin{itemize}
\item {} 
\sphinxAtStartPar
\sphinxcode{\sphinxupquote{:emphasize\sphinxhyphen{}lines:}} \(\rightarrow\) Resalta líneas específicas dentro del bloque.

\item {} 
\sphinxAtStartPar
Puedes especificar varias líneas con \sphinxcode{\sphinxupquote{1,3}} o rangos con \sphinxcode{\sphinxupquote{1\sphinxhyphen{}3}}.

\end{itemize}


\bigskip\hrule\bigskip



\subsubsection{\sphinxstylestrong{4. Mostrar Archivos de Código Completo}}
\label{\detokenize{configuracion_inicial/013.guia_de_myst_parser:mostrar-archivos-de-codigo-completo}}
\sphinxAtStartPar
Si quieres incluir un archivo de código fuente completo en tu documentación, puedes usar la directiva \sphinxstylestrong{\sphinxcode{\sphinxupquote{literalinclude}}}:

\begin{sphinxVerbatim}[commandchars=\\\{\}]
```\PYGZob{}literalinclude\PYGZcb{} ../\PYGZus{}static/ejemplo.py
:language: python
:linenos:
:emphasize\PYGZhy{}lines: 2```
\end{sphinxVerbatim}


\paragraph{\sphinxstylestrong{Parámetros:}}
\label{\detokenize{configuracion_inicial/013.guia_de_myst_parser:parametros}}\begin{itemize}
\item {} 
\sphinxAtStartPar
\sphinxcode{\sphinxupquote{:language:}} \(\rightarrow\) Define el lenguaje para el resaltado de sintaxis.

\item {} 
\sphinxAtStartPar
\sphinxcode{\sphinxupquote{:linenos:}} \(\rightarrow\) Muestra números de línea.

\item {} 
\sphinxAtStartPar
\sphinxcode{\sphinxupquote{:emphasize\sphinxhyphen{}lines:}} \(\rightarrow\) Resalta líneas específicas.

\end{itemize}

\sphinxAtStartPar
\sphinxstylestrong{Estructura del Proyecto:}

\begin{sphinxVerbatim}[commandchars=\\\{\}]
docs/
├──\PYG{+w}{ }source/
│\PYG{+w}{   }├──\PYG{+w}{ }conf.py
│\PYG{+w}{   }├──\PYG{+w}{ }codigo/
│\PYG{+w}{   }│\PYG{+w}{   }├──\PYG{+w}{ }ejemplo.py
\end{sphinxVerbatim}

\sphinxAtStartPar
\sphinxstylestrong{ejemplo.py:}

\begin{sphinxVerbatim}[commandchars=\\\{\}]
\PYG{k}{def} \PYG{n+nf}{saludo}\PYG{p}{(}\PYG{p}{)}\PYG{p}{:}
    \PYG{n+nb}{print}\PYG{p}{(}\PYG{l+s+s2}{\PYGZdq{}}\PYG{l+s+s2}{¡Hola desde un archivo externo!}\PYG{l+s+s2}{\PYGZdq{}}\PYG{p}{)}
\end{sphinxVerbatim}

\sphinxAtStartPar
\sphinxstylestrong{Resultado en la documentación:}

\fvset{hllines={, 2,}}%
\begin{sphinxVerbatim}[commandchars=\\\{\},numbers=left,firstnumber=1,stepnumber=1]
\PYG{k}{def} \PYG{n+nf}{saludo}\PYG{p}{(}\PYG{p}{)}\PYG{p}{:}
    \PYG{n+nb}{print}\PYG{p}{(}\PYG{l+s+s2}{\PYGZdq{}}\PYG{l+s+s2}{¡Hola desde un archivo externo!}\PYG{l+s+s2}{\PYGZdq{}}\PYG{p}{)}
\end{sphinxVerbatim}
\sphinxresetverbatimhllines


\bigskip\hrule\bigskip



\subsubsection{\sphinxstylestrong{5. Documentación de APIs}}
\label{\detokenize{configuracion_inicial/013.guia_de_myst_parser:documentacion-de-apis}}

\paragraph{\sphinxstylestrong{5.1. Autogenerar Documentación de APIs}}
\label{\detokenize{configuracion_inicial/013.guia_de_myst_parser:autogenerar-documentacion-de-apis}}
\sphinxAtStartPar
\sphinxstylestrong{Sphinx} puede autogenerar documentación de APIs usando la extensión \sphinxcode{\sphinxupquote{sphinx.ext.autodoc}}.
\begin{enumerate}
\sphinxsetlistlabels{\arabic}{enumi}{enumii}{}{.}%
\item {} 
\sphinxAtStartPar
\sphinxstylestrong{Asegúrate de tener habilitada la extensión en \sphinxcode{\sphinxupquote{conf.py}}:}

\end{enumerate}

\begin{sphinxVerbatim}[commandchars=\\\{\}]
\PYG{n}{extensions} \PYG{o}{=} \PYG{p}{[}
    \PYG{l+s+s2}{\PYGZdq{}}\PYG{l+s+s2}{sphinx.ext.autodoc}\PYG{l+s+s2}{\PYGZdq{}}\PYG{p}{,}
    \PYG{l+s+s2}{\PYGZdq{}}\PYG{l+s+s2}{myst\PYGZus{}parser}\PYG{l+s+s2}{\PYGZdq{}}
\PYG{p}{]}
\end{sphinxVerbatim}
\begin{enumerate}
\sphinxsetlistlabels{\arabic}{enumi}{enumii}{}{.}%
\setcounter{enumi}{1}
\item {} 
\sphinxAtStartPar
\sphinxstylestrong{Crea un archivo \sphinxcode{\sphinxupquote{.rst}} o \sphinxcode{\sphinxupquote{.md}} para tu módulo:}

\end{enumerate}

\begin{sphinxVerbatim}[commandchars=\\\{\}]
\PYG{g+gh}{\PYGZsh{} Documentación de API}

\PYG{g+gu}{\PYGZsh{}\PYGZsh{} Módulo `mi\PYGZus{}modulo`}

```\PYGZob{}autofunction\PYGZcb{} mi\PYGZus{}modulo.saludo```
\end{sphinxVerbatim}
\begin{enumerate}
\sphinxsetlistlabels{\arabic}{enumi}{enumii}{}{.}%
\setcounter{enumi}{2}
\item {} 
\sphinxAtStartPar
\sphinxstylestrong{Estructura del Proyecto:}

\end{enumerate}

\begin{sphinxVerbatim}[commandchars=\\\{\}]
docs/
├──\PYG{+w}{ }source/
│\PYG{+w}{   }├──\PYG{+w}{ }conf.py
│\PYG{+w}{   }├──\PYG{+w}{ }mi\PYGZus{}modulo.py
\end{sphinxVerbatim}

\sphinxAtStartPar
\sphinxstylestrong{mi\_modulo.py:}

\begin{sphinxVerbatim}[commandchars=\\\{\}]
\PYG{k}{def} \PYG{n+nf}{saludo}\PYG{p}{(}\PYG{p}{)}\PYG{p}{:}
\PYG{+w}{    }\PYG{l+s+sd}{\PYGZdq{}\PYGZdq{}\PYGZdq{}Esta función imprime un saludo.\PYGZdq{}\PYGZdq{}\PYGZdq{}}
    \PYG{n+nb}{print}\PYG{p}{(}\PYG{l+s+s2}{\PYGZdq{}}\PYG{l+s+s2}{¡Hola desde la API!}\PYG{l+s+s2}{\PYGZdq{}}\PYG{p}{)}
\end{sphinxVerbatim}
\begin{enumerate}
\sphinxsetlistlabels{\arabic}{enumi}{enumii}{}{.}%
\setcounter{enumi}{3}
\item {} 
\sphinxAtStartPar
\sphinxstylestrong{Compila la Documentación:}

\end{enumerate}

\begin{sphinxVerbatim}[commandchars=\\\{\}]
make\PYG{+w}{ }html
\end{sphinxVerbatim}


\subsection{6. Cross\sphinxhyphen{}references}
\label{\detokenize{configuracion_inicial/013.guia_de_myst_parser:cross-references}}
\sphinxAtStartPar
Las \sphinxstylestrong{referencias cruzadas (cross\sphinxhyphen{}references)} permiten enlazar diferentes secciones, encabezados, archivos, imágenes, tablas, funciones y otros elementos dentro de tu documentación. Esto facilita la navegación y mantiene la coherencia en documentos extensos.


\subsubsection{\sphinxstylestrong{1. Referencias Cruzadas a Secciones}}
\label{\detokenize{configuracion_inicial/013.guia_de_myst_parser:referencias-cruzadas-a-secciones}}

\paragraph{\sphinxstylestrong{1.1. Sintaxis Básica}}
\label{\detokenize{configuracion_inicial/013.guia_de_myst_parser:id6}}
\sphinxAtStartPar
En \sphinxstylestrong{MyST\sphinxhyphen{}Parser}, puedes referenciar encabezados usando etiquetas (\sphinxcode{\sphinxupquote{\#}}) y referencias directas:

\begin{sphinxVerbatim}[commandchars=\\\{\}]
(sec\PYGZhy{}mi\PYGZhy{}seccion)=
\PYG{g+gu}{\PYGZsh{}\PYGZsh{} 📍 Mi Sección Importante}

Puedes referenciar esta sección usando:

[\PYG{n+nt}{Enlace a la sección}](\PYG{n+na}{\PYGZsh{}sec\PYGZhy{}mi\PYGZhy{}seccion})
\end{sphinxVerbatim}
\begin{itemize}
\item {} 
\sphinxAtStartPar
\sphinxcode{\sphinxupquote{(sec\sphinxhyphen{}mi\sphinxhyphen{}seccion)=}}: Crea una etiqueta para la sección.

\item {} 
\sphinxAtStartPar
\sphinxcode{\sphinxupquote{{[}Enlace a la sección{]}(\#sec\sphinxhyphen{}mi\sphinxhyphen{}seccion)}}: Enlaza a la etiqueta.

\end{itemize}


\paragraph{\sphinxstylestrong{1.2. Referencia desde otro archivo}}
\label{\detokenize{configuracion_inicial/013.guia_de_myst_parser:referencia-desde-otro-archivo}}
\sphinxAtStartPar
Si la sección está en otro archivo, usa:

\begin{sphinxVerbatim}[commandchars=\\\{\}]
[\PYG{n+nt}{Ir a Mi Sección}](\PYG{n+na}{otro\PYGZus{}archivo.md\PYGZsh{}sec\PYGZhy{}mi\PYGZhy{}seccion})
\end{sphinxVerbatim}


\bigskip\hrule\bigskip



\subsubsection{\sphinxstylestrong{2. Referencias Cruzadas a Archivos}}
\label{\detokenize{configuracion_inicial/013.guia_de_myst_parser:referencias-cruzadas-a-archivos}}
\sphinxAtStartPar
Puedes enlazar directamente a otros archivos Markdown:

\begin{sphinxVerbatim}[commandchars=\\\{\}]
[\PYG{n+nt}{Ir a la Guía de Estilo}](\PYG{n+na}{guia\PYGZus{}estilo.md})
\end{sphinxVerbatim}


\bigskip\hrule\bigskip



\subsubsection{\sphinxstylestrong{3. Referencias a Funciones, Clases y Métodos (Autodoc)}}
\label{\detokenize{configuracion_inicial/013.guia_de_myst_parser:referencias-a-funciones-clases-y-metodos-autodoc}}
\sphinxAtStartPar
Si estás documentando con \sphinxstylestrong{\sphinxcode{\sphinxupquote{sphinx.ext.autodoc}}}, puedes enlazar elementos de código automáticamente.

\begin{sphinxVerbatim}[commandchars=\\\{\}]
Consulte la función [\PYG{l+s+sb}{`suma`}](codigo.ejemplo\PYGZus{}autodoc.suma) para más detalles.
\end{sphinxVerbatim}

\sphinxAtStartPar
También puedes usar referencias automáticas:

\begin{sphinxVerbatim}[commandchars=\\\{\}]
:func:`codigo.ejemplo\PYGZus{}autodoc.suma`
:class:`codigo.ejemplo\PYGZus{}autodoc.Calculadora`
\end{sphinxVerbatim}


\bigskip\hrule\bigskip



\subsubsection{\sphinxstylestrong{4. Referencias a Figuras e Imágenes}}
\label{\detokenize{configuracion_inicial/013.guia_de_myst_parser:referencias-a-figuras-e-imagenes}}
\sphinxAtStartPar
Al agregar una imagen con una etiqueta, puedes referenciarla más adelante:

\begin{sphinxVerbatim}[commandchars=\\\{\}]
(fig\PYGZhy{}ejemplo)=
![\PYG{n+nt}{Ejemplo de Imagen}](\PYG{n+na}{imagen.png})

Puedes ver la figura anterior aquí: [\PYG{n+nt}{Figura 1}](\PYG{n+na}{\PYGZsh{}fig\PYGZhy{}ejemplo})
\end{sphinxVerbatim}


\bigskip\hrule\bigskip



\subsubsection{\sphinxstylestrong{5. Referencias a Tablas}}
\label{\detokenize{configuracion_inicial/013.guia_de_myst_parser:referencias-a-tablas}}
\sphinxAtStartPar
De manera similar, puedes etiquetar tablas para referencias cruzadas:

\begin{sphinxVerbatim}[commandchars=\\\{\}]
(tab\PYGZhy{}ejemplo)=
| Encabezado 1 | Encabezado 2 |
|\PYGZhy{}\PYGZhy{}\PYGZhy{}\PYGZhy{}\PYGZhy{}\PYGZhy{}\PYGZhy{}\PYGZhy{}\PYGZhy{}\PYGZhy{}\PYGZhy{}\PYGZhy{}\PYGZhy{}|\PYGZhy{}\PYGZhy{}\PYGZhy{}\PYGZhy{}\PYGZhy{}\PYGZhy{}\PYGZhy{}\PYGZhy{}\PYGZhy{}\PYGZhy{}\PYGZhy{}\PYGZhy{}\PYGZhy{}|
| Valor 1    | Valor 2    |

Referencia a la tabla: [\PYG{n+nt}{Tabla 1}](\PYG{n+na}{\PYGZsh{}tab\PYGZhy{}ejemplo})
\end{sphinxVerbatim}


\bigskip\hrule\bigskip



\subsubsection{\sphinxstylestrong{6. Referencias con la Directiva \sphinxstyleliteralintitle{\sphinxupquote{ref}}}}
\label{\detokenize{configuracion_inicial/013.guia_de_myst_parser:referencias-con-la-directiva-ref}}
\sphinxAtStartPar
También puedes utilizar la directiva \sphinxcode{\sphinxupquote{ref}} de Sphinx:

\begin{sphinxVerbatim}[commandchars=\\\{\}]
Consulte la sección :ref: \PYG{l+s+sb}{`sec\PYGZhy{}mi\PYGZhy{}seccion`} para más detalles.
\end{sphinxVerbatim}
\begin{itemize}
\item {} 
\sphinxAtStartPar
\sphinxcode{\sphinxupquote{:ref:}} permite una referencia robusta y puede redirigir aunque cambie el nombre del archivo.

\item {} 
\sphinxAtStartPar
Usa etiquetas únicas para evitar conflictos.

\end{itemize}


\bigskip\hrule\bigskip



\subsubsection{\sphinxstylestrong{7. Referencias a Admonitions}}
\label{\detokenize{configuracion_inicial/013.guia_de_myst_parser:referencias-a-admonitions}}
\sphinxAtStartPar
Si usas etiquetas en admonitions, también puedes enlazarlas:

\begin{sphinxVerbatim}[commandchars=\\\{\}]
(adm\PYGZhy{}importante)=
	```\PYGZob{}admonition\PYGZcb{} Nota Importante
	Este es un mensaje importante.```
Puedes revisar la [\PYG{n+nt}{Nota Importante}](\PYG{n+na}{\PYGZsh{}adm\PYGZhy{}importante}).
\end{sphinxVerbatim}


\bigskip\hrule\bigskip



\subsubsection{\sphinxstylestrong{8. Buenas Prácticas para Referencias Cruzadas}}
\label{\detokenize{configuracion_inicial/013.guia_de_myst_parser:buenas-practicas-para-referencias-cruzadas}}\begin{itemize}
\item {} 
\sphinxAtStartPar
🏷️ \sphinxstylestrong{Etiqueta todo}: Usa etiquetas únicas para secciones, imágenes, tablas y bloques importantes.

\item {} 
\sphinxAtStartPar
🔗 \sphinxstylestrong{Prefijos útiles}: Usa prefijos claros, como \sphinxcode{\sphinxupquote{sec\sphinxhyphen{}}}, \sphinxcode{\sphinxupquote{fig\sphinxhyphen{}}}, \sphinxcode{\sphinxupquote{tab\sphinxhyphen{}}}, \sphinxcode{\sphinxupquote{adm\sphinxhyphen{}}}.

\item {} 
\sphinxAtStartPar
📝 \sphinxstylestrong{Evita enlaces rotos}: Usa el comando \sphinxcode{\sphinxupquote{make linkcheck}} para verificar enlaces en tu documentación.

\end{itemize}


\bigskip\hrule\bigskip



\subsubsection{\sphinxstylestrong{9. Verificar Referencias Cruzadas}}
\label{\detokenize{configuracion_inicial/013.guia_de_myst_parser:verificar-referencias-cruzadas}}
\sphinxAtStartPar
Al compilar tu documentación:

\begin{sphinxVerbatim}[commandchars=\\\{\}]
make\PYG{+w}{ }clean
make\PYG{+w}{ }html
make\PYG{+w}{ }linkcheck
\end{sphinxVerbatim}
\begin{itemize}
\item {} 
\sphinxAtStartPar
\sphinxcode{\sphinxupquote{make linkcheck}}: Verifica que todos los enlaces y referencias funcionen correctamente.

\end{itemize}


\subsection{7. Organising\sphinxhyphen{}Content}
\label{\detokenize{configuracion_inicial/013.guia_de_myst_parser:organising-content}}
\sphinxAtStartPar
Sphinx te permite organizar tu contenido en múltiples documentos e incluir contenido de otros documentos.

\sphinxAtStartPar
Esta sección describe cómo hacerlo con MyST Markdown.


\subsubsection{1 Estructura del documento}
\label{\detokenize{configuracion_inicial/013.guia_de_myst_parser:estructura-del-documento}}
\sphinxAtStartPar
Los documentos de MyST Markdown individuales están estructurados utilizando {\hyperref[\detokenize{configuracion_inicial/013.guia_de_myst_parser:tipografia.md:encabezados}]{\sphinxcrossref{\DUrole{xref}{\DUrole{myst}{encabezados}}}}}.

\sphinxAtStartPar
Todos los encabezados en el nivel raíz del documento se incluyen en la Tabla de Contenidos (ToC) de esa página.

\sphinxAtStartPar
Muchos temas de HTML ya incluyen esta ToC en una barra lateral, pero también puedes incluirla en el contenido principal de la página usando la directiva \sphinxcode{\sphinxupquote{contents}}:

\sphinxAtStartPar
Opciones:
\begin{itemize}
\item {} 
\sphinxAtStartPar
\sphinxstylestrong{:depth:} Especifica la profundidad de la ToC.

\item {} 
\sphinxAtStartPar
\sphinxstylestrong{:local:} Solo incluye encabezados de la sección actual del documento.

\item {} 
\sphinxAtStartPar
\sphinxstylestrong{:backlinks:} Incluye un enlace a la ToC al final de cada sección.

\item {} 
\sphinxAtStartPar
\sphinxstylestrong{:class:} Permite añadir una clase CSS personalizada a la ToC.

\end{itemize}

\begin{sphinxadmonition}{warning}{Advertencia:}
\sphinxAtStartPar
Debido a que la estructura del documento se determina mediante los encabezados, es problemático tener encabezados «no consecutivos» en un documento, como:

\begin{sphinxVerbatim}[commandchars=\\\{\}]
\PYG{g+gh}{\PYGZsh{} Encabezado 1}
\PYG{g+gu}{\PYGZsh{}\PYGZsh{}\PYGZsh{} Encabezado 3}
\end{sphinxVerbatim}

\sphinxAtStartPar
Si deseas incluir un encabezado así, puedes usar la extensión {\hyperref[\detokenize{configuracion_inicial/013.guia_de_myst_parser:syntax/attributes/block}]{\sphinxcrossref{\DUrole{xref}{\DUrole{myst}{attrs\_block}}}}} para envolver el encabezado en un \sphinxcode{\sphinxupquote{div}}, de modo que no se incluya en la ToC:

\begin{sphinxVerbatim}[commandchars=\\\{\}]
\PYG{g+gh}{\PYGZsh{} Encabezado 1}
\end{sphinxVerbatim}
\subsubsection*{Encabezado 3}

\begin{sphinxVerbatim}[commandchars=\\\{\}]

\end{sphinxVerbatim}
\end{sphinxadmonition}

\begin{sphinxadmonition}{note}{Configurar un título en el front\sphinxhyphen{}matter}

\sphinxAtStartPar
\DUrole{versionmodified}{\DUrole{added}{Added in version 0.17.0.}}

\sphinxAtStartPar
La configuración \sphinxcode{\sphinxupquote{myst\_title\_to\_header = True}} permite que una clave \sphinxcode{\sphinxupquote{title}} esté presente en el {\hyperref[\detokenize{configuracion_inicial/013.guia_de_myst_parser:syntax/frontmatter}]{\sphinxcrossref{\DUrole{xref}{\DUrole{myst}{front matter del documento}}}}}.

\sphinxAtStartPar
Esto será utilizado como el encabezado del documento (analizado como Markdown). Por ejemplo:

\begin{sphinxVerbatim}[commandchars=\\\{\}]
\PYGZhy{}\PYGZhy{}\PYGZhy{}
\PYG{g+gu}{title: Mi Título con *énfasis*}
\PYG{g+gu}{\PYGZhy{}\PYGZhy{}\PYGZhy{}}
\end{sphinxVerbatim}

\sphinxAtStartPar
sería equivalente a:

\begin{sphinxVerbatim}[commandchars=\\\{\}]
\PYG{g+gh}{\PYGZsh{} Mi Título con *énfasis*}
\end{sphinxVerbatim}
\end{sphinxadmonition}


\subsubsection{2 Insertar otros documentos directamente en el documento actual}
\label{\detokenize{configuracion_inicial/013.guia_de_myst_parser:insertar-otros-documentos-directamente-en-el-documento-actual}}
\sphinxAtStartPar
La directiva \sphinxcode{\sphinxupquote{include}} permite insertar el contenido de otro documento directamente en el flujo del documento actual.


\begin{sphinxseealso}{Ver también:}

\sphinxAtStartPar
La directiva {\hyperref[\detokenize{configuracion_inicial/013.guia_de_myst_parser:syntax/literalinclude}]{\sphinxcrossref{\DUrole{xref}{\DUrole{myst}{\sphinxcode{\sphinxupquote{literalinclude}}}}}}} para incluir código fuente desde archivos.


\end{sphinxseealso}


\sphinxAtStartPar
Las siguientes opciones permiten incluir solo una parte de un documento:
\begin{itemize}
\item {} 
\sphinxAtStartPar
\sphinxstylestrong{:start\sphinxhyphen{}line:} Solo incluye el contenido a partir de este número de línea (la numeración comienza en 1, y los números negativos cuentan desde el final).

\item {} 
\sphinxAtStartPar
\sphinxstylestrong{:end\sphinxhyphen{}line:} Solo incluye el contenido hasta (pero excluyendo) esta línea.

\item {} 
\sphinxAtStartPar
\sphinxstylestrong{:start\sphinxhyphen{}after:} Solo incluye el contenido después de la primera aparición del texto especificado.

\item {} 
\sphinxAtStartPar
\sphinxstylestrong{:end\sphinxhyphen{}before:} Solo incluye el contenido antes de la primera aparición del texto especificado (pero después de cualquier texto posterior).

\end{itemize}

\sphinxAtStartPar
Las siguientes opciones permiten modificar el contenido del documento incluido, para hacerlo relativo a la ubicación donde se está insertando:
\begin{itemize}
\item {} 
\sphinxAtStartPar
\sphinxstylestrong{:heading\sphinxhyphen{}offset:} Desplaza todos los niveles de encabezado por este número entero positivo, por ejemplo, cambiando \sphinxcode{\sphinxupquote{\#}} a \sphinxcode{\sphinxupquote{\#\#\#\#}}.

\item {} 
\sphinxAtStartPar
\sphinxstylestrong{:relative\sphinxhyphen{}docs:} Convierte las referencias a archivos Markdown en relativas al documento actual si comienzan con un cierto prefijo.

\item {} 
\sphinxAtStartPar
\sphinxstylestrong{:relative\sphinxhyphen{}images:} Convierte las referencias a imágenes Markdown en relativas al documento actual.

\end{itemize}

\sphinxAtStartPar
Opciones adicionales:
\begin{itemize}
\item {} 
\sphinxAtStartPar
\sphinxstylestrong{:encoding:} La codificación de texto del archivo externo.

\end{itemize}

\begin{sphinxadmonition}{note}{Inclusión desde/hacia archivos reStructuredText}

\sphinxAtStartPar
Como se explica en {\hyperref[\detokenize{configuracion_inicial/013.guia_de_myst_parser:syntax/directives/parsing}]{\sphinxcrossref{\DUrole{xref}{\DUrole{myst}{esta sección}}}}}, todas las directivas de MyST analizan su contenido como Markdown.
Para incluir rST, primero debemos «envolver» la directiva en la directiva {\hyperref[\detokenize{configuracion_inicial/013.guia_de_myst_parser:syntax/directives/parsing}]{\sphinxcrossref{\DUrole{xref}{\DUrole{myst}{eval\sphinxhyphen{}rst}}}}}:

\sphinxAtStartPar
Para incluir un archivo MyST dentro de un archivo reStructuredText, podemos usar la opción \sphinxcode{\sphinxupquote{parser}} de la directiva \sphinxcode{\sphinxupquote{include}}:

\begin{sphinxVerbatim}[commandchars=\\\{\}]
\PYG{p}{..} \PYG{o+ow}{include}\PYG{p}{::} include.md
   \PYG{n+nc}{:parser:} myst\PYGZus{}parser.sphinx\PYGZus{}
\end{sphinxVerbatim}

\begin{sphinxadmonition}{important}{Importante:}
\sphinxAtStartPar
La opción \sphinxcode{\sphinxupquote{parser}} requiere \sphinxcode{\sphinxupquote{docutils\textgreater{}=0.17}}.
\end{sphinxadmonition}
\end{sphinxadmonition}


\subsubsection{3 Usar toctree para incluir otros documentos como hijos}
\label{\detokenize{configuracion_inicial/013.guia_de_myst_parser:usar-toctree-para-incluir-otros-documentos-como-hijos}}
\sphinxAtStartPar
Para estructurar un proyecto con múltiples documentos, se utiliza la .

\sphinxAtStartPar
Esto designa documentos como hijos del documento actual, construyendo una jerarquía de documentos anidada que comienza desde un .

\sphinxAtStartPar
Opciones del \sphinxcode{\sphinxupquote{toctree}}:
\begin{itemize}
\item {} 
\sphinxAtStartPar
\sphinxstylestrong{:glob:} Coincide con todos los documentos disponibles y los inserta alfabéticamente.

\end{itemize}

\sphinxAtStartPar
Opciones de visualización dentro del documento:
\begin{itemize}
\item {} 
\sphinxAtStartPar
\sphinxstylestrong{:caption:} Un título para este \sphinxcode{\sphinxupquote{toctree}}.

\item {} 
\sphinxAtStartPar
\sphinxstylestrong{:hidden:} No se muestra dentro del documento.

\item {} 
\sphinxAtStartPar
\sphinxstylestrong{:includehidden:} Incluye entradas \sphinxcode{\sphinxupquote{toctree}} de hijos ocultos.

\item {} 
\sphinxAtStartPar
\sphinxstylestrong{:maxdepth:} Profundidad de los sub\sphinxhyphen{}encabezados del documento mostrados.

\item {} 
\sphinxAtStartPar
\sphinxstylestrong{:titlesonly:} Solo muestra el primer encabezado de nivel superior.

\item {} 
\sphinxAtStartPar
\sphinxstylestrong{:reversed:} Invierte el orden de las entradas en la lista.

\end{itemize}

\sphinxAtStartPar
Opciones adicionales:
\begin{itemize}
\item {} 
\sphinxAtStartPar
\sphinxstylestrong{:name:} Permite referenciar el \sphinxcode{\sphinxupquote{toctree}}.

\item {} 
\sphinxAtStartPar
\sphinxstylestrong{:numbered:} Numera todos los encabezados en los hijos. Si se especifica un entero, este será la profundidad a numerar.

\end{itemize}

\begin{sphinxadmonition}{tip}{Truco:}
\sphinxAtStartPar
Los sub\sphinxhyphen{}árboles (\sphinxcode{\sphinxupquote{sub\sphinxhyphen{}toctrees}}) se numeran automáticamente, así que no des la bandera \sphinxcode{\sphinxupquote{:numbered:}} a estos.
\end{sphinxadmonition}


\subsection{8. Sintax\sphinxhyphen{}Extension}
\label{\detokenize{configuracion_inicial/013.guia_de_myst_parser:sintax-extension}}
\sphinxAtStartPar
El MyST\sphinxhyphen{}Parser es altamente configurable, utilizando la «plugabilidad» inherente del analizador .
Las siguientes sintaxis son opcionales (deshabilitadas por defecto) y se pueden habilitar \sphinxstyleemphasis{a través de} la configuración \sphinxcode{\sphinxupquote{conf.py}} de Sphinx.
Su objetivo es generalmente añadir sintaxis más «amigables con Markdown»; a menudo habilitando y renderizando  que amplían la \sphinxhref{https://commonmark.org/}{especificación de CommonMark}.

\sphinxAtStartPar
Para habilitar todas las sintaxis explicadas a continuación:

\begin{sphinxVerbatim}[commandchars=\\\{\}]
\PYG{n}{myst\PYGZus{}enable\PYGZus{}extensions} \PYG{o}{=} \PYG{p}{[}
    \PYG{l+s+s2}{\PYGZdq{}}\PYG{l+s+s2}{amsmath}\PYG{l+s+s2}{\PYGZdq{}}\PYG{p}{,}
    \PYG{l+s+s2}{\PYGZdq{}}\PYG{l+s+s2}{attrs\PYGZus{}inline}\PYG{l+s+s2}{\PYGZdq{}}\PYG{p}{,}
    \PYG{l+s+s2}{\PYGZdq{}}\PYG{l+s+s2}{colon\PYGZus{}fence}\PYG{l+s+s2}{\PYGZdq{}}\PYG{p}{,}
    \PYG{l+s+s2}{\PYGZdq{}}\PYG{l+s+s2}{deflist}\PYG{l+s+s2}{\PYGZdq{}}\PYG{p}{,}
    \PYG{l+s+s2}{\PYGZdq{}}\PYG{l+s+s2}{dollarmath}\PYG{l+s+s2}{\PYGZdq{}}\PYG{p}{,}
    \PYG{l+s+s2}{\PYGZdq{}}\PYG{l+s+s2}{fieldlist}\PYG{l+s+s2}{\PYGZdq{}}\PYG{p}{,}
    \PYG{l+s+s2}{\PYGZdq{}}\PYG{l+s+s2}{html\PYGZus{}admonition}\PYG{l+s+s2}{\PYGZdq{}}\PYG{p}{,}
    \PYG{l+s+s2}{\PYGZdq{}}\PYG{l+s+s2}{html\PYGZus{}image}\PYG{l+s+s2}{\PYGZdq{}}\PYG{p}{,}
    \PYG{l+s+s2}{\PYGZdq{}}\PYG{l+s+s2}{linkify}\PYG{l+s+s2}{\PYGZdq{}}\PYG{p}{,}
    \PYG{l+s+s2}{\PYGZdq{}}\PYG{l+s+s2}{replacements}\PYG{l+s+s2}{\PYGZdq{}}\PYG{p}{,}
    \PYG{l+s+s2}{\PYGZdq{}}\PYG{l+s+s2}{smartquotes}\PYG{l+s+s2}{\PYGZdq{}}\PYG{p}{,}
    \PYG{l+s+s2}{\PYGZdq{}}\PYG{l+s+s2}{strikethrough}\PYG{l+s+s2}{\PYGZdq{}}\PYG{p}{,}
    \PYG{l+s+s2}{\PYGZdq{}}\PYG{l+s+s2}{substitution}\PYG{l+s+s2}{\PYGZdq{}}\PYG{p}{,}
    \PYG{l+s+s2}{\PYGZdq{}}\PYG{l+s+s2}{tasklist}\PYG{l+s+s2}{\PYGZdq{}}\PYG{p}{,}
\PYG{p}{]}
\end{sphinxVerbatim}

\sphinxAtStartPar
\DUrole{versionmodified}{\DUrole{changed}{Distinto en la versión 0.13.0: }}\sphinxcode{\sphinxupquote{myst\_enable\_extensions}} reemplaza las opciones de configuración anteriores:
\sphinxcode{\sphinxupquote{admonition\_enable}}, \sphinxcode{\sphinxupquote{figure\_enable}}, \sphinxcode{\sphinxupquote{dmath\_enable}}, \sphinxcode{\sphinxupquote{amsmath\_enable}}, \sphinxcode{\sphinxupquote{deflist\_enable}}, \sphinxcode{\sphinxupquote{html\_img\_enable}}.


\subsubsection{1 Tipografía}
\label{\detokenize{configuracion_inicial/013.guia_de_myst_parser:tipografia}}
\sphinxAtStartPar
Añadiendo \sphinxcode{\sphinxupquote{"smartquotes"}} a \sphinxcode{\sphinxupquote{myst\_enable\_extensions}} (en el \sphinxcode{\sphinxupquote{conf.py}}) se convertirán automáticamente las comillas estándar en variantes de apertura/cierre:
\begin{itemize}
\item {} 
\sphinxAtStartPar
\sphinxcode{\sphinxupquote{\textquotesingle{}comillas simples\textquotesingle{}}}: “comillas simples”

\item {} 
\sphinxAtStartPar
\sphinxcode{\sphinxupquote{"comillas dobles"}}: «comillas dobles»

\end{itemize}

\sphinxAtStartPar
Añadiendo \sphinxcode{\sphinxupquote{"replacements"}} a \sphinxcode{\sphinxupquote{myst\_enable\_extensions}} se convertirán automáticamente algunos textos tipográficos comunes:


\begin{savenotes}\sphinxattablestart
\sphinxthistablewithglobalstyle
\centering
\begin{tabulary}{\linewidth}[t]{TT}
\sphinxtoprule
\sphinxstyletheadfamily 
\sphinxAtStartPar
Texto
&\sphinxstyletheadfamily 
\sphinxAtStartPar
Convertido
\\
\sphinxmidrule
\sphinxtableatstartofbodyhook
\sphinxAtStartPar
\sphinxcode{\sphinxupquote{(c)}}, \sphinxcode{\sphinxupquote{(C)}}
&
\sphinxAtStartPar
(c)
\\
\sphinxhline
\sphinxAtStartPar
\sphinxcode{\sphinxupquote{(tm)}}, \sphinxcode{\sphinxupquote{(TM)}}
&
\sphinxAtStartPar
(tm)
\\
\sphinxhline
\sphinxAtStartPar
\sphinxcode{\sphinxupquote{(r)}}, \sphinxcode{\sphinxupquote{(R)}}
&
\sphinxAtStartPar
(r)
\\
\sphinxhline
\sphinxAtStartPar
\sphinxcode{\sphinxupquote{(p)}}, \sphinxcode{\sphinxupquote{(P)}}
&
\sphinxAtStartPar
(p)
\\
\sphinxhline
\sphinxAtStartPar
\sphinxcode{\sphinxupquote{+\sphinxhyphen{}}}
&
\sphinxAtStartPar
+\sphinxhyphen{}
\\
\sphinxhline
\sphinxAtStartPar
\sphinxcode{\sphinxupquote{...}}
&
\sphinxAtStartPar
…
\\
\sphinxhline
\sphinxAtStartPar
\sphinxcode{\sphinxupquote{,,,}}
&
\sphinxAtStartPar
,,,
\\
\sphinxbottomrule
\end{tabulary}
\sphinxtableafterendhook\par
\sphinxattableend\end{savenotes}


\subsubsection{2 Matemáticas}
\label{\detokenize{configuracion_inicial/013.guia_de_myst_parser:matematicas}}
\sphinxAtStartPar
La matemática se analiza agregando a la lista \sphinxcode{\sphinxupquote{myst\_enable\_extensions}} en el archivo \sphinxcode{\sphinxupquote{conf.py}} una o ambas de las siguientes extensiones:
\begin{itemize}
\item {} 
\sphinxAtStartPar
\sphinxcode{\sphinxupquote{"dollarmath"}} para análisis de matemáticas delimitadas por \sphinxcode{\sphinxupquote{\$}} y \sphinxcode{\sphinxupquote{\$\$}}.

\item {} 
\sphinxAtStartPar
\sphinxcode{\sphinxupquote{"amsmath"}} para análisis directo de entornos LaTeX de \sphinxhref{https://ctan.org/pkg/amsmath}{amsmath}.

\end{itemize}

\sphinxAtStartPar
Esto permite el análisis de matemáticas como:
\begin{itemize}
\item {} 
\sphinxAtStartPar
Matemáticas en línea: \sphinxcode{\sphinxupquote{\$...\$}}

\item {} 
\sphinxAtStartPar
Matemáticas en bloque: \sphinxcode{\sphinxupquote{\$\$...\$\$}}

\end{itemize}

\sphinxAtStartPar
Por ejemplo, \sphinxcode{\sphinxupquote{\$x\_\{hey\}=it+is\textasciicircum{}\{math\}\$}} se renderiza como (x\_\{hey\}=it+is\textasciicircum{}\{math\}).

\sphinxAtStartPar
El bloque de matemáticas se especifica con \sphinxcode{\sphinxupquote{\$\$}} y puede incluir etiquetas:

\begin{sphinxVerbatim}[commandchars=\\\{\}]
\PYGZdl{}\PYGZdl{}
e = mc\PYGZca{}2
\PYGZdl{}\PYGZdl{} (eqn:mejor)

Esta es la mejor ecuación \PYGZob{}eq\PYGZcb{}`eqn:mejor`.
\end{sphinxVerbatim}

\sphinxAtStartPar
Opciones adicionales permiten controlar cómo se analiza el contenido de las matemáticas, como \sphinxcode{\sphinxupquote{myst\_dmath\_allow\_space}} y \sphinxcode{\sphinxupquote{myst\_dmath\_allow\_digits}}.


\subsubsection{3 Listas de Definiciones}
\label{\detokenize{configuracion_inicial/013.guia_de_myst_parser:listas-de-definiciones}}
\sphinxAtStartPar
Añadiendo \sphinxcode{\sphinxupquote{"deflist"}} a \sphinxcode{\sphinxupquote{myst\_enable\_extensions}}, puedes utilizar listas de definiciones.

\begin{sphinxVerbatim}[commandchars=\\\{\}]
Término
: Definición

Otro término
: Definición extendida
\end{sphinxVerbatim}

\sphinxAtStartPar
Las definiciones pueden incluir elementos de bloque como párrafos, citas o bloques de código.


\subsubsection{4 Listas de Tareas}
\label{\detokenize{configuracion_inicial/013.guia_de_myst_parser:listas-de-tareas}}
\sphinxAtStartPar
Añadiendo \sphinxcode{\sphinxupquote{"tasklist"}} a \sphinxcode{\sphinxupquote{myst\_enable\_extensions}}, puedes utilizar listas de tareas:

\begin{sphinxVerbatim}[commandchars=\\\{\}]
\PYG{k}{\PYGZhy{} }\PYG{k}{[ ]} Tarea pendiente
\PYG{k}{\PYGZhy{} }\PYG{k}{[x]} Tarea completada
\end{sphinxVerbatim}


\subsubsection{5 Sustituciones}
\label{\detokenize{configuracion_inicial/013.guia_de_myst_parser:sustituciones}}
\sphinxAtStartPar
Añadiendo \sphinxcode{\sphinxupquote{"substitution"}} a \sphinxcode{\sphinxupquote{myst\_enable\_extensions}}, puedes definir sustituciones, que se agregan al archivo \sphinxcode{\sphinxupquote{conf.py}} o al encabezado del archivo:

\begin{sphinxVerbatim}[commandchars=\\\{\}]
\PYG{n+nn}{\PYGZhy{}\PYGZhy{}\PYGZhy{}}
\PYG{n+nt}{myst}\PYG{p}{:}
\PYG{+w}{  }\PYG{n+nt}{substitutions}\PYG{p}{:}
\PYG{+w}{    }\PYG{n+nt}{key1}\PYG{p}{:}\PYG{+w}{ }\PYG{l+s}{\PYGZdq{}}\PYG{l+s}{Soy}\PYG{n+nv}{ }\PYG{l+s}{una}\PYG{n+nv}{ }\PYG{l+s}{**sustitución**}\PYG{l+s}{\PYGZdq{}}
\PYG{+w}{    }\PYG{n+nt}{key2}\PYG{p}{:}\PYG{+w}{ }\PYG{p+pIndicator}{|}
\PYG{+w}{      }\PYG{n+no}{```\PYGZob{}note\PYGZcb{}}
\PYG{+w}{      }\PYG{n+no}{Sustitución anidada: \PYGZob{}\PYGZob{} key1 \PYGZcb{}\PYGZcb{}}
\PYG{+w}{      }\PYG{n+no}{```}
\PYG{n+nn}{\PYGZhy{}\PYGZhy{}\PYGZhy{}}
\end{sphinxVerbatim}

\sphinxAtStartPar
Estas sustituciones se pueden utilizar en línea o como bloques y admiten anidamiento.


\subsubsection{6 Anclajes Automáticos}
\label{\detokenize{configuracion_inicial/013.guia_de_myst_parser:anclajes-automaticos}}
\sphinxAtStartPar
El MyST Parser puede generar automáticamente «slugs» para anclas de encabezados, lo que permite referenciarlos directamente:

\begin{sphinxVerbatim}[commandchars=\\\{\}]
\PYG{n}{myst\PYGZus{}heading\PYGZus{}anchors} \PYG{o}{=} \PYG{l+m+mi}{3}
\end{sphinxVerbatim}

\sphinxAtStartPar
Esto habilita enlaces como \sphinxcode{\sphinxupquote{{[}{]}(\#mi\sphinxhyphen{}ancla)}} para referenciar encabezados en Markdown.


\subsection{9. Roles y Directivas}
\label{\detokenize{configuracion_inicial/013.guia_de_myst_parser:roles-y-directivas}}
\sphinxAtStartPar
Los roles y las directivas ofrecen una manera de extender la sintaxis de MyST de manera ilimitada,
interpretando un fragmento de texto como un tipo específico de marcado, según su nombre.

\sphinxAtStartPar
Casi todos los \sphinxhref{https://docutils.sourceforge.io/docs/ref/rst/roles.html}{roles de docutils},
\sphinxhref{https://docutils.sourceforge.io/docs/ref/rst/directives.html}{directivas de docutils},
, o 
se pueden usar en MyST.


\subsubsection{1. Sintaxis}
\label{\detokenize{configuracion_inicial/013.guia_de_myst_parser:sintaxis}}

\paragraph{1.1 Directivas \sphinxhyphen{} una extensión a nivel de bloque}
\label{\detokenize{configuracion_inicial/013.guia_de_myst_parser:directivas-una-extension-a-nivel-de-bloque}}
\sphinxAtStartPar
La sintaxis de las directivas se define con triples de acentos graves y llaves.
Es efectivamente un bloque de código de Markdown con llaves alrededor del lenguaje, y un nombre de directiva en lugar de un nombre de lenguaje.
Aquí está la estructura básica:

\begin{sphinxVerbatim}[commandchars=\\\{\}]
```\PYGZob{}directivename\PYGZcb{} argumentos
:key1: valor1
:key2: valor2

Este es el contenido de la directiva
\end{sphinxVerbatim}

\sphinxAtStartPar
Por ejemplo:

\begin{sphinxadmonition}{note}{Esta es mi advertencia}

\sphinxAtStartPar
Este es mi contenido
\end{sphinxadmonition}


\paragraph{1.2 Configuración de las directivas (opciones)}
\label{\detokenize{configuracion_inicial/013.guia_de_myst_parser:configuracion-de-las-directivas-opciones}}
\sphinxAtStartPar
Muchas directivas pueden tomar pares clave/valor en un bloque de opciones opcional al inicio de la directiva.

\fvset{hllines={, 1, 3,}}%
\begin{sphinxVerbatim}[commandchars=\\\{\},numbers=left,firstnumber=10,stepnumber=1]
\PYG{n}{a} \PYG{o}{=} \PYG{l+m+mi}{2}
\PYG{n+nb}{print}\PYG{p}{(}\PYG{l+s+s1}{\PYGZsq{}}\PYG{l+s+s1}{mi primera línea}\PYG{l+s+s1}{\PYGZsq{}}\PYG{p}{)}
\PYG{n+nb}{print}\PYG{p}{(}\PYG{l+s+sa}{f}\PYG{l+s+s1}{\PYGZsq{}}\PYG{l+s+s1}{mi }\PYG{l+s+si}{\PYGZob{}}\PYG{n}{a}\PYG{l+s+si}{\PYGZcb{}}\PYG{l+s+s1}{nda línea}\PYG{l+s+s1}{\PYGZsq{}}\PYG{p}{)}
\end{sphinxVerbatim}
\sphinxresetverbatimhllines

\sphinxAtStartPar
Las opciones también pueden incluir valores multilínea, o estar delimitadas con \sphinxcode{\sphinxupquote{\sphinxhyphen{}\sphinxhyphen{}\sphinxhyphen{}}} en lugar de \sphinxcode{\sphinxupquote{:}}.


\paragraph{1.3 Roles \sphinxhyphen{} un punto de extensión en línea}
\label{\detokenize{configuracion_inicial/013.guia_de_myst_parser:roles-un-punto-de-extension-en-linea}}
\sphinxAtStartPar
Los roles son similares a las directivas, pero se usan en línea.
Para definir un rol en línea, usa la siguiente forma:

\begin{sphinxVerbatim}[commandchars=\\\{\}]
\PYGZob{}nombre\PYGZhy{}del\PYGZhy{}rol\PYGZcb{}`contenido del rol`
\end{sphinxVerbatim}

\sphinxAtStartPar
Por ejemplo:

\begin{sphinxVerbatim}[commandchars=\\\{\}]
Sabemos desde Pitágoras que \PYGZob{}math\PYGZcb{}`a\PYGZca{}2 + b\PYGZca{}2 = c\PYGZca{}2`.
\end{sphinxVerbatim}


\subsubsection{2. Roles y Directivas de MyST}
\label{\detokenize{configuracion_inicial/013.guia_de_myst_parser:roles-y-directivas-de-myst}}

\paragraph{2.1 Insertar la fecha y el tiempo de lectura}
\label{\detokenize{configuracion_inicial/013.guia_de_myst_parser:insertar-la-fecha-y-el-tiempo-de-lectura}}
\sphinxAtStartPar
\DUrole{versionmodified}{\DUrole{added}{Added in version 0.14.0: }}El rol \sphinxcode{\sphinxupquote{sub\sphinxhyphen{}ref}} y el conteo de palabras.

\sphinxAtStartPar
Puedes insertar la fecha de «última actualización» y el tiempo estimado de lectura en tu documento mediante sustituciones accesibles \sphinxstyleemphasis{a través de} \sphinxcode{\sphinxupquote{sub\sphinxhyphen{}ref}}.

\sphinxAtStartPar
Por ejemplo:

\begin{sphinxVerbatim}[commandchars=\\\{\}]
\PYG{k}{\PYGZgt{} }\PYG{g+ge}{\PYGZob{}sub\PYGZhy{}ref\PYGZcb{}`today` | \PYGZob{}sub\PYGZhy{}ref\PYGZcb{}`wordcount\PYGZhy{}words` palabras | \PYGZob{}sub\PYGZhy{}ref\PYGZcb{}`wordcount\PYGZhy{}minutes` min de lectura}
\end{sphinxVerbatim}

\sphinxAtStartPar
\sphinxcode{\sphinxupquote{today}} se reemplaza por la fecha en la que se analiza el documento, y el tiempo de lectura se calcula usando la configuración \sphinxcode{\sphinxupquote{myst\_words\_per\_minute}}.

\sphinxstepscope


\section{Comandos utilizados al trabajar con Markdown, MyST\sphinxhyphen{}Parser, Sphinx}
\label{\detokenize{comandos_mas_usados/comandos:comandos-utilizados-al-trabajar-con-markdown-myst-parser-sphinx}}\label{\detokenize{comandos_mas_usados/comandos::doc}}

\subsection{\sphinxstylestrong{1. Crear un entorno virtual}}
\label{\detokenize{comandos_mas_usados/comandos:crear-un-entorno-virtual}}
\begin{sphinxVerbatim}[commandchars=\\\{\}]
\PYG{n}{python} \PYG{o}{\PYGZhy{}}\PYG{n}{m} \PYG{n}{venv} \PYG{n}{venv}
\end{sphinxVerbatim}


\subsection{\sphinxstylestrong{2. Activar el entorno virtual}}
\label{\detokenize{comandos_mas_usados/comandos:activar-el-entorno-virtual}}
\begin{sphinxVerbatim}[commandchars=\\\{\}]
\PYG{n}{venv}\PYGZbs{}\PYG{n}{Scripts}\PYGZbs{}\PYG{n}{activate}
\end{sphinxVerbatim}


\subsection{\sphinxstylestrong{3. Configuración inicial del proyecto Sphinx}}
\label{\detokenize{comandos_mas_usados/comandos:configuracion-inicial-del-proyecto-sphinx}}

\subsubsection{\sphinxstylestrong{Comando para iniciar un proyecto Sphinx}}
\label{\detokenize{comandos_mas_usados/comandos:comando-para-iniciar-un-proyecto-sphinx}}
\begin{sphinxVerbatim}[commandchars=\\\{\}]
\PYG{n}{sphinx}\PYG{o}{\PYGZhy{}}\PYG{n}{quickstart}
\end{sphinxVerbatim}
\begin{itemize}
\item {} 
\sphinxAtStartPar
Este comando inicializa un proyecto Sphinx, creando la estructura básica con carpetas como \sphinxcode{\sphinxupquote{docs}}, \sphinxcode{\sphinxupquote{source}}, y archivos como \sphinxcode{\sphinxupquote{conf.py}}, \sphinxcode{\sphinxupquote{index.rst}}, etc.

\end{itemize}


\bigskip\hrule\bigskip



\subsubsection{\sphinxstylestrong{Comando para actualizar la versión de Sphinx}}
\label{\detokenize{comandos_mas_usados/comandos:comando-para-actualizar-la-version-de-sphinx}}
\begin{sphinxVerbatim}[commandchars=\\\{\}]
\PYG{n}{pip} \PYG{n}{install} \PYG{o}{\PYGZhy{}}\PYG{n}{U} \PYG{n}{sphinx}
\end{sphinxVerbatim}
\begin{enumerate}
\sphinxsetlistlabels{\arabic}{enumi}{enumii}{}{.}%
\item {} 
\sphinxAtStartPar
\sphinxstylestrong{\sphinxcode{\sphinxupquote{pip install}}}: Instala un paquete de Python (en este caso, \sphinxcode{\sphinxupquote{sphinx}}) desde el repositorio de PyPI (Python Package Index).

\item {} 
\sphinxAtStartPar
\sphinxstylestrong{\sphinxcode{\sphinxupquote{\sphinxhyphen{}U}} o \sphinxcode{\sphinxupquote{\sphinxhyphen{}\sphinxhyphen{}upgrade}}}: Actualiza el paquete a la versión más reciente disponible en PyPI.

\end{enumerate}


\subsection{\sphinxstylestrong{4. Instalación de dependencias necesarias}}
\label{\detokenize{comandos_mas_usados/comandos:instalacion-de-dependencias-necesarias}}

\subsubsection{\sphinxstylestrong{Instalar Sphinx}}
\label{\detokenize{comandos_mas_usados/comandos:instalar-sphinx}}
\begin{sphinxVerbatim}[commandchars=\\\{\}]
\PYG{n}{pip} \PYG{n}{install} \PYG{n}{sphinx}
\end{sphinxVerbatim}


\subsubsection{\sphinxstylestrong{Instalar MyST\sphinxhyphen{}Parser}}
\label{\detokenize{comandos_mas_usados/comandos:instalar-myst-parser}}
\begin{sphinxVerbatim}[commandchars=\\\{\}]
\PYG{n}{pip} \PYG{n}{install} \PYG{n}{myst}\PYG{o}{\PYGZhy{}}\PYG{n}{parser}
\end{sphinxVerbatim}
\begin{itemize}
\item {} 
\sphinxAtStartPar
Habilita el soporte para archivos Markdown en Sphinx.

\end{itemize}


\subsubsection{\sphinxstylestrong{Instalar un tema}}
\label{\detokenize{comandos_mas_usados/comandos:instalar-un-tema}}
\begin{sphinxVerbatim}[commandchars=\\\{\}]
\PYG{n}{pip} \PYG{n}{install} \PYG{n}{sphinx\PYGZus{}rtd\PYGZus{}theme}
\end{sphinxVerbatim}
\begin{itemize}
\item {} 
\sphinxAtStartPar
Instala el tema utilizado en la documentación.

\end{itemize}


\bigskip\hrule\bigskip



\subsection{\sphinxstylestrong{4. Generación de documentación}}
\label{\detokenize{comandos_mas_usados/comandos:generacion-de-documentacion}}

\subsubsection{\sphinxstylestrong{Para generar documentación en formato HTML}}
\label{\detokenize{comandos_mas_usados/comandos:para-generar-documentacion-en-formato-html}}
\begin{sphinxVerbatim}[commandchars=\\\{\}]
\PYG{n}{make} \PYG{n}{html}
\end{sphinxVerbatim}
\begin{itemize}
\item {} 
\sphinxAtStartPar
Genera la documentación en HTML. El resultado se encuentra en la carpeta \sphinxcode{\sphinxupquote{docs/build/html}}.

\end{itemize}


\subsubsection{\sphinxstylestrong{Para limpiar los archivos generados}}
\label{\detokenize{comandos_mas_usados/comandos:para-limpiar-los-archivos-generados}}
\begin{sphinxVerbatim}[commandchars=\\\{\}]
\PYG{n}{make} \PYG{n}{clean}
\end{sphinxVerbatim}
\begin{itemize}
\item {} 
\sphinxAtStartPar
Elimina todos los archivos generados previamente (HTML, PDF, etc.).

\end{itemize}


\bigskip\hrule\bigskip



\subsubsection{\sphinxstylestrong{Actualizar dependencias}}
\label{\detokenize{comandos_mas_usados/comandos:actualizar-dependencias}}
\begin{sphinxVerbatim}[commandchars=\\\{\}]
\PYG{n}{pip} \PYG{n}{install} \PYG{o}{\PYGZhy{}}\PYG{o}{\PYGZhy{}}\PYG{n}{upgrade} \PYG{n}{sphinx} \PYG{n}{myst}\PYG{o}{\PYGZhy{}}\PYG{n}{parser}
\end{sphinxVerbatim}


\bigskip\hrule\bigskip


\sphinxstepscope


\section{src}
\label{\detokenize{modules:src}}\label{\detokenize{modules::doc}}
\sphinxAtStartPar
.. toctree::
:maxdepth: 4

\sphinxAtStartPar
math\_operations

\sphinxstepscope


\section{math\_operations module}
\label{\detokenize{math_operations:math-operations-module}}\label{\detokenize{math_operations::doc}}
\sphinxAtStartPar
.. automodule:: math\_operations
:members:
:undoc\sphinxhyphen{}members:
:show\sphinxhyphen{}inheritance:



\renewcommand{\indexname}{Índice}
\printindex
\end{document}